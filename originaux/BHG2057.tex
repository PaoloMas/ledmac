

\titulus{Περίοδοι καὶ μαρτύριον τῶν ἁγίων ἀποστόλων
Βαρθολομαίου καὶ Βαρνάβα.}

Καὶ τὸ τῶν ἄλλων ἁγίων τοὺς βίους διεξιέναι καὶ
τὰ παλαίσματα, βιωφελὲς ἅμα καὶ δίκαιον˙ τὸ δὲ καὶ
τῶν ἀποστόλων τὰς περιόδους, ναὶ δὴ καὶ τοὺς ἀγῶνας\CCSG{5}
παραδιδόναι γραφῇ, πλέον εἰς εὐφροσύνης ὑπερβολὴν
ἔστι τὲ καὶ νομίζεται. Διὸ δὴ καὶ τὰ τῶν σήμερον
προτεθέντων τούτων, Βαρθολομαίου καὶ Βαρνάβα φημί,
διεξελθεῖν ἀναγκαῖον ἐν ἐπιτόμῳ συνεῖδον, ἵν ἔχοιεν οἱ
τῶν καλῶν ἐρασταὶ σαφῶς εἰδέναι, ὅπου τὲ ἕκαστος\CCSG{10}
τὸν λόγον ἐκήρυξε καὶ οἵῳ τέλει τὸν βίον κατέλιπεν˙
ἀρκτέον δὲ οὕτως τῆς διηγήσεως.

Βαρθολομαῖος ὁ πάντιμος, ὁ τῆς εὐσεβείας διδάσκαλος
καὶ τῶν Χριστοῦ φύλαξ ἐντολῶν ἀκριβέστατος, ὁ τῆς
δωδεκάδος τῶν ἀποστόλων σύγκληρος καὶ τῶν πλανω\CCSG{15}%
μένων ἀπλανὴς ὁδηγός, τῆς Εὐδαίμονος Ἀραβίας κήρυξ
γνωρίζεται καὶ διδάσκαλος˙ αὕτη γὰρ αὐτῷ κεκλήρωτο
παρὰ τῆς τοῦ Πνεύματος χάριτος. Ἣν καὶ καταλαβών,
καὶ τὸν σπόρον καταβάλλων τῆς πίστεως, οὐκ ἔστιν
εἰπεῖν ὅσοις ὑπεβλήθη παρ’ αὐτῶν τοῖς κακοῖς, νῦν μὲν\CCSG{20}
διωκόμενος, νῦν δὲ τυπτόμενος, ἄλλοτε πάλιν καὶ κα%
θειργνύμενος, λιμῷ τὲ καὶ δίψει καταπονούμενος, κρυ%
μῷ καὶ καύσωνι προσταλαιπωρούμενος, καὶ τέλος πᾶ%
σαν ὡς εἰπεῖν δεχόμενος κάκωσιν, πᾶσαν αἰκίαν, πᾶν
εἶδος χαλεπῆς τιμωρίας˙ ὅμως ὀψέποτε πολλοὺς φωτίσας\CCSG{25}
τῷ θείῳ βαπτίσματι καὶ πολλοὺς πρὸς τὴν ἐπίγνωσιν
ἑλκύσας τῆς ἀληθείας, τὸ κατὰ Ματθαῖον αὐτοῖς καλῶς
εὐαγγέλιον παραδέδωκε, τῇ ἰδίᾳ τοῦτο γεγραφὼς δια%
λέκτῳ, καὶ δὴ καὶ ἐκκλησίας αὐτοῖς συστησάμενος καὶ
πρεσβυτέρους ταύταις καταστησάμενος.\CCSG{30}

Ἐπεὶ οὖν καλῶς εἶχεν αὐτῷ ταῦτα καὶ κατὰ σκοπὸν
τὸν προσήκοντα, τούτων ἀπαναστάς, ἐν Ἀλιβανίᾳ πόλει
τῆς Μεγάλης Ἀρμενίας ἀπαίρει, καταγγέλλων κἀν ταύτῃ
τὸ τοῦ Χριστοῦ εὐαγγέλιον, πολλοὺς ἐνάγων πρὸς τὴν
ἀλήθειαν, πολλαῖς πρότερον προσταλαιπωρήσας κἀνταῦθα\CCSG{35}
καὶ ἀναριθμήτοις ταῖς θλίψεσιν. Ἐκεῖθεν δὲ πρὸς Ἀρ%
βανούπολιν ἔρχεται καὶ διδάσκει θεὸν τὸν Χριστὸν καὶ
τῶν ὅλων δημιουργόν˙ ἐν ᾗ πλεῖστα παθὼν τὰ δεινά,
τέλος καὶ σταυρῷ παραδίδοται καὶ οὕτω τὸν βίον ἀλ%
λάττεται.\CCSG{40}

Ἀλλ’ οὐκ ἔστησαν μέχρι τούτου τὰ τοῦ πονηροῦ σπέρ%
ματα, οἱ τῶν δαιμόνων θεραπευταί, ἀλλὰ δὴ καὶ μολιβδίνῃ
λάρνακι τοῦτον ἐμβαλόντες οἱ ἄθεοι, τῇ θαλάσσῃ δεδώ%
κασιν. Θεὸς δὲ μὴ ἀμελῶν τοῦ ἰδίου θεράποντος, διασώ%
ζει ταύτην μέχρις αὐτῆς Σικελίας˙ καὶ γὰρ ἐν Λιπάρῳ\CCSG{45}
τῇ νήσῳ τῆς λάρνακος ἐκβρασθείσης, οἱ κάτοικοι Χρι%
στιανοί, διαγνόντες ὅ τι καὶ εἶεν, θείας ἀποκαλύψεως
γνωρισάσης αὐτοῖς ὡς ὁ Βαρθολομαίου νεκρός ἐστι,
φιλοτίμως αὐτὸν κατατίθενται καὶ σεμνῶς. Καὶ ταῦτα μὲν
εἰς τοσοῦτον τὰ περὶ τούτου.\CCSG{50}

Ὁ δέ γε καλὸς Βαρνάβας, ὁ τῷ χορῷ τῶν ἑβδομή%
κοντα φαιδρὸς ἐκλάμπων ὡς ἥλιος, τὸ τῶν Κυπρίων
σεμνὸν ἐγκαλλώπισμα, τῆς Λευῖ φυλῆς καὶ οὗτος
ἐξώρμητο, Σαμουὴλ ἀπὸ τοῦ προφήτου σαφῶς κατάγων
τὸ γένος. Καὶ γὰρ οἱ τούτου πρόγονοι πολέμων ἐν\CCSG{55}
περιστάσεσι τὴν Κύπρον κατειληφότες, φιλοτίμως ᾤκουν
αὐτήν, ἐπιφανέστατοι καὶ τῶν εὐπόρων τυγχάνοντες. Ὅθεν
καί, ὥς φησι Κλήμης ὁ Στρωματεύς, περικαλλέστατον
ὅτι κτῆμα τοῖς Ἱεροσολύμοις καταλελοίπεσαν, οὗ τὸ
δίκαιον τῷ Βαρνάβᾳ κατεκληρώσαντο, κατὰ τὴν Κύπρον\CCSG{60}
τεχθέντι καὶ παιδὶ χρηματίζοντι. Τοῦτον οὖν οἱ γονεῖς
ἄρτι πρόσηβον ὄντα πρὸς Ἱεροσόλυμα στείλαντες, τοὺς
προφήτας παιδεύεσθαι καὶ τὸν νόμον ἔσπευδον˙ ἃ δὴ
καὶ ἐξεπαιδεύετο παρὰ τοὺς πόδας Γαμαλιήλου---νο%
μοδιδάσκαλος οὗτος---, ἔχων συμφοιτητὴν τὸν ἀπόστολον\CCSG{65}
Παῦλον, Σαῦλον τέως ὀνομαζόμενον. Οὕτως οὗν ἐξε%
παιδεύθη τὸν νόμον, ὡς μὴ δέεσθαι βίβλων. Ἄρτι δὲ
τὸν παράλυτον ὁ σωτὴρ ἰασάμενος κατὰ τὴν προβατικήν,
εἶχε θεατὴν τοῦ θαύματος τὸν Βαρνάβαν. Διὸ καὶ πρὸς
τὰ βάθη τῆς αὐτοῦ καρδίας ἀποβλεψάμενος, καὶ λόγου\CCSG{70}
μετέδωκε καὶ παρρησίας τῆς πρὸς αὐτόν.

Ὁ τοίνυν Βαρνάβας τὴν οἰκίαν Μαρίας τῆς μητρὸς
Ἰωάννου, ὃς ἐπελέγετο Μάρκος, καταλαβών --- θείαν δὲ
τούτου ταύτην εἶχεν ὁ λόγος ---, φησὶ πρὸς αὐτήν˙ \enquote{Ὁ
μέλλων ἔρχεσθαι, ὃν ἐπεθύμουν ἰδεῖν οἱ πατέρες ἡμῶν,\CCSG{75}
ἰδοὺ παραγέγονεν.} Καὶ ἅμα καταλιποῦσα πάντα, κατέλαβε
τὸν ναὸν καὶ Χριστῷ προσπεσοῦσα, πρὸς τὸν οἶκον
αὐτῆς εἰσελθεῖν αὐτὸν παρεκάλει. Καὶ δὴ ἀφίκετο, καὶ
ἔστιν οὗτος, ἐν ᾧ τὸ Πάσχα μετὰ τῶν μαθητῶν ἐπε%
τέλεσεν˙ λόγος δὲ ἄνωθεν εἰς ἡμᾶς καταβέβηκεν, ὡς\CCSG{80}
ὁ τὸ κεράμιον βαστάζων τοῦ ὕδατος, ᾧ κατακολουθεῖν
ὁ Χριστὸς τοὺς μαθητὰς ἐπέτρεψε, Μάρκος ἦν ὁ σοφός,
ὁ τῆς Μαρίας ταύτης υἱός. Ἐν δὴ τῷ τοιούτῳ οἴκῳ
καὶ μετὰ τὴν ἀνάληψιν ἀνῆλθον ὑποστρέψαντες οἱ μαθηταὶ
μετὰ τῶν λοιπῶν ἀδελφῶν, ἐν οἷς ἦν καὶ Βαρνάβας καὶ\CCSG{85}
Μάρκος˙ καὶ γὰρ ὁ Βαρνάβας ἀπὸ Ἱεροσολύμων εἰς
τὴν Γαλιλαίαν πορευομένῳ τῷ Χριστῷ ἠκολούθησε καὶ
οὕτω τοῖς ἑβδομήκοντα συνηρίθμητο, παρακλήσεως υἱὸς
ὑπὸ πάντων λεγόμενος. Ἀκούσας γὰρ τοῦ κυρίου λέ%
γοντος˙ \enquote{Πωλήσατε τὰ ὑπάρχοντα ὑμῶν} καὶ τὰ ἑξῆς,\CCSG{90}
μηδὲν με\textins{λ}λήσας, τὴν ὑπολειφθεῖσαν τῶν γεννητόρων
οὐσίαν, πολύτιμον οὖσαν, τοῖς ἐνδεέσι διαπωλήσας διέ%
νειμε, μόνον τὸν ἐν Ἱεροσολύμοις ἀγρὸν εἰς διατροφὴν
ἰδίαν καταλιπών˙ ὃν καὶ αὐτὸν μετὰ τὴν τοῦ ἀγαθοῦ
Πνεύματος κάθοδον τιμῆς ἀποδόμενος, παρὰ τοὺς πόδας\CCSG{95}
αὐτὴν φέρων τῶν ἀποστόλων τέθηκεν.


Ἐλάλει δὲ καὶ συνεζήτει μετὰ τοῦ Σαύλου, βουλόμενος
ἀγαγεῖν αὐτὸν πρὸς τὴν τοῦ κυρίου ἐπίγνωσιν. Σαῦλος
δὲ τοῦ μὲν Βαρνάβα κατεγέλα, πρὸς δὲ τὸν Χριστὸν
ἐβλασφήμει. Βλέπων δὲ καὶ τὰ παρὰ τῶν ἀποστόλων\CCSG{100}
γινόμενα θαύματα καὶ τὸ πλῆθος προστιθέμενον τῷ λόγῳ
τῆς πίστεως, ἐδάκνετο τὴν ψυχήν, καὶ προσβαλὼν τῷ
Στεφάνῳ καὶ μὴ δυνηθεὶς ἀντιστῆναι τῇ σοφίᾳ καὶ
τῷ Πνεύματι, ᾧ ἐλάλει, πρὸς ὀργὴν καὶ τοῦτον ἀνελὼν
διὰ τῶν ἀτάκτων, ἤγειρε διωγμὸν ἐν Ἱεροσολύμοις κατὰ\CCSG{105}
τῆς ἐκκλησίας μέγαν. Μετὰ γοῦν τὸ πηρωθῆναι τὰς
ὄψεις ὁ Βαρνάβας αὐτῷ συναντήσας˙ \enquote{Ἵνα τί τὸν εὐερ%
γέτην} ἔφη \enquote{διώκεις; Παῦσαι πορθῶν τὴν ἐκκλησίαν
Χριστοῦ.} Οὕτως εἶπε, καὶ ὁ Παῦλος μετάμελον λαβὼν
τῶν πρακτέων, συγγνώμην ᾔτει καὶ θεὸν τὸν Χριστὸν\CCSG{110}
ὡμολόγει. Τότε ὁ μέγας Βαρνάβας ἐπιλαβόμενος αὐτοῦ
τῆς χειρός, ἤγαγε πρὸς τοὺς ἀποστόλους, πρόβατον
αὐτὸν ἀντὶ λύκου τούτοις γνωρίσας. Τότε διηγήσατο
αὐτοῖς ὁ Παῦλος πῶς ἐν Δαμασκῷ κατὰ τὴν ὁδὸν τὸν
κύριον εἶδε˙ καὶ ἦν σὺν αὐτοῖς διδάσκων τὸν λόγον\CCSG{115}
καὶ κηρύσσων ἐν Ἱερουσαλήμ. Οἱ τοίνυν Ἰουδαῖοι ἀνελεῖν
αὐτὸν ἐπεχείρουν˙ ἀλλ’ οἱ ἀπόστολοι πρὸς τὴν ἰδίαν ἐξέ%
πεμψαν αὐτὸν πατρίδα κηρύττειν καὶ διδάσκειν τὸ εὐαγ%
γέλιον˙ ἔτι μὴν καὶ τὸν Βαρνάβαν, ὡς δυνατόν, τὴν
ἐκεῖσε ποιμαίνειν ἁγιωτάτην ἐκκλησίαν ἐπέτρεψαν. Καὶ\CCSG{120}
τούτου γενομένου, πολὺς προσετέθη τῷ κυρίῳ λαός.

Ἐκεῖθεν οὖν ἐξελθών, τὰς πόλεις πάσας περιῄει καὶ
χώρας, καὶ μέχρις αὐτῆς κατήντησε Ῥώμης˙ αὐτὸς γὰρ
πρὸ παντὸς ἄλλου ἐκήρυξεν ἐν Ῥώμῃ τὸ εὐαγγέλιον.
Εἶτα λάθρᾳ φυγών, τὴν Ἀλεξάνδρου καταλαμβάνει. Ἐκεῖθεν\CCSG{125}
πάλιν εἰς Ἀντιόχειαν ἔρχεται καὶ ἐξ αὐτῆς εἰς Ταρσὸν
παραγίνεται. Καὶ τὸν Παῦλον εὑρηκὼς ἐν αὐτῇ καὶ εἰς
Ἀντιόχειαν ἀγαγών, ἐνιαυτὸν ὅλον ἐτέλεσε, διδάσκων
σὺν αὐτῷ τὸν λαόν. Εἶτα μετὰ δεκατέσσαρα ἔτη, καθὼς
αὐτὸς φησί, τοῦ σωτηρίου πάθους, αὖθις κατέλαβον τὰ\CCSG{130}
Ἱεροσόλυμα καὶ λαβόντες δεξιὰς παρὰ τῶν ἀποστόλων,
ὡς ἂν αὐτοὶ μὲν εἰς τὰ ἔθνη κηρύξωσιν, οἱ δὲ περὶ
Πέτρον εἰς τὴν περιτομήν, κατῆλθον εἰς Ἀντιόχειαν,
ἐπαγόμενοι καὶ τὸν Μάρκον. Ἀπὸ δὲ ταύτης εἰς Κύπρον
ἦλθον καὶ διοδεύσαντες αὐτὴν ὅλην ἀπὸ Σαλαμῖνος καὶ\CCSG{135}
μέχρι τῆς Πάφου, ὅτε καὶ τὸν Ἐλύμαν ἐτύφλωσαν καὶ
ἐφώτισαν τὸν ἀνθύπατον, καὶ μαθητεύσαντες ἱκανούς,
ἀναχθέντες ἐκεῖθεν ἦλθον εἰς Παμφυλίαν. Ὁ τοίνυν Μάρ%
κος, ἅτε νέος ὢν καὶ ἀντέχειν πρὸς τοὺς ἀγῶνας μὴ
σθένων, πρὸς τὴν μητέρα κατὰ τὰ Ἱεροσόλυμα γίνεται.\CCSG{140}
Ὡς δὲ καὶ πάλιν ὁ Βαρνάβας καὶ Παῦλος ταῦτα κα%
τέλαβον, ἰδὼν αὐτοὺς ὁ Μάρκος ὑπὸ πάντων ἀγαπω%
μένους καὶ μετὰ τοσούτους κινδύνους καὶ μάστιγας
ἐρρωμένους καὶ ὑγιεῖς, πρόσεισι τῷ Βαρνάβᾳ --- καὶ γὰρ
ᾐδεῖτο τὸν Παῦλον --- καὶ δάκρυσι τὴν συγχώρησιν ᾔ-\CCSG{145}
τει. Ὁ δὲ \enquote{Τὸ θέλημα τοῦ κυρίου γενέσθω} πρὸς αὐτὸν
ἔφη, \enquote{μόνον σὺ ἕτοιμος ἔσο τὰς ὑποσχέσεις ποιεῖν.}
Μετὰ ταῦτα προσελθὼν Βαρνάβας τῷ Παύλῳ, παρεκάλει
συνέκδημον αὐτοῖς γενέσθαι τὸν Μάρκον, ἀλλ’ ὁ Παῦλος
οὐκ ἠβούλετο τοῦτο, τοῦ θεοῦ πάντως κρεῖττον περὶ\CCSG{150}
τοῦ Μάρκου προβλεπομένου. Ἐντεῦθεν ἐγένετο τοῖς ἀ%
ποστόλοις παροξυσμός, οὐκ ὀργῆς ἐμπεσούσης, ἄπαγε
 --- πῶς γὰρ οἱ εἰρηνοποιοὶ καὶ τοῦ Χριστοῦ μαθηταὶ
τοῦ εἰρηνικοῦ τούτῳ εἶχον ἁλῶναι τῷ πάθει;  ---, ἀλλὰ
παροξυσμὸς ζήλου θεϊκοῦ˙ ὁ μὲν γὰρ ἀκρίβειαν ἐπεζήτει,\CCSG{155}
ἀποστολικῇ πρέπουσαν τελειότητι, ὁ δὲ Βαρνάβας ἐτίμα
καλῶς τὸ φιλάνθρωπον. Διὸ καὶ τὸν Μάρκον λαβών,
εἰς Κύπρον ἀπέπλευσε καὶ τὴν Σαλαμῖνα κατέλαβε, τοῦ
θεοῦ κηρύσσων τὴν βασιλείαν. Σκόπει δέ μοι πῶς ἡ
τῶν ἀποστόλων διαίρεσις πλείονας ἐποίει τοὺς προσιόντας\CCSG{160}
Χριστῷ, ὅπερ ἦν τῆς προμηθείας θεοῦ τὸ ἐξαίρετον˙
ἄλλοθεν μὲν γὰρ ὁ Παῦλος τούτους συνῆγε καὶ ἄλλοθεν
ὁ Βαρνάβας˙ εἶδες παροξυσμὸν ζήλου θείου καὶ τρόπον
εὐμήχανον;


Ἀλλὰ τοῦ Βαρνάβα τῇ Σαλαμῖνι προσμένοντος, ἐκ τῆς\CCSG{165}
Συρίας Ἰουδαῖοι παραγενόμενοι διέσυραν αὐτόν, μὴ ἀλη%
θεύειν ἐν οἷς διδάσκει, τὸν λαὸν ἀναπείθοντες. Γνοὺς
οὖν ὁ ἀπόστολος τὴν ἐπιβουλὴν καὶ ὅτι μέλλοι τελευτᾶν,
ἄρτον λαβὼν καὶ ποτήριον, καὶ τὴν μυστικὴν ἐκτελέσας
παράδοσιν, τῶν ἀχράντων μυστηρίων μετέλαβε, μεταδοὺς\CCSG{170}
αὐτῶν καὶ τοῖς ἀδελφοῖς. Εἶτα τὸν Μάρκον παραλαβὼν
κατ’ ἰδίαν, γνωρίζει τούτῳ τὴν αὐτοῦ τελευτὴν καὶ
ἐπισκήπτει τῆς πόλεως αὐτὸν ἐξελθόντα κατὰ δυσμὰς
εὑρεῖν αὐτοῦ τὸ σῶμα καὶ θάψαι, ἀλλὰ δὴ καὶ πρὸς
τὸν Παῦλον ἀπελθεῖν μετὰ ταῦτα καὶ τούτῳ συνεῖναι\CCSG{175}
καὶ κηρύσσειν τὸν λόγον. \enquote{Μέλλεις γάρ φησιν ἐπὶ πᾶσαν
μεγαλυνθῆναι τὴν γῆν.} Ἔφη καὶ τὴν συναγωγὴν ὁ
Βαρνάβας κατέλαβε, πείθων τοὺς ὄχλους τοῖς θεοπνεύστοις
λόγοις, ὡς Χριστός, αὐτός ἐστιν ὁ τοῦ θεοῦ τοῦ ζῶντος
Υἱός. Πλησθέντες οὖν θυμοῦ οἱ ἀπὸ Συρίας Ἑβραῖοι\CCSG{180}
ἐπιβάλλουσι τὰς χεῖρας ἐπ’ αὐτὸν καὶ οἴκῳ σκοτεινῷ
κατακλείουσιν. Εἶτα καθ’ ἑσπέραν βαθεῖαν ἐξαγαγόντες καὶ
βασανίσαντες αὐτὸν ἱκανῶς, ἔξω τῆς πόλεως ἄγουσι˙
κἀκεῖ τοῦτον λίθοις ὡς τὸν πρώταθλον βάλλουσι. Καὶ
ὅρα μοι τῆς αὐτῶν κακίας τὴν ὑπερβολήν˙ καὶ πυρὰν\CCSG{185}
ἀνάψαντες, μέσον αὐτῆς ἀκοντίζουσι τοῦτον, εἰ καὶ προ%
νοίᾳ θεοῦ συνετηρήθη ἀκέραιος. Ὁ Μάρκος οὖν κατὰ
τὰ διατεταγμένα ποιήσας, συγκομίζει μετά τινων ἀδελφῶν
τὸ λείψανον τοῦ Βαρνάβα καὶ θάπτει φιλοτίμως αὐτό.
Καὶ οὕτως ἐκπλεύσας ἔρχεται πρὸς Παῦλον εἰς Ἔφεσον\CCSG{190}
καὶ διηγεῖται τούτῳ περὶ τῆς τοῦ Βαρνάβα τελειώσεως.
Ἔνθεν τοι καὶ δακρύσας ὁ Παῦλος, κατέσχε τὸν Μάρκον
παρ’ ἑαυτῷ. Μετὰ δὲ ταῦτα πρὸς Ῥώμην ὁ Πέτρος
ἀπαίρων, συμπαρέλαβε καὶ τὸν Μάρκον, τρόπον τινὰ
τοῦτον υἱοποιήσας. Ἐκεῖσε τοίνυν γνώμῃ τοῦ πρωτο-\CCSG{195}
θρόνου συγγραψάμενος ὁ Μάρκος τὸ εὐαγγέλιον καὶ μέγας
ἀναφανείς, πρόεδρος Ἀλεξανδρείας ὑπὸ τούτου χειρο%
τονεῖται καὶ πρὸς ταύτην ἐκπέμπεται˙ ἐν ᾗ καὶ χρόνους
ἐννέα τὸν λόγον διδάξας τῆς πίστεως, μαρτυρικῶς τε%
τελείωται. Καὶ ταῦτα μὲν εἰς τοσοῦτον.\CCSG{200}


Κατὰ δὲ τοὺς χρόνους Ζήνωνος τοῦ βασιλέως τὸ ἀ%
ποστολικὸν ἐκεῖνο σῶμα τοῦ καλοῦ Βαρνάβα τοιούτῳ πε%
φανέρωται τρόπῳ. Τῷ τῆς Σαλαμῖνος ἐπισκόπῳ --- Ἀν%
θέμιος οὗτος ἦν ὁ κλεινός --- νύκτωρ ἐπιφανεὶς ὁ ἀ%
πόστολος, \enquote{Ἄγε δὴ} πρὸς αὐτὸν εἶπεν, \enquote{Ἀνθέμιε, πο-\CCSG{205}
ρεύθητι τῆς πόλεως ἔξω κατὰ δυσμὰς εἰς τόπον σταδίους
αὐτῆς ἀπέχοντα πέντε καὶ λεγόμενον τῆς ὑγείας, ὅπου
δὴ καὶ θαύματα δι’ ἐμοῦ τελεῖται συχνά˙ καὶ ὄρυξον ἐπὶ
τὴν κερατέαν, καὶ εὑρήσεις σπήλαιον καὶ λάρνακα τούτῳ
ἐναποτεθειμένην, κἀν ταύτῃ μου τὸ σῶμα κείμενον καὶ\CCSG{210}
τὸ τοῦ Χριστοῦ εὐαγγέλιον, ὅπερ ἀπὸ Ματθαίου τοῦ
εὐαγγελιστοῦ αὐτὸς ἰδιοχείρως ἐξελαβόμην.} Καὶ ὁ ἐπί%
σκοπος \enquote{Καὶ τίς εἶ, κύριέ μου} φησίν, \enquote{ἵνα ἔχω σαφῶς
εἰδέναι, τίνος ἄρα τὸ λείψανον;} Καὶ ὃς \enquote{Βαρνάβας ἐγώ}
φησιν \enquote{ὁ τοῦ κυρίου ἀπόστολος.} Αὐτίκα γοῦν τὸν\CCSG{215}
τόπον καταλαβὼν καὶ πιστῶς διορύξας, εὑρίσκει τὴν
λάρνακα, τό τε ἀποστολικὸν ἐν αὐτῇ σῶμα --- εἶπες ἂν
ἄρτι ταφῆναι ---, καὶ τὸ εὐαγγέλιον, τῷ στήθει τούτου
προσεπικείμενον˙ ἣν καὶ μολιβδίναις σφραγῖσιν ἀσφαλι%
σάμενος, τάχος τὴν βασιλεύουσαν ἔφθασε καὶ τῷ βασιλεῖ\CCSG{220}
διηγεῖται πάντα κατὰ λεπτόν. Ἐκεῖνος δέ, τῷ θεῷ μεγάλα
δοξάσας ἐπ’ αὐτῷ, τὸ εὐαγγέλιον καὶ μόνον ἐξαιτεῖται
λαβεῖν˙ ὃ δὴ καὶ δεξάμενος, χρυσίῳ κατακοσμεῖ καὶ
τῷ παλατίῳ κατατίθησι˙ καὶ μένει μέχρι τοῦ νῦν φυ%
λαττόμενον˙ κατὰ γὰρ τὴν μεγάλην ὥς φασι πέμπτην ἐν\CCSG{225}
αὐτῷ τὸ εὐαγγέλιον τῆς ἡμέρας ἀναγινώσκεται.

 Καὶ ταῦτα μέν σου τὰ διὰ Χριστὸν ἄθλα, πανθαύμαστε,
καὶ οἱ ἀγῶνες καὶ τὰ παλαίσματα. Σὺ δὲ καὶ ἔτι
πρεσβείας ὑπὲρ τοῦ κόσμου ποιούμενος, ἀξιόθεε, αἴτησαι
καὶ βασιλέως ἡμῶν ὑπὲρ τοῦ καλοῦ καὶ τὰ πάντα\CCSG{230}
χρηστοῦ, δοθῆναι τούτῳ


\stanza
μήκιστον ζωὴν καὶ παθῶν ὑπερτέραν,&
ἰλύος πάσης ὅλως ἀπηλλαγμένην,&
χειρὶ σκεπομένην τε τοῦ θεοῦ λόγου,&
ἀπαλείφουσαν ἐθνῶν ἄπειρα πλήθη,\CCSG{235}&
ἡμεροῦσαν δὴ χώρας αὐτῶν καὶ πόλεις,&
λαμπροῖς τροπαίοις καὶ νίκαις εὐπρεπέσι,&
πάντων ἀγαθῶν οὐρανίων τὴν δόσιν\&
\noindent καὶ βασιλείας θεοῦ τὴν κληρουχίαν˙ ὅτι αὐτῷ Χρι%
στῷ τῷ θεῷ ἡμῶν πρέπει ἡ δόξα, νῦν καὶ εἰς τοὺς\CCSG{240}
ἀτελευτήτους αἰῶνας ἀμήν.

