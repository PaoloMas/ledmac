

%\ledsection{\cite{BHG226}}

\titulus{Ἀλεξάνδρου μοναχοῦ ἐγκώμιον εἰς τὸν ἅγιον
Βαρνάβαν τὸν ἀπόστολον, προτραπέντος ὑπὸ τοῦ
πρεσβυτέρου καὶ κλειδούχου τοῦ σεβασμίου αὐτοῦ
ναοῦ, ἐν ᾧ ἱστορεῖται καὶ ὁ τρόπος τῆς ἀποκαλύψεως τῶν ἁγίων αὐτοῦ λειψάνων.}\CCSG{5}

Μεγίστην λόγων ὑπόθεσιν προέθετο τοῖς πτωχοτάτοις
ἡμῖν ἡ πατρικὴ ὑμῶν φιλομαθία, ὦ πατέρων ἄριστε καὶ
ἀσκητῶν δοκιμώτατε πάτερ˙ προεβάλετο γὰρ ἡμῖν εἰς
εὐφημίαν τὸν θεσπέσιον καὶ ὡς ἀληθῶς τρισμακάριον
καὶ ἐν ἀποστόλοις ἐπισημότατον υἱὸν τῆς παρακλήσεως\CCSG{10}
καὶ φωστῆρα τῆς οἰκουμένης, Βαρνάβαν τὸν πολυΰμνητον.
Ἀλλ’ ἐγὼ πρὸς ταύτην τὴν ὑπερλίαν ὑπόθεσιν τὴν ἐμαυτοῦ
ἀπαιδευσίαν ἀντεξάγων, τὴν ὑπακοὴν ἐπὶ πλεῖστον χρόνον
παρῃτησάμην, τὴν ἐγχείρησιν ὀρρωδῶν˙ ποῖος γὰρ καὶ
λόγος ἐφικέσθαι δύναται τῆς ἀποστολικῆς τελειότητος;\CCSG{15}
Οὐ μόνον γὰρ ἐμοὶ ἀδύνατον τὸ τοιοῦτον καθέστηκεν
ἐγχείρημα, ἀλλὰ καὶ τοῖς πολλοῖς δυσχερές˙ οἶμαι γὰρ
πᾶσαν ἀνθρώπων φύσιν λείπεσθαι πρὸς διήγησιν τῶν
τοῦ μεγίστου ἀποστόλου Βαρνάβα ἀγαθῶν πράξεων˙ εἰ
δὲ καὶ ἐπιχειρήσειέ τις περὶ αὐτοῦ λέγειν τί, παρὰ\CCSG{20}
πολὺ μὲν τοῦ μέτρου τῆς ἀξίας τῶν πραγμάτων ἀπο%
λειφθήσεται, κἂν λίαν σοφὸς καὶ ἐπιστήμων τυγχάνῃ καὶ
περὶ λόγους δυνατώτατος, εὑρεθήσεται δὲ βυθῷ θαυμάτων
ἐγκυβιστούμενος. Ἄλλος μὲν οὖν ἄλλο τι τῶν δικαίων
ἕκαστος παρὰ τοῦ Πνεύματος τοῦ ἁγίου εἴληφε χάρισμα,\CCSG{25}
οἱ δὲ ἀπόστολοι πάντα ὁμοῦ τῶν χαρισμάτων ὑπεδέξαντο
τὸν θησαυρὸν καὶ πᾶσαν ἀρετὴν ἐπεδείξαντο. Πῶς οὖν
ὁ τάλας ἐγώ, ὑπὸ μυρίων παθῶν καταπεποντισμένος,
δυνήσομαι τὴν ἀποστολικὴν θάλασσαν διανήξασθαι; Ἰσ%
χνόφωνος γὰρ γραφικῶς εἰπεῖν καὶ βραδύγλωσσος, καὶ\CCSG{30}
οὐκ εὔλαλός εἰμι ἐγώ, ἵνα τὰς ὑπὲρ ἄνθρωπον ἀρετὰς
τοῦ θεσπεσίου ἀποστόλου Βαρνάβα διηγήσωμαι˙ πάσας
γὰρ ὁμοῦ οὗτος τὰς ἀρετὰς κατώρθωσεν ἀκριβῶς, ὡς
οὐδεμίαν τῶν ἄλλων οὐδείς. Διατοῦτο ἀνεβαλόμην ἐπὶ
πολὺ τὴν ὑπακοήν, εἰδὼς τὴν ἐμαυτοῦ ἀναξιότητα μὴ\CCSG{35}
ἐπαρκοῦσαν πρὸς διήγησιν τῶν τοῦ ἀνδρὸς εὐκλεεστάτων
κατορθωμάτων.

Μνησθεὶς δὲ ἔναγχος τῆς θεοπνεύστου Γραφῆς, λε%
γούσης ὅτι δὴ υἱὸς ἀνήκοος ἐν ἀπωλείᾳ ἔσται, ὁ δὲ
ὑπήκοος ἔσται ταύτης ἐκτός, μόλις μετὰ πολλοῦ τοῦ\CCSG{40}
δέους διανέστησα ἐμαυτὸν εἰς τὴν τοῦ ὑμετέρου ἐπι%
τάγματος ἐκπλήρωσιν, κρεῖττον ἡγησάμενος περὶ ἐλλεί%
ψεως ἐγκληθῆναι μᾶλλον, ἢ περὶ τελείας ἀγνωμοσύνης
κατακριθῆναι. Μόνον συναγωνίσασθέ μοι ἐν ταῖς προ%
σευχαῖς ἀρωγὸν τοῦ λόγου μοι γενέσθαι τὸ πανάγιον\CCSG{45}
Πνεῦμα˙ ἀρέσει γάρ, ὡς οἶμαι, καὶ τῷ εὐφημουμένῳ τὸ
λεγόμενον παρ’ ἡμῶν τῶν μηδαμινῶν, ὡς ποτὲ ἤρεσε
τῷ αὐτοῦ δεσπότῃ ἡ τῶν δύο λεπτῶν τῆς πενιχρᾶς
χήρας προσκομιδή˙ οὐ γὰρ ἡ ποσότης τῶν διδομένων
τῷ θεῷ εὐπρόσδεκτος, ἀλλὰ τὸ μετὰ γνώμης ἀγαθῆς\CCSG{50}
προσφερόμενον, εἴτε μικρόν, εἴτε πολύ, εὐαγές.

Βαρνάβαν δὲ ἐπαινῶν, πάντων ὁμοῦ τῶν ἀποστόλων
τὸ ἱερὸν ἱεράτευμα ἐπαινέσομαι˙ ὧν γὰρ ἡ κλῆσις μία
καὶ ἡ δόξα ἡ αὐτή, καὶ ὧν ἡ τάξις μία καὶ τὸ ἀξίωμα
τὸ αὐτό, καὶ ὧν οἱ ἀγῶνες ἴσοι καὶ οἱ στέφανοι οὐ\CCSG{55}
διαλλάττουσι, καὶ ὧν ἡ πολιτεία κοινή, τούτων ἰσότιμα
καὶ τὰ βραβεῖα. Δεῦτε τοίνυν, εἰ δοκεῖ, ἑλκύσαντες εἰς
μέσον διὰ τοῦ λόγου τὸν θεσπέσιον Βαρνάβαν, καθὼς
οἷον τέ ἐστιν ὑφάνωμεν αὐτῷ τὰ ἐγκώμια˙ καὶ γὰρ
εἰμὶ τοῦ ἀνδρὸς θερμότατος ἐραστής˙ οἶδα δὲ καὶ πέ\CCSG{60}%
πεισμαι ὅτι καὶ ὑμῖν γλυκύτατος καὶ ἁπλῶς ὀνομαζόμενος
ὁ ἀπόστολος. Τιμήσωμεν οὖν αὐτὸν κατὰ δύναμιν˙ τὸ
γὰρ κατὰ δύναμιν εἰπεῖν εὐγνωμοσύνης ἀπόδειξις˙ τὸ
δὲ κατ’ ἀξίαν λέγειν συγχωρήσωμεν ἔχειν κατὰ παντὸς
λόγου τὰ νικητήρια.\CCSG{65}

\saut

  Ὑμνείσθω τοίνυν πρὸς ἡμῶν τῶν ἐλαχίστων ὁ μέγας
ἐν ἀποστόλοις Βαρνάβας˙ εὐφημείσθω ὁ υἱὸς τῆς πα-
ρακλήσεως παρὰ πάντων τῶν εἰς Χριστὸν πιστευόντων˙
δοξαζέσθω ὑπὸ πάσης κτίσεως ὁ τιμηθεὶς ὑπὸ τοῦ
Πατρὸς καὶ κληθεὶς ὑπὸ τοῦ Υἱοῦ καὶ τελειωθεὶς τοῦ (70)
τοῦ ἁγίου Πνεύματος. Βαρνάβας, ὁ μέγας τῆς ἐκκλησίας
ῥήτωρ, ἡ τοῦ εὐαγγελικοῦ κηρύγματος σάλπιγξ, τοῦ
μονογενοῦς ἡ φωνή, ἡ κιθάρα τοῦ Πνεύματος, τὸ τῆς
χάριτος πλῆκτρον. Βαρνάβας, ὁ ἰσχυρὸς τῶν ὑπὲρ Χρι-
στοῦ στρατιώτης πολέμων, ὁ ἐν τῷ σώματι τὸν ἀσώ- (75)
ματον τύραννον τροπωσάμενος μετὰ παντὸς τοῦ συνα-
ποστάτου αὐτοῦ στρατεύματος, τῶν ἀγαθῶν μαθημά-
των ὁ εὐσεβὴς παιδοτρίβης, ὁ ἀπλανὴς ὁδηγὸς τῆς
τοῦ Χριστοῦ ποίμνης, ὁ λογικὸς παράδεισος τοῦ θεοῦ,
ὁ πάσης ἀρετῆς τὰ φυτὰ τεθηλότα ἐν ἑαυτῷ κεκτημένος, (80)
τὸ πολυανθὲς τῆς πίστεως βλάστημα, τὸ τῆς ἀγάπης
εὐωδέστατον ῥόδον, τὸ τῆς ἐλπίδος ἀμαράντινον ἄνθος,
ὁ τῆς χάριτος εὐωδέστατος στάχυς, τῆς ζωοποιοῦ ἀμ-
πέλου τὸ πολύφορον κλῆμα, τῆς ἀθανασίας ὁ μελίστακτος
βότρυς, ὁ τῆς ὑπομονῆς ἀσάλευτος πρόβολος, ὁ τῆς (85)
ἐγκρατείας σταυροφόρος ὁπλίτης, ὁ κατὰ τῶν χριστο-
κτόνων ἄφοβος στρατηλάτης, ἡ τοῦ σταυρωθέντος Χριστοῦ
οὐρανομήκης παραφυάς. Βαρνάβας, ὁ υἱὸς τῆς παρακλή-
σεως, ὁ τῆς εὐσεβείας διδάσκαλος, ὁ στύλος καὶ ὄροφος
τῆς πίστεως, ὁ πύργος ὁ ἄσειστος, ὁ θεμέλιος ὁ ἀρ- (90)
ραγής, ἡ κρηπὶς ἡ ἀδιάλυτος, ἡ πέτρα ἡ ἀμετακίνητος,
ὁ λιμὴν ὁ γαληνοφόρος τῶν χειμαζομένων, ἡ παράκλησις
πάντων τῶν θλιβομένων, ὁ πιστὸς οἰκονόμος καὶ φρό-
νιμος, ὁ ἄριστος ἀρχιτέκτων ὁ τὸν αἰθέριον πλοῦν καὶ
ἰσάγγελον βίον ἐπὶ γῆς κατορθώσας, ὁ τῶν ἐκκλησιῶν (95)
προστάτης, ὁ τῶν πτωχῶν ἀντιλήπτωρ καὶ τῶν πενήτων
χορηγὸς καὶ τροφεύς, ἡ τῶν χηρῶν παρηγορία καὶ τῶν
ὀρφανῶν κηδεμονέστατος πατήρ. Βαρνάβας, ὁ θησαυρὸς
τῶν τοῦ Χριστοῦ μυστηρίων, ὁ ταμίας τῶν ὀρθοδόξων
τῆς ἁγίας τοῦ θεοῦ ἐκκλησίας δογμάτων, ὁ τῶν νο- (100)
σούντων ἰατρὸς ἀκαπήλευτος, ἡ τῶν ὑγιαινόντων ἀ-
διάπτωτος εὐφροσύνη, ὁ τῆς τοῦ Χριστοῦ ἀγέλης φι-
λάγρυπνος φύλαξ, ὁ βαίνων ἐπὶ γῆς καὶ ἐν οὐρανοῖς
τὸ πολίτευμα ἔχων, ὁ βοήσας ἐν κόσμῳ δόγματα εὐ-
σεβείας καὶ λαβὼν ἐν οὐρανοῖς αἰώνια στέμματα, ὁ τῶν (105)
ἐθνῶν θεοφόρος ὁδηγὸς καὶ τῶν ἐκκλησιῶν θεομακά-
ριστος κήρυξ, ὁ λειμὼν τῆς εὐωδίας Χριστοῦ, ἡ ῥοδωνία
τῶν οὐρανοπνόων ἀρετῶν, τὸ πολύφορον γεώργιον τῶν
τοῦ Χριστοῦ χαρισμάτων, ὁ ἐπὶ γῆς μετελθὼν τὴν
πολιτείαν τὴν ἄφθαρτον, κρείττονα βιώσας ἐπὶ γῆς ὡς (110)
ἐν οὐρανοῖς οἱ ἀσώματοι ἄγγελοι, ὁ ἐν ἀσκήσει δια-
πρέψας καὶ ἐν θεωρίαις προκόψας καὶ ἐν θαύμασι με-
γαλυνθείς, ὁ τῆς παρθενίας πολύτιμος μαργαρίτης, τῆς
ἁγνείας τὸ ἐκλεκτὸν βηρύλλιον, τῆς σωφροσύνης τὸ
καθαρὸν πάζιον, ὁ τὸν Χριστὸν ἔχων ἐν ἑαυτῷ λαλοῦντα, (115)
τὸ σκεῦος τῆς ἐκλογῆς τοῦ θεοῦ, ὁ τὸν κόσμον κα-
ταλείψας καὶ τὰ ἐν αὐτῷ πάντα σκύβαλα ἡγησάμενος,
ἵνα Χριστὸν μόνον κερδήσῃ τὸν βασιλέα τῶν αἰώνων,
ὁ τὸν ἑαυτοῦ σταυρὸν ἀδιστάκτως ἀράμενος καὶ τῷ
Χριστῷ προθύμως ἀκολουθήσας ὡς ἀληθινὸς μαθητής, ὁ (120)
τῶν δαιμόνων φυγαδευτὴς καὶ τοῦ διαβόλου τραυματιστής,
ὁ τὴν τετραπέρατον οἰκουμένην τῇ ἀκαμάτῳ πορείᾳ
βαδίσας ὡς λειτουργὸς ἀγαθὸς καὶ φιλοδέσποτος, καὶ
πάντα τὰ ἔθνη διὰ τοῦ εὐαγγελίου προσαγαγὼν τῇ
πίστει τοῦ Χριστοῦ, ὁ ἑαυτὸν ἑτοιμάσας εἰς κατοικητήριον (125)
τῆς ὁμοουσίου τριάδος καὶ ναὸς γενόμενος τοῦ ἐπὶ
πάντων θεοῦ. Βαρνάβας, τὸ τῶν Κυπρίων σεμνὸν ἐγ-
καλλώπισμα καὶ τῆς οἰκουμένης ἀκαταγώνιστος πρόμαχος,
ὁ ὑπερεκπερισσοῦ ἀγαπήσας τὸν Χριστὸν καὶ ὑπὲρ αὐτοῦ
τὴν ψυχὴν καθ’ ἑκάστην ἡμέραν προϊέμενος καὶ σὺν (130)
αὐτῷ βασιλεύσας εἰς αἰῶνας αἰώνων.
  
  Ἔκαμεν ὁ λόγος, τοῦτον τὸν θεσπέσιον καὶ τριπόθητον
Βαρνάβαν τὸν ἀπόστολον ἐπᾶραι τοῖς ἐγκωμίοις ποθῶν,
καὶ οὔπω ἥψατο τοῦ προοιμίου˙ τοῖς γὰρ ἐπαίνοις ἀπρό-
σιτος τυγχάνει ὁ θαυμάσιος. Διὸ ἀφέντες ὡς ἀνέφικτον (135)
τὸν περὶ τῶν ἐγκωμίων λόγον, ὀλίγα τῶν εἰς ἡμᾶς
ἰόντων περὶ τῆς τούτου βιώσεως καὶ τελειώσεως ἔκ τε
τοῦ Στρωματέως καὶ ἐξ ἑτέρων ἀρχαίων συγγραμμάτων
παραθήσομαι τῇ ὑμετέρᾳ ὁσιότητι, καὶ οὕτω τὸ πέρας
ἐπιθήσομεν τῷ διηγήματι, παραχωρήσαντες τῇ θεοπνεύστῳ (140)
Γραφῇ τοῦ ἀοιδίμου τὴν κεφαλὴν μεγαλοπρεπῶς στε-
φανῶσαι˙ φησὶ γάρ˙ “Ἦν δὲ Βαρνάβας ἀνὴρ ἀγαθὸς καὶ
πλήρης Πνεύματος ἁγίου καὶ πίστεως.” Οὗ τί ἂν γένοιτο
ἴσον ἢ παραπλήσιον πώποτε;

\saut


Ὡρμᾶτο μὲν οὖν ὁ τρισόλβιος οὗτος ἐκ τῆς εὐλογη\CCSG{145}%
μένης φυλῆς τοῦ Λευΐ, ἐξ ἧς ὑπῆρχον Μωϋσῆς καὶ
Ἀαρών, οἱ μεγάλοι τοῦ θεοῦ προφῆται καὶ πρῶτοι τοῦ
δήμου Καάθ, ἐκ τῆς συγγενείας Σαμουὴλ τοῦ προφήτου
κατάγων τὸ γένος. Οἱ δὲ τούτου πρόγονοι, διὰ περί%
στασιν πολέμων καταλαβόντες τὴν Κυπρίων χώραν, φι\CCSG{150}%
λοφρόνως ᾤκουν αὐτήν˙ ἦσαν δὲ εὐλαβεῖς κατὰ τὸν
νόμον καὶ πλούσιοι σφόδρα, ὅθεν καὶ ἐν Ἱεροσολύμοις
εἶχον οὐσίαν ἱκανὴν καὶ ἀγρὸν πλησίον τῆς πόλεως
περικαλλέστατον, οὐ μόνον παντοίοις καρπίμοις φυτοῖς
ὡραϊζόμενον, ἀλλὰ καὶ μεγέθει κτισμάτων περιφανέστατον˙\CCSG{155}
ἐξότε γὰρ ἤκουσαν Ἡσαΐου τοῦ προφήτου λέγοντος˙
\enquote{Μακάριος ὁ ἔχων σπέρμα ἐν Σιὼν καὶ οἰκείους ἐν
Ἱερουσαλήμ}, σαρκικῶς τὴν προφητείαν ἐκδεχόμενοι
Ἑβραίων παῖδες, ἔσπευδεν ἕκαστος τῶν εὐπορούντων
ἔχειν τὸ κτῆμα ἐν Ἱερουσαλήμ.\CCSG{160}

Τοῦ δὲ δικαίου τούτου τεχθέντος ἐν Κύπρῳ, ὡς εἶδον
αὐτὸν οἱ γονεῖς αὐτοῦ ἀστεῖον ὄντα τῷ θεῷ, εὐθέως
μὲν αὐτὸν Ἰωσὴφ ἐπωνόμασαν, τῇ τοῦ πατριάρχου
προσηγορίᾳ τιμῶντες τὸν παῖδα˙ συνέδραμε δὲ τῇ προση%
γορίᾳ τοῦ ὀνόματος ἡ τῶν τρόπων εὐγένεια˙ Ἰωσὴφ\CCSG{165}
γὰρ ἑρμηνεύεται πρό[σ]θεσις θεοῦ˙ προσθήκην γὰρ χά%
ριτος ἔλαβε παρὰ θεοῦ ὁ δίκαιος, ὅπως φθάσῃ εἰς τὴν
ἀποστολικὴν τελειότητα˙ πάλιν δὲ Ἰωσὴφ ἑρμηνεύεται
δόξα θεοῦ˙ γέγονε γὰρ διὰ τῆς ἀρίστης πολιτείας δόξα
θεοῦ˙ καὶ μηδεὶς ὑπερβολὴν εἶναι νομιζέτω τὸ ῥῆμα,\CCSG{170}
ἀλλ’ ἐπιστησάτω ὅτι τῆς θείας ἐστὶ Γραφῆς˙ φησὶ γοῦν
ὁ Παῦλος˙ \enquote{Ἀνὴρ οὐκ ὀφείλει κατακαλύπτεσθαι τὴν
κεφαλήν, εἰκὼν καὶ δόξα θεοῦ ὑπάρχων.} Εἰ οὖν τὸν
κοινὸν ἄνδρα εἰκόνα καὶ δόξαν θεοῦ ἐκάλεσε τὸ σκεῦος
τῆς ἐκλογῆς, τί ἄν τις εἴποι περὶ τοῦ κατὰ θεὸν\CCSG{175}
τελειοτάτου ἀνδρός;

Ἐπειδὴ δὲ αὐξήσας ὁ Βαρνάβας ἐγένετο παῖς, ἀνήγαγον
οἱ γονεῖς εἰς Ἱεροσόλυμα καὶ παρέδωκαν αὐτὸν μαθεῖν
τὸν νόμον ἀκριβῶς καὶ τοὺς προφήτας παρὰ τοὺς πόδας
Γαμαλιήλ˙ εἶχε δὲ τὸν Παῦλον συμφοιτητήν, Σαῦλον τέως\CCSG{180}
ὀνομαζόμενον. Προέκοπτε δὲ ὁσημέραι ὁ Βαρνάβας ἔν
τε τῇ μαθήσει καὶ ἐν πάσῃ ἀρετῇ˙ οὔπω μέντοι εἰς
τὴν ἐφημερίαν τῶν Λευϊτῶν ἐτέτακτο διὰ τὸ ἐλλιπὲς τῆς
ἡλικίας˙ ἔτι γὰρ ἔφηβος ἦν. Οὐκ ἀφίστατο δὲ ἀπὸ τοῦ
ἱεροῦ, νηστείαις καὶ δεήσεσι λατρεύων νύκτα καὶ ἡμέραν˙\CCSG{185}
οὕτω δὲ ἀπήγγειλε τὸν νόμον καὶ τὰς λοιπὰς γραφάς,
ὡς μὴ δεῖσθαι τῆς ἀπὸ τῶν γραμμάτων ὑπομνήσεως˙
τὴν δὲ ἡσυχίαν οὕτως εἶχε φίλην, ὡς μητέρα τῆς
σωφροσύνης˙ τὰς δὲ ἐπιβλαβεῖς ὁμιλίας μυσαττόμενος
ἔφευγε, καθαρὸν ὑπάρχων καὶ ἀκέραιον καὶ ἀμόλυντον\CCSG{190}
ἄγαλμα˙ ἦν δὲ πᾶσιν ἐπίδοξος δι’ ἀρετήν.

Κατ’ ἐκεῖνον δὲ τὸν καιρὸν συνέβη τὸν κύριον παρα%
γενέσθαι εἰς Ἱεροσόλυμα καὶ ἰάσασθαι τὸν παράλυτον
ἐν τῇ προβατικῇ, καὶ ἕτερα πολλὰ σημεῖα καὶ τέρατα
ἐργάσασθαι ἐν τῷ ἱερῷ. Ταῦτα θεασάμενος ὁ μακάριος,  \CCSG{195}
ἐξεπλήσσετο καὶ εὐθέως προσελθών, ἔπεσε παρὰ τοὺς
πόδας αὐτοῦ καὶ ἐδέετο εὐλογηθῆναι παρ’ αὐτοῦ. Ὁ δὲ
τὰς καρδίας ἐμβατεύων Χριστός, ἀποδεξάμενος αὐτοῦ τὴν
πίστιν, εὐμενῶς αὐτὸν ὑπεδέξατο καὶ τῆς θείας αὐτοῦ
συντυχίας μετέδωκεν˙ ὁ δὲ πλέον ἐξεκαίετο εἰς τὴν τοῦ\CCSG{200}
κυρίου ἀγάπην. Καταλαβὼν δὲ τὸ τάχος τὴν οἰκίαν
Μαρίας, τῆς μητρὸς Ἰωάννου τοῦ ἐπικαλουμένου Μάρκου,
ἥτις ἐλέγετο εἶναι αὐτοῦ θεία --- διὸ καὶ Μάρκον τὸν
ἀνεψιὸν Βαρνάβα ἐκάλουν αὐτόν ---, εἶπε πρὸς αὐτήν˙
\enquote{Δεῦρο, λέγων, ὦ γύναι, ἴδε ἅπερ ἐπεθύμουν ἰδεῖν οἱ\CCSG{205}
πατέρες ἡμῶν˙ ἰδοὺ γὰρ Ἰησοῦς τίς προφήτης ἀπὸ
Ναζαρὲτ τῆς Γαλιλαίας ἐστὶν ἐν τῷ ἱερῷ, μεγαλοπρεπῶς
θαυματουργῶν, καὶ ὡς τοῖς πολλοῖς δοκεῖ, αὐτός ἐστι
Μεσσίας ὁ μέλλων ἔρχεσθαι.} Ἀκούσασα δὲ ταῦτα ἡ
θαυμασία γυνὴ καὶ καταλιποῦσα τὰ ἐν χερσί, κατέλαβε\CCSG{210}
τὸν ναὸν τοῦ θεοῦ, καὶ ἰδοῦσα τὸν κύριον καὶ δεσπότην
τοῦ ναοῦ, ἔπεσεν εἰς τοὺς πόδας αὐτοῦ, δεομένη καὶ
λέγουσα˙ \enquote{Κύριε, εἰ εὖρον χάριν ἐναντίον σου, δεῦρο
εἰς τὸν οἶκον τῆς δούλης σου καὶ εὐλόγησον τῇ εἰσόδῳ
σου τὰ οἰκετικά σου.} Ὁ δὲ κύριος ἐπένευσε τῇ\CCSG{215}
παρακλήσει˙ ὃν παραγενόμενον ὑπεδέξατο χαίρουσα εἰς
τὸ ὑπερῶον αὐτῆς. Ἀπ’ ἐκείνης οὖν τῆς ἡμέρας, ἡνίκα
ἤρχετο ὁ κύριος εἰς Ἱεροσόλυμα, ἐκεῖ ἀνεπαύετο μετὰ
τῶν μαθητῶν αὐτοῦ, ἐκεῖ ἐποίησε τὸ Πάσχα μετὰ τῶν
μαθητῶν αὐτοῦ, ἐκεῖ ἐμυσταγώγησε τοὺς μαθητὰς διὰ\CCSG{220}
τῆς μεταλήψεως τῶν ἀπορρήτων μυστηρίων. Λόγος γὰρ
ἦλθεν εἰς ἡμᾶς ἀπὸ γερόντων ὅτι ὁ τὸ κεράμιον βα%
στάζων τοῦ ὕδατος, ᾧ κατακολουθῆσαι προσέταξεν ὁ
κύριος τοῖς μαθηταῖς, Μάρκος ἦν, ὁ υἱὸς ταύτης τῆς
μακαρίας Μαρίας˙ ὁ δὲ κύριος \enquote{πρὸς τὸν δεῖνα} εἶπεν\CCSG{225}
οἰκονομικῶς, ὡς φασὶν οἱ πατέρες, ἑρμηνεύοντες τοῦτο
τὸ χωρίον, διδάσκων ἡμᾶς διὰ τοῦ αἰνίγματος ὅτι παντὶ
τῷ εὐτρεπίζοντι ἑαυτόν, παρ’ αὐτῷ ὁ κύριος αὐλίζεται.
Ἐν αὐτῷ τοίνυν τῷ ὑπερώῳ ἐποίησεν ὁ κύριος τὸ
Πάσχα˙ ἐν αὐτῷ ἐφάνη τοῖς περὶ τὸν Θωμᾶν, ἐγερθεὶς\CCSG{230}
ἐκ νεκρῶν˙ ἐκεῖ μετὰ τὴν ἀνάληψιν ἀνῆλθον οἱ μαθηταί,
ἐλθόντες ἀπὸ τοῦ ὄρους τῶν Ἐλαιῶν μετὰ τῶν λοιπῶν
ἀδελφῶν, ὄντων τὸν ἀριθμὸν ὡς ἐκατὸν εἴκοσιν, ἐν οἷς
ἦν Βαρνάβας καὶ Μάρκος˙ ἐκεῖ κατέβη τὸ Πνεῦμα τὸ
ἅγιον ἐν πυρίναις γλώσσαις ἐπὶ τοὺς μαθητὰς ἐν τῇ\CCSG{235}
ἡμέρᾳ τῆς Πεντηκοστῆς˙ ἐκεῖ ἵδρυται νῦν ἡ μεγάλη καὶ
ἁγιωτάτη Σιών, ἡ μήτηρ πασῶν τῶν ἐκκλησιῶν.

Τότε δὲ Βαρνάβας ἠκολούθησε τῷ κυρίῳ, ὑποστρέφοντι
ἀπὸ Ἱεροσολύμων εἰς τὴν Γαλιλαίαν˙ πολλῶν δὲ παν%
ταχόθεν προσερχομένων τῷ κυρίῳ καὶ εἰς αὐτὸν πι\CCSG{240}%
στευόντων, τότε εἶπε τοῖς μαθηταῖς˙ \enquote{Ὁ μὲν θερισμὸς
πολύς, οἱ δὲ ἐργάται ὀλίγοι.} Τότε ἀνέδειξε τοὺς ἑ%
βδομήκοντα μαθητάς, ὧν πρῶτος καὶ ἔξαρχος καὶ κο%
ρυφαῖος ὁ μέγας Βαρνάβας ἐτύγχανεν. Ἀλλὰ μηδείς,
ἀκούων ὅτι οἱ ἀπόστολοι ἐπέθηκαν αὐτῷ τὴν προση\CCSG{245}%
γορίαν ταύτην, οἰέσθω ἀμαθῶς ὅτι ἄνευ θείας ἐπιπνοίας
τὴν προσηγορίαν ταύτην ἐδέξατο˙ Πέτρος γὰρ αὐτῷ
ἐπέθηκε τὸ ὄνομα τοῦτο, δι’ ἀποκαλύψεως τοῦ ἁγίου
Πνεύματος, ὁ καὶ δι’ ἀποκαλύψεως τοῦ Πατρὸς τὴν τοῦ
Υἱοῦ θεολογίαν δεξάμενος˙ καὶ ὥσπερ Ἰάκωβος καὶ\CCSG{250}
Ἰωάννης υἱοὶ βροντῆς ἀπὸ τῆς ἀρετῆς ἐκλήθησαν, οὕτω
καὶ Βαρνάβας ἀπὸ τῆς ἀρετῆς ἐκλήθη υἱὸς παρακλήσεως,
πάντων παράκλησις γενόμενος δι’ ὑπερβολὴν ἁγιότητος.

Ἀκούσας δὲ τοῦ κυρίου διδάσκοντος καὶ λέγοντος˙
\enquote{Πωλήσατε τὰ ὑπάρχοντα ὑμῶν καὶ δότε ἐλεημοσύνην˙\CCSG{255}
ποιήσατε ἑαυτοῖς βαλλάντια μὴ παλαιούμενα, θησαυρὸν
ἀνέκλειπτον ἐν τοῖς οὐρανοῖς,} μηδὲν μελλήσας, εὐθέως
τὴν ὑπολειφθεῖσαν αὐτῷ οὐσίαν, πολύτιμον οὖσαν, ὑπὸ
τῶν γονέων --- ἦσαν γὰρ ὑπεξελθόντες τὸν βίον --- 
πᾶσαν καταπωλήσας, διένειμε τοῖς χρείαν ἔχουσι, κα\CCSG{260}%
ταλείψας ἑαυτῷ μόνον τὸν ἀγρὸν ἐκεῖνον εἰς ἰδίαν
ἀποτροφήν. Μετὰ δὲ τὸ πάθος καὶ τὴν ἀνάστασιν καὶ
τὴν ἀνάληψιν τοῦ κυρίου καὶ τὴν τοῦ ἁγίου Πνεύματος
ἐπιφοίτησιν, ἐπὶ πλεῖον πυρωθεὶς ὁ θεῖος Βαρνάβας
τῇ εἰς τὸν κύριον ἀγάπῃ καὶ αὐτὸν τὸν ἀγρὸν ἐκεῖνον\CCSG{265}
ἀποδόμενος καὶ λαβὼν ἱκανὰ χρήματα, ἅπαντα ἐνέγκας
παρὰ τοὺς πόδας τῶν ἀποστόλων ἔθηκε, μηδὲν τὸ
σύνολον καταλείψας ἑαυτῷ ἐξ αὐτῶν, τῷ καθ’ ἑαυτὸν
ὑποδείγματι εἰς τὴν ὁμοίαν ἀρετὴν πάντας τοὺς μαθητὰς
διεγείρων.\CCSG{270}

Ἐλάλει δὲ καὶ συνεζήτει πρὸς τὸν Σαῦλον, βουλόμενος
εἰς τὴν τοῦ κυρίου αὐτὸν πίστιν ἀγαγεῖν. Σαῦλος δὲ
τῇ κατὰ τὸν νόμον ἀκριβεῖ πολιτείᾳ δῆθεν ἐπερειδόμενος,
τοῦ μὲν Βαρνάβα κατεγέλα ὡς ἀπατηθέντος, τὸν δὲ
κύριον ἐβλασφήμει, υἱὸν τοῦ τέκτονος αὐτὸν ἀποκαλῶν\CCSG{275}
καὶ ἰδιώτην καὶ ἄγροικον καὶ βιοθανῆ. Ὡς δὲ εἶδε τὰ
διὰ τῶν ἀποστόλων γινόμενα τῶν θαυμάτων μεγα%
λουργήματα καὶ τοῦ λαοῦ τὸ πλῆθος τῶν καθ’ ἡμέραν
προστιθεμένων τῷ λόγῳ τῆς πίστεως, ἐδάκνετο τὴν
ψυχήν. Προσβαλὼν δὲ μετὰ τῶν Λιβερτίνων καὶ Κυρη\CCSG{280}%
ναίων καὶ Ἀλεξανδρέων τῷ μεγάλῳ τῆς ἐκκλησίας ῥήτορι
Στεφάνῳ, καὶ μὴ δυνηθεὶς ἀντιστῆναι τῇ σοφίᾳ καὶ
τῷ Πνεύματι, ᾧ ἐλάλει, εἰς μανίαν ἐτράπη˙ καὶ γενόμενος
πλήρης θυμοῦ, ἤγειρε κατ’ αὐτοῦ τοὺς ἀτάκτους τοῦ λαοῦ,
καὶ τοῦτον ἀνελών, ἐπήγειρε διωγμὸν μέγαν ἐπὶ τὴν\CCSG{285}
ἐκκλησίαν τὴν ἐν Ἱεροσολύμοις. Ἀλλ’ οὕτως αὐτὸν ἀτάκτως
πορευόμενον εἰς Δαμασκὸν ἐπὶ κακοποιήσει τῶν πιστῶν,
ὁ κύριος ὑποσκελίσας, ἔβαλεν ἐπὶ πρόσωπον˙ ὁ δὲ
πεσὼν εἰς τὴν γῆν, ἐπέγνω τίνα διώκει, καὶ πηρωθεὶς
τὰς ὄψεις, ἀνέβλεψε διόλου εἰς τὸ ὕψος τοῦ οὐρανοῦ.\CCSG{290}
Ὑποστρέψας δὲ εἰς Ἱερουσαλήμ, ἐζήτει κολλᾶσθαι τοῖς
μαθηταῖς, καὶ πάντες ἔφευγον ἀπ’ αὐτοῦ, φοβούμενοι τὴν
πολλὴν αὐτοῦ ὠμότητα. Ὁ δὲ μέγας Βαρνάβας, ἀπαντήσας
αὐτῷ, εἶπεν˙ \enquote{Ἕως πότε, Σαῦλε, Σαοὺλ τυγχάνεις; Ἵνα
τί οὕτως ἰταμῶς τὸν εὐεργέτην διώκεις; Παῦσαι πορθῶν\CCSG{295}
τὸ ὑπὸ τῶν προφητῶν πάλαι βοώμενον φρικτὸν μυστή%
ριον καὶ ἐν τοῖς ἡμετέροις καιροῖς ἀποκαλυφθὲν εἰς
σωτηρίαν ἡμῶν.} Ταῦτα ἀκούσας ὁ Σαῦλος, ἔπεσεν ἐπὶ
τοὺς πόδας Βαρνάβα, μετὰ πολλῶν δακρύων κράζων καὶ
λέγων˙ \enquote{Συγχώρησόν μοι, ὁδηγὲ τοῦ φωτὸς καὶ διδά\CCSG{300}%
σκαλε τῆς ἀληθείας˙ ἔγνων γὰρ τῇ πείρᾳ τῶν λόγων
σου τὴν ἀλήθειαν˙ ὃν γὰρ ἐγὼ βλασφημῶν ἔλεγον υἱὸν
τοῦ τέκτονος, νῦν ὁμολογῶ αὐτὸν τοῦ θεοῦ τοῦ ζῶντος
Υἱὸν μονογενῆ, ὁμοούσιόν τε καὶ ὁμόδοξον καὶ ὁμό%
θρονον, συναΐδιόν τε καὶ συνάναρχον˙ ὃς ὢν ἀπαύγασμα\CCSG{305}
τῆς δόξης καὶ χαρακτὴρ τῆς ὑποστάσεως τοῦ ἀοράτου
θεοῦ, ἐπ’ ἐσχάτου τῶν ἡμερῶν τούτων, δι’ ἡμᾶς καὶ διὰ
τὴν ἡμετέραν σωτηρίαν, ἐκένωσεν ἑαυτόν, μορφὴν δούλου
λαβών, τουτέστι τέλειον ἄνθρωπον ἐκ τῆς ἁγίας παρθένου
καὶ θεοτόκου Μαρίας, ἀσυγχύτως, ἀτρέπτως, ἀδιαιρέτως,\CCSG{310}
ἀχωρίστως˙ καὶ σχήματι εὑρεθεὶς ὡς ἄνθρωπος, ἐτα%
πείνωσεν ἑαυτόν, ὑπήκοος γενόμενος μέχρι θανάτου, θα%
νάτου δὲ σταυροῦ˙ ὃς καὶ ἀνέστη ἐκ νεκρῶν τῇ τρίτῃ
ἡμέρᾳ καὶ ὤφθη ὑμῖν, τοῖς ἀποστόλοις αὐτοῦ, καὶ
ἀνελήφθη εἰς τοὺς οὐρανοὺς καὶ ἐκάθισεν ἐν δεξιᾷ τοῦ\CCSG{315}
Πατρὸς καὶ πάλιν ἔρχεται μετὰ δόξης κρῖναι ζῶντας
καὶ νεκρούς, καὶ τῆς βασιλείας αὐτοῦ οὐκ ἔσται τέλος.}

Ταῦτα ἀκούσας ὁ θεσπέσιος Βαρνάβας παρὰ τοῦ
βλασφήμου καὶ διώκτου, ἐξεπλάγη καὶ ὑπὸ τῆς χαρᾶς
ἐγένετο τὸ πρόσωπον αὐτοῦ ὡσεὶ ἄνθος πρωϊνόν˙ συμ\CCSG{320}%
περιλαβόμενος δὲ αὐτὸν καὶ καταφιλήσας, εἶπεν˙ \enquote{Τίς
σε, Σαῦλε, τοιαῦτα ἐδίδαξε θεόπνευστα ῥήματα φθέγ%
γεσθαι; Ἢ τίς σε ἔπεισεν Ἰησοῦν τὸν Ναζωραῖον Υἱὸν
θεοῦ ὁμολογεῖν; Ἢ πόθεν ἔμαθες οὐρανίων δογμάτων
τοσαύτην ἀκρίβειαν;} Ὁ δὲ κεκυφὼς καὶ δακρύων, μετὰ\CCSG{325}
πολλῆς τῆς κατανύξεως ἔφη˙ \enquote{Αὐτὸς ὁ κύριος Ἰησοῦς,
ὁ πολλάκις βλασφημηθεὶς καὶ διωχθεὶς ὑπ’ ἐμοῦ τοῦ
ἁμαρτωλοῦ, ἐδίδαξέ με ταῦτα πάντα˙ ὥσπερ γὰρ εἰ
τῷ ἐκτρώματι ὤφθη κἀμοὶ καὶ ἔτι ἔναυλον ἐν τοῖς ὠσὶν
ἔχω τὴν θείαν καὶ γλυκεῖαν αὐτοῦ φωνήν˙ πᾶσαν γὰρ\CCSG{330}
ἀγαθότητα ὑπερβάς, εἴρηκέ μοι, κειμένῳ ἐλεεινῶς ἐπὶ
πρόσωπον, ἀπολογούμενος μᾶλλον ἢ ἐγκαλῶν˙ \enquote{Σαούλ,
Σαούλ, τί με διώκεις;} Ἐγὼ δὲ μετὰ φρίκης καὶ φόβου
ἀπεκρίθην˙ \enquote{Τίς εἶ, κύριε;} Ὁ δὲ κύριος μετὰ πολλῆς
τῆς ἐπιεικείας καὶ συμπαθείας εἶπε μοι˙ \enquote{Ἐγώ εἰμι Ἰησοῦς\CCSG{335}
ὁ Ναζωραῖος, ὃν σὺ διώκεις.} Ἐκπλαγεὶς δὲ ἐγὼ ἐπὶ
τῇ ἀφάτῳ αὐτοῦ ἀνεξικακίᾳ, ἐδεήθην αὐτοῦ, λέγων˙ \enquote{Τί
ποιήσω, κύριε;} Ὁ δὲ παραχρῆμα συνεβίβασέ με ταῦτα
πάντα καὶ ἔτι πλείονα τούτων.}

Τότε ὁ μέγας Βαρνάβας ἐπιλαβόμενος τῆς χειρὸς αὐτοῦ,\CCSG{340}
ἤγαγε πρὸς τοὺς ἀποστόλους, λέγων˙ \enquote{Τί φεύγετε τὸν
ποιμένα, λύκον αὐτὸν εἶναι ὑπολαμβάνοντες; Τί τὸν κυ%
βερνήτην ὡς πειρατὴν διώκετε; Τί τὸν ἀριστέα ὡς
προδότην μυσάττεσθε; Τί τὸν νυμφαγωγὸν ὡς ληστὴν
τοῦ παστοῦ ἀποπέμπετε; Παστὸς γὰρ πνευματικὸς ἡ\CCSG{345}
ἐκκλησία τυγχάνει, ἧς ποιμένα καὶ κυβερνήτην καὶ ὑπέρ%
μαχον ὁ κύριος δι’ ἑαυτοῦ ἐχειροτόνησεν.} Τότε διηγήσατο
αὐτοῖς ὁ Παῦλος ὅσα συνέβη αὐτῷ κατὰ τὴν ὁδόν, καὶ
ὅτι εἶδε τὸν κύριον καὶ ὅτι ἐλάλησεν αὐτῷ, καὶ πῶς
ἐν Δαμασκῷ ἐπαρρησιάσατο ἐπὶ τῷ ὀνόματι τοῦ κυρίου˙\CCSG{350}
καὶ ἦν διδάσκων σὺν αὐτοῖς τὸν λόγον τοῦ κυρίου ἐν
Ἱερουσαλήμ. Ἦν δὲ βαρὺς τοῖς Ἰουδαίοις σφόδρα, ὅτι
ὁ χθὲς τὸν Ἰησοῦν διώκων, τοῦτον σήμερον Υἱὸν θεοῦ
κηρύττει, καὶ ἐβουλεύσαντο ἀνελεῖν αὐτόν. Μαθόντες δὲ
οἱ ἀπόστολοι, ἀπέστειλαν αὐτὸν κηρύξαι ἐν τῇ ἰδίᾳ\CCSG{355}
πατρίδι.








