

%\ledsection{\cite{BHG226}}

\titulus{Ἀλεξάνδρου μοναχοῦ ἐγκώμιον εἰς τὸν ἅγιον
Βαρνάβαν τὸν ἀπόστολον, προτραπέντος ὑπὸ τοῦ
πρεσβυτέρου καὶ κλειδούχου τοῦ σεβασμίου αὐτοῦ
ναοῦ, ἐν ᾧ ἱστορεῖται καὶ ὁ τρόπος τῆς ἀποκαλύψεως τῶν ἁγίων αὐτοῦ λειψάνων.}\CCSG{5}

Μεγίστην λόγων ὑπόθεσιν προέθετο τοῖς πτωχοτάτοις
ἡμῖν ἡ πατρικὴ ὑμῶν φιλομαθία, ὦ πατέρων ἄριστε καὶ
ἀσκητῶν δοκιμώτατε πάτερ˙ προεβάλετο γὰρ ἡμῖν εἰς
εὐφημίαν τὸν θεσπέσιον καὶ ὡς ἀληθῶς τρισμακάριον
καὶ ἐν ἀποστόλοις ἐπισημότατον υἱὸν τῆς παρακλήσεως\CCSG{10}
καὶ φωστῆρα τῆς οἰκουμένης, Βαρνάβαν τὸν πολυΰμνητον.
Ἀλλ’ ἐγὼ πρὸς ταύτην τὴν ὑπερλίαν ὑπόθεσιν τὴν ἐμαυτοῦ
ἀπαιδευσίαν ἀντεξάγων, τὴν ὑπακοὴν ἐπὶ πλεῖστον χρόνον
παρῃτησάμην, τὴν ἐγχείρησιν ὀρρωδῶν˙ ποῖος γὰρ καὶ
λόγος ἐφικέσθαι δύναται τῆς ἀποστολικῆς τελειότητος;\CCSG{15}
Οὐ μόνον γὰρ ἐμοὶ ἀδύνατον τὸ τοιοῦτον καθέστηκεν
ἐγχείρημα, ἀλλὰ καὶ τοῖς πολλοῖς δυσχερές˙ οἶμαι γὰρ
πᾶσαν ἀνθρώπων φύσιν λείπεσθαι πρὸς διήγησιν τῶν
τοῦ μεγίστου ἀποστόλου Βαρνάβα ἀγαθῶν πράξεων˙ εἰ
δὲ καὶ ἐπιχειρήσειέ τις περὶ αὐτοῦ λέγειν τί, παρὰ\CCSG{20}
πολὺ μὲν τοῦ μέτρου τῆς ἀξίας τῶν πραγμάτων ἀπο%
λειφθήσεται, κἂν λίαν σοφὸς καὶ ἐπιστήμων τυγχάνῃ καὶ
περὶ λόγους δυνατώτατος, εὑρεθήσεται δὲ βυθῷ θαυμάτων
ἐγκυβιστούμενος. Ἄλλος μὲν οὖν ἄλλο τι τῶν δικαίων
ἕκαστος παρὰ τοῦ Πνεύματος τοῦ ἁγίου εἴληφε χάρισμα,\CCSG{25}
οἱ δὲ ἀπόστολοι πάντα ὁμοῦ τῶν χαρισμάτων ὑπεδέξαντο
τὸν θησαυρὸν καὶ πᾶσαν ἀρετὴν ἐπεδείξαντο. Πῶς οὖν
ὁ τάλας ἐγώ, ὑπὸ μυρίων παθῶν καταπεποντισμένος,
δυνήσομαι τὴν ἀποστολικὴν θάλασσαν διανήξασθαι; Ἰσ%
χνόφωνος γὰρ γραφικῶς εἰπεῖν καὶ βραδύγλωσσος, καὶ\CCSG{30}
οὐκ εὔλαλός εἰμι ἐγώ, ἵνα τὰς ὑπὲρ ἄνθρωπον ἀρετὰς
τοῦ θεσπεσίου ἀποστόλου Βαρνάβα διηγήσωμαι˙ πάσας
γὰρ ὁμοῦ οὗτος τὰς ἀρετὰς κατώρθωσεν ἀκριβῶς, ὡς
οὐδεμίαν τῶν ἄλλων οὐδείς. Διατοῦτο ἀνεβαλόμην ἐπὶ
πολὺ τὴν ὑπακοήν, εἰδὼς τὴν ἐμαυτοῦ ἀναξιότητα μὴ\CCSG{35}
ἐπαρκοῦσαν πρὸς διήγησιν τῶν τοῦ ἀνδρὸς εὐκλεεστάτων
κατορθωμάτων.

Μνησθεὶς δὲ ἔναγχος τῆς θεοπνεύστου Γραφῆς, λε%
γούσης ὅτι δὴ υἱὸς ἀνήκοος ἐν ἀπωλείᾳ ἔσται, ὁ δὲ
ὑπήκοος ἔσται ταύτης ἐκτός, μόλις μετὰ πολλοῦ τοῦ\CCSG{40}
δέους διανέστησα ἐμαυτὸν εἰς τὴν τοῦ ὑμετέρου ἐπι%
τάγματος ἐκπλήρωσιν, κρεῖττον ἡγησάμενος περὶ ἐλλεί%
ψεως ἐγκληθῆναι μᾶλλον, ἢ περὶ τελείας ἀγνωμοσύνης
κατακριθῆναι. Μόνον συναγωνίσασθέ μοι ἐν ταῖς προ%
σευχαῖς ἀρωγὸν τοῦ λόγου μοι γενέσθαι τὸ πανάγιον\CCSG{45}
Πνεῦμα˙ ἀρέσει γάρ, ὡς οἶμαι, καὶ τῷ εὐφημουμένῳ τὸ
λεγόμενον παρ’ ἡμῶν τῶν μηδαμινῶν, ὡς ποτὲ ἤρεσε
τῷ αὐτοῦ δεσπότῃ ἡ τῶν δύο λεπτῶν τῆς πενιχρᾶς
χήρας προσκομιδή˙ οὐ γὰρ ἡ ποσότης τῶν διδομένων
τῷ θεῷ εὐπρόσδεκτος, ἀλλὰ τὸ μετὰ γνώμης ἀγαθῆς\CCSG{50}
προσφερόμενον, εἴτε μικρόν, εἴτε πολύ, εὐαγές.

Βαρνάβαν δὲ ἐπαινῶν, πάντων ὁμοῦ τῶν ἀποστόλων
τὸ ἱερὸν ἱεράτευμα ἐπαινέσομαι˙ ὧν γὰρ ἡ κλῆσις μία
καὶ ἡ δόξα ἡ αὐτή, καὶ ὧν ἡ τάξις μία καὶ τὸ ἀξίωμα
τὸ αὐτό, καὶ ὧν οἱ ἀγῶνες ἴσοι καὶ οἱ στέφανοι οὐ\CCSG{55}
διαλλάττουσι, καὶ ὧν ἡ πολιτεία κοινή, τούτων ἰσότιμα
καὶ τὰ βραβεῖα. Δεῦτε τοίνυν, εἰ δοκεῖ, ἑλκύσαντες εἰς
μέσον διὰ τοῦ λόγου τὸν θεσπέσιον Βαρνάβαν, καθὼς
οἷον τέ ἐστιν ὑφάνωμεν αὐτῷ τὰ ἐγκώμια˙ καὶ γὰρ
εἰμὶ τοῦ ἀνδρὸς θερμότατος ἐραστής˙ οἶδα δὲ καὶ πέ\CCSG{60}%
πεισμαι ὅτι καὶ ὑμῖν γλυκύτατος καὶ ἁπλῶς ὀνομαζόμενος
ὁ ἀπόστολος. Τιμήσωμεν οὖν αὐτὸν κατὰ δύναμιν˙ τὸ
γὰρ κατὰ δύναμιν εἰπεῖν εὐγνωμοσύνης ἀπόδειξις˙ τὸ
δὲ κατ’ ἀξίαν λέγειν συγχωρήσωμεν ἔχειν κατὰ παντὸς
λόγου τὰ νικητήρια.\CCSG{65}

\saut

  Ὑμνείσθω τοίνυν πρὸς ἡμῶν τῶν ἐλαχίστων ὁ μέγας
ἐν ἀποστόλοις Βαρνάβας˙ εὐφημείσθω ὁ υἱὸς τῆς πα%
ρακλήσεως παρὰ πάντων τῶν εἰς Χριστὸν πιστευόντων˙
δοξαζέσθω ὑπὸ πάσης κτίσεως ὁ τιμηθεὶς ὑπὸ τοῦ
Πατρὸς καὶ κληθεὶς ὑπὸ τοῦ Υἱοῦ καὶ τελειωθεὶς τοῦ\CCSG{70}
τοῦ ἁγίου Πνεύματος. Βαρνάβας, ὁ μέγας τῆς ἐκκλησίας
ῥήτωρ, ἡ τοῦ εὐαγγελικοῦ κηρύγματος σάλπιγξ, τοῦ
μονογενοῦς ἡ φωνή, ἡ κιθάρα τοῦ Πνεύματος, τὸ τῆς
χάριτος πλῆκτρον. Βαρνάβας, ὁ ἰσχυρὸς τῶν ὑπὲρ Χρι%
στοῦ στρατιώτης πολέμων, ὁ ἐν τῷ σώματι τὸν ἀσώ\CCSG{75}
ματον τύραννον τροπωσάμενος μετὰ παντὸς τοῦ συνα%
ποστάτου αὐτοῦ στρατεύματος, τῶν ἀγαθῶν μαθημά%
των ὁ εὐσεβὴς παιδοτρίβης, ὁ ἀπλανὴς ὁδηγὸς τῆς
τοῦ Χριστοῦ ποίμνης, ὁ λογικὸς παράδεισος τοῦ θεοῦ,
ὁ πάσης ἀρετῆς τὰ φυτὰ τεθηλότα ἐν ἑαυτῷ κεκτημένος,\CCSG{80}
τὸ πολυανθὲς τῆς πίστεως βλάστημα, τὸ τῆς ἀγάπης
εὐωδέστατον ῥόδον, τὸ τῆς ἐλπίδος ἀμαράντινον ἄνθος,
ὁ τῆς χάριτος εὐωδέστατος στάχυς, τῆς ζωοποιοῦ ἀμ%
πέλου τὸ πολύφορον κλῆμα, τῆς ἀθανασίας ὁ μελίστακτος
βότρυς, ὁ τῆς ὑπομονῆς ἀσάλευτος πρόβολος, ὁ τῆς\CCSG{85}
ἐγκρατείας σταυροφόρος ὁπλίτης, ὁ κατὰ τῶν χριστο%
κτόνων ἄφοβος στρατηλάτης, ἡ τοῦ σταυρωθέντος Χριστοῦ
οὐρανομήκης παραφυάς. Βαρνάβας, ὁ υἱὸς τῆς παρακλή%
σεως, ὁ τῆς εὐσεβείας διδάσκαλος, ὁ στύλος καὶ ὄροφος
τῆς πίστεως, ὁ πύργος ὁ ἄσειστος, ὁ θεμέλιος ὁ ἀρ\CCSG{90}
ραγής, ἡ κρηπὶς ἡ ἀδιάλυτος, ἡ πέτρα ἡ ἀμετακίνητος,
ὁ λιμὴν ὁ γαληνοφόρος τῶν χειμαζομένων, ἡ παράκλησις
πάντων τῶν θλιβομένων, ὁ πιστὸς οἰκονόμος καὶ φρό%
νιμος, ὁ ἄριστος ἀρχιτέκτων ὁ τὸν αἰθέριον πλοῦν καὶ
ἰσάγγελον βίον ἐπὶ γῆς κατορθώσας, ὁ τῶν ἐκκλησιῶν\CCSG{95}
προστάτης, ὁ τῶν πτωχῶν ἀντιλήπτωρ καὶ τῶν πενήτων
χορηγὸς καὶ τροφεύς, ἡ τῶν χηρῶν παρηγορία καὶ τῶν
ὀρφανῶν κηδεμονέστατος πατήρ. Βαρνάβας, ὁ θησαυρὸς
τῶν τοῦ Χριστοῦ μυστηρίων, ὁ ταμίας τῶν ὀρθοδόξων
τῆς ἁγίας τοῦ θεοῦ ἐκκλησίας δογμάτων, ὁ τῶν νο\CCSG{100}
σούντων ἰατρὸς ἀκαπήλευτος, ἡ τῶν ὑγιαινόντων ἀ%
διάπτωτος εὐφροσύνη, ὁ τῆς τοῦ Χριστοῦ ἀγέλης φι%
λάγρυπνος φύλαξ, ὁ βαίνων ἐπὶ γῆς καὶ ἐν οὐρανοῖς
τὸ πολίτευμα ἔχων, ὁ βοήσας ἐν κόσμῳ δόγματα εὐ%
σεβείας καὶ λαβὼν ἐν οὐρανοῖς αἰώνια στέμματα, ὁ τῶν\CCSG{105}
ἐθνῶν θεοφόρος ὁδηγὸς καὶ τῶν ἐκκλησιῶν θεομακά%
ριστος κήρυξ, ὁ λειμὼν τῆς εὐωδίας Χριστοῦ, ἡ ῥοδωνία
τῶν οὐρανοπνόων ἀρετῶν, τὸ πολύφορον γεώργιον τῶν
τοῦ Χριστοῦ χαρισμάτων, ὁ ἐπὶ γῆς μετελθὼν τὴν
πολιτείαν τὴν ἄφθαρτον, κρείττονα βιώσας ἐπὶ γῆς ὡς\CCSG{110}
ἐν οὐρανοῖς οἱ ἀσώματοι ἄγγελοι, ὁ ἐν ἀσκήσει δια%
πρέψας καὶ ἐν θεωρίαις προκόψας καὶ ἐν θαύμασι με%
γαλυνθείς, ὁ τῆς παρθενίας πολύτιμος μαργαρίτης, τῆς
ἁγνείας τὸ ἐκλεκτὸν βηρύλλιον, τῆς σωφροσύνης τὸ
καθαρὸν πάζιον, ὁ τὸν Χριστὸν ἔχων ἐν ἑαυτῷ λαλοῦντα,\CCSG{115}
τὸ σκεῦος τῆς ἐκλογῆς τοῦ θεοῦ, ὁ τὸν κόσμον κα%
ταλείψας καὶ τὰ ἐν αὐτῷ πάντα σκύβαλα ἡγησάμενος,
ἵνα Χριστὸν μόνον κερδήσῃ τὸν βασιλέα τῶν αἰώνων,
ὁ τὸν ἑαυτοῦ σταυρὸν ἀδιστάκτως ἀράμενος καὶ τῷ
Χριστῷ προθύμως ἀκολουθήσας ὡς ἀληθινὸς μαθητής, ὁ\CCSG{120}
τῶν δαιμόνων φυγαδευτὴς καὶ τοῦ διαβόλου τραυματιστής,
ὁ τὴν τετραπέρατον οἰκουμένην τῇ ἀκαμάτῳ πορείᾳ
βαδίσας ὡς λειτουργὸς ἀγαθὸς καὶ φιλοδέσποτος, καὶ
πάντα τὰ ἔθνη διὰ τοῦ εὐαγγελίου προσαγαγὼν τῇ
πίστει τοῦ Χριστοῦ, ὁ ἑαυτὸν ἑτοιμάσας εἰς κατοικητήριον\CCSG{125}
τῆς ὁμοουσίου τριάδος καὶ ναὸς γενόμενος τοῦ ἐπὶ
πάντων θεοῦ. Βαρνάβας, τὸ τῶν Κυπρίων σεμνὸν ἐγ%
καλλώπισμα καὶ τῆς οἰκουμένης ἀκαταγώνιστος πρόμαχος,
ὁ ὑπερεκπερισσοῦ ἀγαπήσας τὸν Χριστὸν καὶ ὑπὲρ αὐτοῦ
τὴν ψυχὴν καθ’ ἑκάστην ἡμέραν προϊέμενος καὶ σὺν\CCSG{130}
αὐτῷ βασιλεύσας εἰς αἰῶνας αἰώνων.
  
  Ἔκαμεν ὁ λόγος, τοῦτον τὸν θεσπέσιον καὶ τριπόθητον
Βαρνάβαν τὸν ἀπόστολον ἐπᾶραι τοῖς ἐγκωμίοις ποθῶν,
καὶ οὔπω ἥψατο τοῦ προοιμίου˙ τοῖς γὰρ ἐπαίνοις ἀπρό%
σιτος τυγχάνει ὁ θαυμάσιος. Διὸ ἀφέντες ὡς ἀνέφικτον\CCSG{135}
τὸν περὶ τῶν ἐγκωμίων λόγον, ὀλίγα τῶν εἰς ἡμᾶς
ἰόντων περὶ τῆς τούτου βιώσεως καὶ τελειώσεως ἔκ τε
τοῦ Στρωματέως καὶ ἐξ ἑτέρων ἀρχαίων συγγραμμάτων
παραθήσομαι τῇ ὑμετέρᾳ ὁσιότητι, καὶ οὕτω τὸ πέρας
ἐπιθήσομεν τῷ διηγήματι, παραχωρήσαντες τῇ θεοπνεύστῳ\CCSG{140}
Γραφῇ τοῦ ἀοιδίμου τὴν κεφαλὴν μεγαλοπρεπῶς στε%
φανῶσαι˙ φησὶ γάρ˙ “Ἦν δὲ Βαρνάβας ἀνὴρ ἀγαθὸς καὶ
πλήρης Πνεύματος ἁγίου καὶ πίστεως.” Οὗ τί ἂν γένοιτο
ἴσον ἢ παραπλήσιον πώποτε;

\saut


Ὡρμᾶτο μὲν οὖν ὁ τρισόλβιος οὗτος ἐκ τῆς εὐλογη\CCSG{145}%
μένης φυλῆς τοῦ Λευΐ, ἐξ ἧς ὑπῆρχον Μωϋσῆς καὶ
Ἀαρών, οἱ μεγάλοι τοῦ θεοῦ προφῆται καὶ πρῶτοι τοῦ
δήμου Καάθ, ἐκ τῆς συγγενείας Σαμουὴλ τοῦ προφήτου
κατάγων τὸ γένος. Οἱ δὲ τούτου πρόγονοι, διὰ περί%
στασιν πολέμων καταλαβόντες τὴν Κυπρίων χώραν, φι\CCSG{150}%
λοφρόνως ᾤκουν αὐτήν˙ ἦσαν δὲ εὐλαβεῖς κατὰ τὸν
νόμον καὶ πλούσιοι σφόδρα, ὅθεν καὶ ἐν Ἱεροσολύμοις
εἶχον οὐσίαν ἱκανὴν καὶ ἀγρὸν πλησίον τῆς πόλεως
περικαλλέστατον, οὐ μόνον παντοίοις καρπίμοις φυτοῖς
ὡραϊζόμενον, ἀλλὰ καὶ μεγέθει κτισμάτων περιφανέστατον˙\CCSG{155}
ἐξότε γὰρ ἤκουσαν Ἡσαΐου τοῦ προφήτου λέγοντος˙
\enquote{Μακάριος ὁ ἔχων σπέρμα ἐν Σιὼν καὶ οἰκείους ἐν
Ἱερουσαλήμ}, σαρκικῶς τὴν προφητείαν ἐκδεχόμενοι
Ἑβραίων παῖδες, ἔσπευδεν ἕκαστος τῶν εὐπορούντων
ἔχειν τὸ κτῆμα ἐν Ἱερουσαλήμ.\CCSG{160}

Τοῦ δὲ δικαίου τούτου τεχθέντος ἐν Κύπρῳ, ὡς εἶδον
αὐτὸν οἱ γονεῖς αὐτοῦ ἀστεῖον ὄντα τῷ θεῷ, εὐθέως
μὲν αὐτὸν Ἰωσὴφ ἐπωνόμασαν, τῇ τοῦ πατριάρχου
προσηγορίᾳ τιμῶντες τὸν παῖδα˙ συνέδραμε δὲ τῇ προση%
γορίᾳ τοῦ ὀνόματος ἡ τῶν τρόπων εὐγένεια˙ Ἰωσὴφ\CCSG{165}
γὰρ ἑρμηνεύεται πρό[σ]θεσις θεοῦ˙ προσθήκην γὰρ χά%
ριτος ἔλαβε παρὰ θεοῦ ὁ δίκαιος, ὅπως φθάσῃ εἰς τὴν
ἀποστολικὴν τελειότητα˙ πάλιν δὲ Ἰωσὴφ ἑρμηνεύεται
δόξα θεοῦ˙ γέγονε γὰρ διὰ τῆς ἀρίστης πολιτείας δόξα
θεοῦ˙ καὶ μηδεὶς ὑπερβολὴν εἶναι νομιζέτω τὸ ῥῆμα,\CCSG{170}
ἀλλ’ ἐπιστησάτω ὅτι τῆς θείας ἐστὶ Γραφῆς˙ φησὶ γοῦν
ὁ Παῦλος˙ \enquote{Ἀνὴρ οὐκ ὀφείλει κατακαλύπτεσθαι τὴν
κεφαλήν, εἰκὼν καὶ δόξα θεοῦ ὑπάρχων.} Εἰ οὖν τὸν
κοινὸν ἄνδρα εἰκόνα καὶ δόξαν θεοῦ ἐκάλεσε τὸ σκεῦος
τῆς ἐκλογῆς, τί ἄν τις εἴποι περὶ τοῦ κατὰ θεὸν\CCSG{175}
τελειοτάτου ἀνδρός;

Ἐπειδὴ δὲ αὐξήσας ὁ Βαρνάβας ἐγένετο παῖς, ἀνήγαγον
οἱ γονεῖς εἰς Ἱεροσόλυμα καὶ παρέδωκαν αὐτὸν μαθεῖν
τὸν νόμον ἀκριβῶς καὶ τοὺς προφήτας παρὰ τοὺς πόδας
Γαμαλιήλ˙ εἶχε δὲ τὸν Παῦλον συμφοιτητήν, Σαῦλον τέως\CCSG{180}
ὀνομαζόμενον. Προέκοπτε δὲ ὁσημέραι ὁ Βαρνάβας ἔν
τε τῇ μαθήσει καὶ ἐν πάσῃ ἀρετῇ˙ οὔπω μέντοι εἰς
τὴν ἐφημερίαν τῶν Λευϊτῶν ἐτέτακτο διὰ τὸ ἐλλιπὲς τῆς
ἡλικίας˙ ἔτι γὰρ ἔφηβος ἦν. Οὐκ ἀφίστατο δὲ ἀπὸ τοῦ
ἱεροῦ, νηστείαις καὶ δεήσεσι λατρεύων νύκτα καὶ ἡμέραν˙\CCSG{185}
οὕτω δὲ ἀπήγγειλε τὸν νόμον καὶ τὰς λοιπὰς γραφάς,
ὡς μὴ δεῖσθαι τῆς ἀπὸ τῶν γραμμάτων ὑπομνήσεως˙
τὴν δὲ ἡσυχίαν οὕτως εἶχε φίλην, ὡς μητέρα τῆς
σωφροσύνης˙ τὰς δὲ ἐπιβλαβεῖς ὁμιλίας μυσαττόμενος
ἔφευγε, καθαρὸν ὑπάρχων καὶ ἀκέραιον καὶ ἀμόλυντον\CCSG{190}
ἄγαλμα˙ ἦν δὲ πᾶσιν ἐπίδοξος δι’ ἀρετήν.

Κατ’ ἐκεῖνον δὲ τὸν καιρὸν συνέβη τὸν κύριον παρα%
γενέσθαι εἰς Ἱεροσόλυμα καὶ ἰάσασθαι τὸν παράλυτον
ἐν τῇ προβατικῇ, καὶ ἕτερα πολλὰ σημεῖα καὶ τέρατα
ἐργάσασθαι ἐν τῷ ἱερῷ. Ταῦτα θεασάμενος ὁ μακάριος,  \CCSG{195}
ἐξεπλήσσετο καὶ εὐθέως προσελθών, ἔπεσε παρὰ τοὺς
πόδας αὐτοῦ καὶ ἐδέετο εὐλογηθῆναι παρ’ αὐτοῦ. Ὁ δὲ
τὰς καρδίας ἐμβατεύων Χριστός, ἀποδεξάμενος αὐτοῦ τὴν
πίστιν, εὐμενῶς αὐτὸν ὑπεδέξατο καὶ τῆς θείας αὐτοῦ
συντυχίας μετέδωκεν˙ ὁ δὲ πλέον ἐξεκαίετο εἰς τὴν τοῦ\CCSG{200}
κυρίου ἀγάπην. Καταλαβὼν δὲ τὸ τάχος τὴν οἰκίαν
Μαρίας, τῆς μητρὸς Ἰωάννου τοῦ ἐπικαλουμένου Μάρκου,
ἥτις ἐλέγετο εἶναι αὐτοῦ θεία --- διὸ καὶ Μάρκον τὸν
ἀνεψιὸν Βαρνάβα ἐκάλουν αὐτόν ---, εἶπε πρὸς αὐτήν˙
\enquote{Δεῦρο, λέγων, ὦ γύναι, ἴδε ἅπερ ἐπεθύμουν ἰδεῖν οἱ\CCSG{205}
πατέρες ἡμῶν˙ ἰδοὺ γὰρ Ἰησοῦς τίς προφήτης ἀπὸ
Ναζαρὲτ τῆς Γαλιλαίας ἐστὶν ἐν τῷ ἱερῷ, μεγαλοπρεπῶς
θαυματουργῶν, καὶ ὡς τοῖς πολλοῖς δοκεῖ, αὐτός ἐστι
Μεσσίας ὁ μέλλων ἔρχεσθαι.} Ἀκούσασα δὲ ταῦτα ἡ
θαυμασία γυνὴ καὶ καταλιποῦσα τὰ ἐν χερσί, κατέλαβε\CCSG{210}
τὸν ναὸν τοῦ θεοῦ, καὶ ἰδοῦσα τὸν κύριον καὶ δεσπότην
τοῦ ναοῦ, ἔπεσεν εἰς τοὺς πόδας αὐτοῦ, δεομένη καὶ
λέγουσα˙ \enquote{Κύριε, εἰ εὖρον χάριν ἐναντίον σου, δεῦρο
εἰς τὸν οἶκον τῆς δούλης σου καὶ εὐλόγησον τῇ εἰσόδῳ
σου τὰ οἰκετικά σου.} Ὁ δὲ κύριος ἐπένευσε τῇ\CCSG{215}
παρακλήσει˙ ὃν παραγενόμενον ὑπεδέξατο χαίρουσα εἰς
τὸ ὑπερῶον αὐτῆς. Ἀπ’ ἐκείνης οὖν τῆς ἡμέρας, ἡνίκα
ἤρχετο ὁ κύριος εἰς Ἱεροσόλυμα, ἐκεῖ ἀνεπαύετο μετὰ
τῶν μαθητῶν αὐτοῦ, ἐκεῖ ἐποίησε τὸ Πάσχα μετὰ τῶν
μαθητῶν αὐτοῦ, ἐκεῖ ἐμυσταγώγησε τοὺς μαθητὰς διὰ\CCSG{220}
τῆς μεταλήψεως τῶν ἀπορρήτων μυστηρίων. Λόγος γὰρ
ἦλθεν εἰς ἡμᾶς ἀπὸ γερόντων ὅτι ὁ τὸ κεράμιον βα%
στάζων τοῦ ὕδατος, ᾧ κατακολουθῆσαι προσέταξεν ὁ
κύριος τοῖς μαθηταῖς, Μάρκος ἦν, ὁ υἱὸς ταύτης τῆς
μακαρίας Μαρίας˙ ὁ δὲ κύριος \enquote{πρὸς τὸν δεῖνα} εἶπεν\CCSG{225}
οἰκονομικῶς, ὡς φασὶν οἱ πατέρες, ἑρμηνεύοντες τοῦτο
τὸ χωρίον, διδάσκων ἡμᾶς διὰ τοῦ αἰνίγματος ὅτι παντὶ
τῷ εὐτρεπίζοντι ἑαυτόν, παρ’ αὐτῷ ὁ κύριος αὐλίζεται.
Ἐν αὐτῷ τοίνυν τῷ ὑπερώῳ ἐποίησεν ὁ κύριος τὸ
Πάσχα˙ ἐν αὐτῷ ἐφάνη τοῖς περὶ τὸν Θωμᾶν, ἐγερθεὶς\CCSG{230}
ἐκ νεκρῶν˙ ἐκεῖ μετὰ τὴν ἀνάληψιν ἀνῆλθον οἱ μαθηταί,
ἐλθόντες ἀπὸ τοῦ ὄρους τῶν Ἐλαιῶν μετὰ τῶν λοιπῶν
ἀδελφῶν, ὄντων τὸν ἀριθμὸν ὡς ἐκατὸν εἴκοσιν, ἐν οἷς
ἦν Βαρνάβας καὶ Μάρκος˙ ἐκεῖ κατέβη τὸ Πνεῦμα τὸ
ἅγιον ἐν πυρίναις γλώσσαις ἐπὶ τοὺς μαθητὰς ἐν τῇ\CCSG{235}
ἡμέρᾳ τῆς Πεντηκοστῆς˙ ἐκεῖ ἵδρυται νῦν ἡ μεγάλη καὶ
ἁγιωτάτη Σιών, ἡ μήτηρ πασῶν τῶν ἐκκλησιῶν.

Τότε δὲ Βαρνάβας ἠκολούθησε τῷ κυρίῳ, ὑποστρέφοντι
ἀπὸ Ἱεροσολύμων εἰς τὴν Γαλιλαίαν˙ πολλῶν δὲ παν%
ταχόθεν προσερχομένων τῷ κυρίῳ καὶ εἰς αὐτὸν πι\CCSG{240}%
στευόντων, τότε εἶπε τοῖς μαθηταῖς˙ \enquote{Ὁ μὲν θερισμὸς
πολύς, οἱ δὲ ἐργάται ὀλίγοι.} Τότε ἀνέδειξε τοὺς ἑ%
βδομήκοντα μαθητάς, ὧν πρῶτος καὶ ἔξαρχος καὶ κο%
ρυφαῖος ὁ μέγας Βαρνάβας ἐτύγχανεν. Ἀλλὰ μηδείς,
ἀκούων ὅτι οἱ ἀπόστολοι ἐπέθηκαν αὐτῷ τὴν προση\CCSG{245}%
γορίαν ταύτην, οἰέσθω ἀμαθῶς ὅτι ἄνευ θείας ἐπιπνοίας
τὴν προσηγορίαν ταύτην ἐδέξατο˙ Πέτρος γὰρ αὐτῷ
ἐπέθηκε τὸ ὄνομα τοῦτο, δι’ ἀποκαλύψεως τοῦ ἁγίου
Πνεύματος, ὁ καὶ δι’ ἀποκαλύψεως τοῦ Πατρὸς τὴν τοῦ
Υἱοῦ θεολογίαν δεξάμενος˙ καὶ ὥσπερ Ἰάκωβος καὶ\CCSG{250}
Ἰωάννης υἱοὶ βροντῆς ἀπὸ τῆς ἀρετῆς ἐκλήθησαν, οὕτω
καὶ Βαρνάβας ἀπὸ τῆς ἀρετῆς ἐκλήθη υἱὸς παρακλήσεως,
πάντων παράκλησις γενόμενος δι’ ὑπερβολὴν ἁγιότητος.

Ἀκούσας δὲ τοῦ κυρίου διδάσκοντος καὶ λέγοντος˙
\enquote{Πωλήσατε τὰ ὑπάρχοντα ὑμῶν καὶ δότε ἐλεημοσύνην˙\CCSG{255}
ποιήσατε ἑαυτοῖς βαλλάντια μὴ παλαιούμενα, θησαυρὸν
ἀνέκλειπτον ἐν τοῖς οὐρανοῖς,} μηδὲν μελλήσας, εὐθέως
τὴν ὑπολειφθεῖσαν αὐτῷ οὐσίαν, πολύτιμον οὖσαν, ὑπὸ
τῶν γονέων --- ἦσαν γὰρ ὑπεξελθόντες τὸν βίον --- 
πᾶσαν καταπωλήσας, διένειμε τοῖς χρείαν ἔχουσι, κα\CCSG{260}%
ταλείψας ἑαυτῷ μόνον τὸν ἀγρὸν ἐκεῖνον εἰς ἰδίαν
ἀποτροφήν. Μετὰ δὲ τὸ πάθος καὶ τὴν ἀνάστασιν καὶ
τὴν ἀνάληψιν τοῦ κυρίου καὶ τὴν τοῦ ἁγίου Πνεύματος
ἐπιφοίτησιν, ἐπὶ πλεῖον πυρωθεὶς ὁ θεῖος Βαρνάβας
τῇ εἰς τὸν κύριον ἀγάπῃ καὶ αὐτὸν τὸν ἀγρὸν ἐκεῖνον\CCSG{265}
ἀποδόμενος καὶ λαβὼν ἱκανὰ χρήματα, ἅπαντα ἐνέγκας
παρὰ τοὺς πόδας τῶν ἀποστόλων ἔθηκε, μηδὲν τὸ
σύνολον καταλείψας ἑαυτῷ ἐξ αὐτῶν, τῷ καθ’ ἑαυτὸν
ὑποδείγματι εἰς τὴν ὁμοίαν ἀρετὴν πάντας τοὺς μαθητὰς
διεγείρων.\CCSG{270}

Ἐλάλει δὲ καὶ συνεζήτει πρὸς τὸν Σαῦλον, βουλόμενος
εἰς τὴν τοῦ κυρίου αὐτὸν πίστιν ἀγαγεῖν. Σαῦλος δὲ
τῇ κατὰ τὸν νόμον ἀκριβεῖ πολιτείᾳ δῆθεν ἐπερειδόμενος,
τοῦ μὲν Βαρνάβα κατεγέλα ὡς ἀπατηθέντος, τὸν δὲ
κύριον ἐβλασφήμει, υἱὸν τοῦ τέκτονος αὐτὸν ἀποκαλῶν\CCSG{275}
καὶ ἰδιώτην καὶ ἄγροικον καὶ βιοθανῆ. Ὡς δὲ εἶδε τὰ
διὰ τῶν ἀποστόλων γινόμενα τῶν θαυμάτων μεγα%
λουργήματα καὶ τοῦ λαοῦ τὸ πλῆθος τῶν καθ’ ἡμέραν
προστιθεμένων τῷ λόγῳ τῆς πίστεως, ἐδάκνετο τὴν
ψυχήν. Προσβαλὼν δὲ μετὰ τῶν Λιβερτίνων καὶ Κυρη\CCSG{280}%
ναίων καὶ Ἀλεξανδρέων τῷ μεγάλῳ τῆς ἐκκλησίας ῥήτορι
Στεφάνῳ, καὶ μὴ δυνηθεὶς ἀντιστῆναι τῇ σοφίᾳ καὶ
τῷ Πνεύματι, ᾧ ἐλάλει, εἰς μανίαν ἐτράπη˙ καὶ γενόμενος
πλήρης θυμοῦ, ἤγειρε κατ’ αὐτοῦ τοὺς ἀτάκτους τοῦ λαοῦ,
καὶ τοῦτον ἀνελών, ἐπήγειρε διωγμὸν μέγαν ἐπὶ τὴν\CCSG{285}
ἐκκλησίαν τὴν ἐν Ἱεροσολύμοις. Ἀλλ’ οὕτως αὐτὸν ἀτάκτως
πορευόμενον εἰς Δαμασκὸν ἐπὶ κακοποιήσει τῶν πιστῶν,
ὁ κύριος ὑποσκελίσας, ἔβαλεν ἐπὶ πρόσωπον˙ ὁ δὲ
πεσὼν εἰς τὴν γῆν, ἐπέγνω τίνα διώκει, καὶ πηρωθεὶς
τὰς ὄψεις, ἀνέβλεψε διόλου εἰς τὸ ὕψος τοῦ οὐρανοῦ.\CCSG{290}
Ὑποστρέψας δὲ εἰς Ἱερουσαλήμ, ἐζήτει κολλᾶσθαι τοῖς
μαθηταῖς, καὶ πάντες ἔφευγον ἀπ’ αὐτοῦ, φοβούμενοι τὴν
πολλὴν αὐτοῦ ὠμότητα. Ὁ δὲ μέγας Βαρνάβας, ἀπαντήσας
αὐτῷ, εἶπεν˙ \enquote{Ἕως πότε, Σαῦλε, Σαοὺλ τυγχάνεις; Ἵνα
τί οὕτως ἰταμῶς τὸν εὐεργέτην διώκεις; Παῦσαι πορθῶν\CCSG{295}
τὸ ὑπὸ τῶν προφητῶν πάλαι βοώμενον φρικτὸν μυστή%
ριον καὶ ἐν τοῖς ἡμετέροις καιροῖς ἀποκαλυφθὲν εἰς
σωτηρίαν ἡμῶν.} Ταῦτα ἀκούσας ὁ Σαῦλος, ἔπεσεν ἐπὶ
τοὺς πόδας Βαρνάβα, μετὰ πολλῶν δακρύων κράζων καὶ
λέγων˙ \enquote{Συγχώρησόν μοι, ὁδηγὲ τοῦ φωτὸς καὶ διδά\CCSG{300}%
σκαλε τῆς ἀληθείας˙ ἔγνων γὰρ τῇ πείρᾳ τῶν λόγων
σου τὴν ἀλήθειαν˙ ὃν γὰρ ἐγὼ βλασφημῶν ἔλεγον υἱὸν
τοῦ τέκτονος, νῦν ὁμολογῶ αὐτὸν τοῦ θεοῦ τοῦ ζῶντος
Υἱὸν μονογενῆ, ὁμοούσιόν τε καὶ ὁμόδοξον καὶ ὁμό%
θρονον, συναΐδιόν τε καὶ συνάναρχον˙ ὃς ὢν ἀπαύγασμα\CCSG{305}
τῆς δόξης καὶ χαρακτὴρ τῆς ὑποστάσεως τοῦ ἀοράτου
θεοῦ, ἐπ’ ἐσχάτου τῶν ἡμερῶν τούτων, δι’ ἡμᾶς καὶ διὰ
τὴν ἡμετέραν σωτηρίαν, ἐκένωσεν ἑαυτόν, μορφὴν δούλου
λαβών, τουτέστι τέλειον ἄνθρωπον ἐκ τῆς ἁγίας παρθένου
καὶ θεοτόκου Μαρίας, ἀσυγχύτως, ἀτρέπτως, ἀδιαιρέτως,\CCSG{310}
ἀχωρίστως˙ καὶ σχήματι εὑρεθεὶς ὡς ἄνθρωπος, ἐτα%
πείνωσεν ἑαυτόν, ὑπήκοος γενόμενος μέχρι θανάτου, θα%
νάτου δὲ σταυροῦ˙ ὃς καὶ ἀνέστη ἐκ νεκρῶν τῇ τρίτῃ
ἡμέρᾳ καὶ ὤφθη ὑμῖν, τοῖς ἀποστόλοις αὐτοῦ, καὶ
ἀνελήφθη εἰς τοὺς οὐρανοὺς καὶ ἐκάθισεν ἐν δεξιᾷ τοῦ\CCSG{315}
Πατρὸς καὶ πάλιν ἔρχεται μετὰ δόξης κρῖναι ζῶντας
καὶ νεκρούς, καὶ τῆς βασιλείας αὐτοῦ οὐκ ἔσται τέλος.}

Ταῦτα ἀκούσας ὁ θεσπέσιος Βαρνάβας παρὰ τοῦ
βλασφήμου καὶ διώκτου, ἐξεπλάγη καὶ ὑπὸ τῆς χαρᾶς
ἐγένετο τὸ πρόσωπον αὐτοῦ ὡσεὶ ἄνθος πρωϊνόν˙ συμ\CCSG{320}%
περιλαβόμενος δὲ αὐτὸν καὶ καταφιλήσας, εἶπεν˙ \enquote{Τίς
σε, Σαῦλε, τοιαῦτα ἐδίδαξε θεόπνευστα ῥήματα φθέγ%
γεσθαι; Ἢ τίς σε ἔπεισεν Ἰησοῦν τὸν Ναζωραῖον Υἱὸν
θεοῦ ὁμολογεῖν; Ἢ πόθεν ἔμαθες οὐρανίων δογμάτων
τοσαύτην ἀκρίβειαν;} Ὁ δὲ κεκυφὼς καὶ δακρύων, μετὰ\CCSG{325}
πολλῆς τῆς κατανύξεως ἔφη˙ \enquote{Αὐτὸς ὁ κύριος Ἰησοῦς,
ὁ πολλάκις βλασφημηθεὶς καὶ διωχθεὶς ὑπ’ ἐμοῦ τοῦ
ἁμαρτωλοῦ, ἐδίδαξέ με ταῦτα πάντα˙ ὥσπερ γὰρ εἰ
τῷ ἐκτρώματι ὤφθη κἀμοὶ καὶ ἔτι ἔναυλον ἐν τοῖς ὠσὶν
ἔχω τὴν θείαν καὶ γλυκεῖαν αὐτοῦ φωνήν˙ πᾶσαν γὰρ\CCSG{330}
ἀγαθότητα ὑπερβάς, εἴρηκέ μοι, κειμένῳ ἐλεεινῶς ἐπὶ
πρόσωπον, ἀπολογούμενος μᾶλλον ἢ ἐγκαλῶν˙ \enquote{Σαούλ,
Σαούλ, τί με διώκεις;} Ἐγὼ δὲ μετὰ φρίκης καὶ φόβου
ἀπεκρίθην˙ \enquote{Τίς εἶ, κύριε;} Ὁ δὲ κύριος μετὰ πολλῆς
τῆς ἐπιεικείας καὶ συμπαθείας εἶπε μοι˙ \enquote{Ἐγώ εἰμι Ἰησοῦς\CCSG{335}
ὁ Ναζωραῖος, ὃν σὺ διώκεις.} Ἐκπλαγεὶς δὲ ἐγὼ ἐπὶ
τῇ ἀφάτῳ αὐτοῦ ἀνεξικακίᾳ, ἐδεήθην αὐτοῦ, λέγων˙ \enquote{Τί
ποιήσω, κύριε;} Ὁ δὲ παραχρῆμα συνεβίβασέ με ταῦτα
πάντα καὶ ἔτι πλείονα τούτων.}

Τότε ὁ μέγας Βαρνάβας ἐπιλαβόμενος τῆς χειρὸς αὐτοῦ,\CCSG{340}
ἤγαγε πρὸς τοὺς ἀποστόλους, λέγων˙ \enquote{Τί φεύγετε τὸν
ποιμένα, λύκον αὐτὸν εἶναι ὑπολαμβάνοντες; Τί τὸν κυ%
βερνήτην ὡς πειρατὴν διώκετε; Τί τὸν ἀριστέα ὡς
προδότην μυσάττεσθε; Τί τὸν νυμφαγωγὸν ὡς ληστὴν
τοῦ παστοῦ ἀποπέμπετε; Παστὸς γὰρ πνευματικὸς ἡ\CCSG{345}
ἐκκλησία τυγχάνει, ἧς ποιμένα καὶ κυβερνήτην καὶ ὑπέρ%
μαχον ὁ κύριος δι’ ἑαυτοῦ ἐχειροτόνησεν.} Τότε διηγήσατο
αὐτοῖς ὁ Παῦλος ὅσα συνέβη αὐτῷ κατὰ τὴν ὁδόν, καὶ
ὅτι εἶδε τὸν κύριον καὶ ὅτι ἐλάλησεν αὐτῷ, καὶ πῶς
ἐν Δαμασκῷ ἐπαρρησιάσατο ἐπὶ τῷ ὀνόματι τοῦ κυρίου˙\CCSG{350}
καὶ ἦν διδάσκων σὺν αὐτοῖς τὸν λόγον τοῦ κυρίου ἐν
Ἱερουσαλήμ. Ἦν δὲ βαρὺς τοῖς Ἰουδαίοις σφόδρα, ὅτι
ὁ χθὲς τὸν Ἰησοῦν διώκων, τοῦτον σήμερον Υἱὸν θεοῦ
κηρύττει, καὶ ἐβουλεύσαντο ἀνελεῖν αὐτόν. Μαθόντες δὲ
οἱ ἀπόστολοι, ἀπέστειλαν αὐτὸν κηρύξαι ἐν τῇ ἰδίᾳ\CCSG{355}
πατρίδι.

Τῶν δὲ διασπαρέντων ἐπὶ τῆς θλίψεως τοῦ Στεφάνου,
ἐλθόντων εἰς Ἀντιόχειαν καὶ εὐαγγελισαμένων τὸν κύριον
Ἰησοῦν, ἠκούσθη περὶ αὐτῶν ἐν Ἱεροσολύμοις. Τότε οἱ
ἀπόστολοι τὸν μακάριον Βαρνάβαν, ὡς μέγαν καὶ δυ\CCSG{360}%
νατόν, ἀπέστειλαν τῇ ἐκεῖσε ἁγιωτάτῃ ἐκκλησίᾳ ποιμαί%
νειν τὸ ποίμνιον τοῦ Χριστοῦ˙ ὃς παραγενόμενος,
συνεβάλλετο πολὺ τοῖς πεπιστευκόσι, καὶ διὰ τῆς ἐνθέου
αὐτοῦ διδασκαλίας προσετέθη λαὸς ἱκανὸς τῷ κυρίῳ.
Ἐκεῖθεν, ὑπὸ τοῦ ἁγίου Πνεύματος ὁδηγούμενος, ἐξελθών,\CCSG{365}
διῆλθεν εὐαγγελιζόμενος τὰς πόλεις πάσας καὶ χώρας,
ἕως τοῦ ἐλθεῖν αὐτὸν εἰς τὴν μεγίστην Ῥώμην˙ αὐτὸς
γὰρ πρὸ παντὸς ἑτέρου τῶν τοῦ κυρίου μαθητῶν ἐκή%
ρυξεν ἐν Ῥώμῃ τὸ εὐαγγέλιον τοῦ Χριστοῦ. Πολλῶν
δὲ πιστευσάντων καὶ ὑπερβαλλόντως τιμώντων αὐτόν,\CCSG{370}
τὴν τῶν ἀνθρώπων δόξαν ἀποβαλών, λάθρᾳ φυγών,
ἐξῆλθε τῆς Ῥώμης˙ καὶ γὰρ πάλιν ὁ μακάριος οὗτος
πάντας τοὺς κατ’ ἐκεῖνο καιροῦ ἀνθρώπους ἐπλεονέκτει
εἰς τὴν ταπεινοφροσύνην, εἰς ἄκρον αὐτὴν κατορθώσας˙
καὶ δῆλον τοῦτο πᾶσι καθέστηκεν ἐκ τῆς κατ’ αὐτὸν\CCSG{375}
ἱστορίας. Τῆς γὰρ θεοπνεύστου Γραφῆς πανταχοῦ αὐτὸν
πρῶτον ὀνομαζούσης, αὐτὸς τὰ πρωτεῖα τοῖς περὶ αὐτὸν
παραχωρῶν, τὴν δευτέραν τάξιν ἠσπάζετο, ἀκριβῶς μιμη%
σάμενος τὸν κύριον τὸν εἰρηκότα˙ \enquote{Μάθετε ἀπ’ ἐμοῦ ὅτι
πρᾶός εἰμι καὶ ταπεινὸς τῇ καρδίᾳ.}\CCSG{380}

Καταλαβὼν δὲ Βαρνάβας Ἀλεξάνδρειαν τὴν πρὸς Αἴ%
γυπτον καὶ λαλήσας ἐκεῖ τὸν λόγον τοῦ θεοῦ, ἐξῆλθε,
διερχόμενος καθεξῆς τὰς πόλεις πάσας, ἕως τοῦ ἐλθεῖν
αὐτὸν εἰς Ἱεροσόλυμα. Ἐκεῖθεν πάλιν ἐξελθών, κατῆλθεν
εἰς Ἀντιόχειαν καὶ ἰδὼν τὴν χάριν τοῦ θεοῦ καὶ τὴν\CCSG{385}
ἐκκλησίαν πληθυνθεῖσαν, ἐχάρη λίαν. Τότε ἐξῆλθεν εἰς
Ταρσόν, ζητῶν τὸν Παῦλον, καὶ εὑρών, ἤγαγεν εἰς
Ἀντιόχειαν˙ καὶ ποιήσαντες ἐκεῖ ἐνιαυτὸν ὅλον καὶ μαθη%
τεύσαντες λαὸν ἱκανὸν καὶ καλέσαντες ἐκεῖ πρῶτον τοὺς
μαθητὰς Χριστιανούς, λαβόντες δὲ τὰ εἰς διάδοσιν τῶν\CCSG{390}
πτωχῶν παρὰ τῆς ἐκκλησίας, ἀνῆλθον πάλιν εἰς Ἱερο%
σόλυμα μετὰ δεκατέσσαρα ἔτη τοῦ σωτηρίου πάθους,
καθὼς γράφει Παῦλος, λέγων˙ \enquote{Διὰ δεκατεσσάρων ἐτῶν
ἀνῆλθον εἰς Ἱεροσόλυμα μετὰ Βαρνάβα.} Πληρώσαντες
δὲ τὴν διακονίαν καὶ λαβόντες δεξιὰν παρὰ τῶν ἀπο\CCSG{395}%
στόλων, ἵνα αὐτοὶ μὲν εἰς τὰ ἔθνη κηρύξωσιν, οἱ δὲ
περὶ τὸν Πέτρον εἰς τὴν περιτομήν, κατῆλθον εἰς Ἀν%
τιόχειαν, ἔχοντες μεθ’ ἑαυτῶν Μάρκον ὑπηρέτην. Ἀπὸ δὲ
τῆς Ἀντιοχέων ἐκπεμφθέντες ὑπὸ τοῦ ἁγίου Πνεύματος,
ἦλθον εἰς Κύπρον˙ καὶ διοδεύσαντες ὅλην τῆν νῆσον\CCSG{400}
ἀπὸ Σαλαμίνης ἕως Πάφου, εὐαγγελιζόμενοι καὶ θαυ%
ματουργοῦντες, ὅτε καὶ τὸν Ἐλύμαν ἐτύφλωσαν καὶ τὸν
ἀνθύπατον ἐφώτισαν, καὶ μαθητεύσαντες ἱκανούς, τότε
ἀναχθέντες ἀπὸ τῆς Κύπρου οἱ περὶ Βαρνάβαν, ἦλθον
εἰς Παμφυλίαν˙ ἰδὼν δὲ ὁ Μάρκος τοὺς ἀποστόλους\CCSG{405}
ὁμόσε χωροῦντας πρὸς τοὺς ὑπὲρ τοῦ εὐαγγελίου κιν%
δύνους, καὶ ὅτι ὅπου ἤμελλον τιμᾶσθαι, καταλιπόντες,
πρὸς τὸν κατὰ τῶν ἀπίστων πόλεμον ἐχώρουν, ὀκλάσας
πρὸς τὰ δεινά, ἅτε μειράκιον ὑπάρχων, δειλὸς καὶ ἀτελὴς
πρὸς τὴν κατὰ τοῦ θανάτου ὑπεροψίαν, καταλιπὼν τοὺς\CCSG{410}
ἀποστόλους, ὑπέστρεψεν εἰς Ἱερουσαλὴμ πρὸς τὴν ἰδίαν
μητέρα.

Ὡς δὲ ἐπλήρωσαν οἱ ἀπόστολοι Βαρνάβας καὶ Παῦλος
τὸ ἔργον, εἰς ὃ ἐκλήθησαν, μυρίους ὑπὲρ Χριστοῦ
ἀγῶνας ὑπομείναντες, ἦλθον εἰς Ἀντιόχειαν˙ ὅθεν ὑπὸ\CCSG{415}
τῆς χάριτος τοῦ θεοῦ ἀπεστάλησαν εἰς τὰ ἔθνη. Ἐγένετο
δὲ αὐτοῖς χρεία πάλιν ἀνελθεῖν εἰς Ἱεροσόλυμα πρὸς
τοὺς περὶ Πέτρον ἀποστόλους, ἕνεκα τῶν ψευδαποστόλων
τῶν διδασκόντων τοὺς μαθητὰς περιτέμνεσθαι καὶ τηρεῖν
τὸν νόμον. Ἰδὼν δὲ αὐτοὺς ὁ Μάρκος ὑπὸ πάντων\CCSG{420}
τιμωμένους καὶ ὅτι μετὰ τοσαύτας αἰκίας καὶ κινδύνους
ἐρρωμένοι εἰσὶ καὶ ὁλόκληροι, καταγνοὺς τῆς ἑαυτοῦ
ἀνανδρίας, ἔκλαυσε πικρῶς˙ καὶ τῷ μὲν Παύλῳ προ%
σελθεῖν ᾐσχύνετο, τῷ δὲ Βαρνάβᾳ προσῆλθε μετὰ δα%
κρύων καὶ προσέπεσε τοῖς ποσὶν αὐτοῦ, δεόμενος\CCSG{425}
συγχώρησιν λαβεῖν ὑπὲρ τῶν παρελθόντων καὶ εὐχὴν
ὑπὲρ ἀσφαλείας τῶν μελλόντων ὑποσχόμενος καὶ λέγων
ὅτι πᾶσαν ἰδέαν θανάτου ἑτοίμως ἔχω ὑπομεῖναι ὑπὲρ
τοῦ ὀνόματος τοῦ κυρίου ἡμῶν Ἰησοῦ Χριστοῦ. Ἐπι%
καμφθεὶς δὲ τοῖς πολλοῖς δάκρυσιν ὁ μέγας οὗτος ἐν\CCSG{430}
ἀρεταῖς Βαρνάβας, παρεκάλει αὐτὸν παύσασθαι τοῦ κλαυ%
θμοῦ, λέγων˙ \enquote{Τὸ θέλημα τοῦ κυρίου γενέσθω. Μόνον
σὺ ἕτοιμος ἔσο τὰς ὑποσχέσεις σου πληρῶσαι.} Λαβόντες
δὲ ἐγγράφως τὰ δογματισθέντα ὑπὸ τῶν ἐν Ἱεροσολύμοις
ἀποστόλων, κατῆλθον εἰς Ἀντιόχειαν καὶ ἦσαν ἀγαλλιώ\CCSG{435}%
μενοι μετὰ τῶν ἀδελφῶν˙ Μάρκος δὲ ἠκολούθει τῷ
Βαρνάβᾳ, οὐ μετὰ παρρησίας.

Μετὰ δὲ ταῦτα ἔδοξε τοῖς περὶ Βαρνάβαν καὶ Παῦλον
διελθεῖν τὰς πόλεις πάσας καὶ ἐπισκέψασθαι τοὺς ἀδελ%
φούς. Τότε Βαρνάβας προσελθών, παρεκάλει τὸν Παῦλον,\CCSG{440}
ὥστε συνέκδημον αὐτοῖς γενέσθαι τὸν Μάρκον, ἑτοίμως
ἔχοντα ἕως θανάτου ἀγωνίσασθαι ὑπὲρ τῆς εἰς Χριστὸν
πίστεως˙ ὁ δὲ Παῦλος ἀντεπαρεκάλει τοῦτον μεθ’ ἑαυτῶν
μὴ συμπαραλαμβάνεσθαι. Ἐντεῦθεν γέγονε πρόφασις τοῦ
ἀποχωρισθῆναι αὐτοὺς ἀπ’ ἀλλήλων, τοῦ κυρίου τοῦτο\CCSG{445}
οἰκονομήσαντος εἰς τὸ συμφέρον˙ ἤμελλε γὰρ ὁ θεὸς
τὸν Μάρκον λαῶν καὶ ἐθνῶν ἀναδεῖξαι ποιμένα καὶ
διδάσκαλον. Ἀλλὰ μηδεὶς ἀνοήτως λαμβανέτω πρὸς ἴδιον
πάθος τὸ εἰρημένον ἐν τῇ ἱερᾷ τῶν Πράξεων ἱστορίᾳ
περὶ τοῦ Μάρκου ὅτι ἐγένετο παροξυσμός˙ οὐ γὰρ εἰς\CCSG{450}
πάθος ὀργῆς καὶ θυμοῦ ἐξηνέχθησαν οἱ τοῦ Χριστοῦ
ἀπόστολοι, ἄπαγε˙ πῶς γάρ, οἵτινες τὴν σάρκα ἐσταύρω%
σαν σὺν τοῖς παθήμασι καὶ ταῖς ἐπιθυμίαις, οἱ βοῶντες
φωνῇ μεγάλῃ πρὸς πάντας τοὺς εἰς Χριστὸν πιστεύοντας
καὶ λέγοντες˙ \enquote{Πᾶσα πικρία καὶ θυμὸς καὶ ὀργὴ καὶ\CCSG{455}
κραυγὴ καὶ βλασφημία ἀρθήτω ἀφ’ ὑμῶν σὺν πάσῃ κακίᾳ.}
Ἡμεῖς γὰρ ἐγνώκαμεν ἐκ τῆς θείας Γραφῆς παροξυσμῶν
διαφοράν˙ φησὶ γάρ˙ \enquote{Καὶ κατανοῶμεν ἀλλήλους εἰς
παροξυσμὸν ἀγάπης καὶ καλῶν ἔργων.} Ἐγένετο οὖν
τοῖς ἀποστόλοις παροξυσμός, τουτέστι ζῆλος θεοῦ. Ὁ\CCSG{460}
μὲν γὰρ Παῦλος ἀκρίβειαν ἐπεζήτει, ἀποστολικῇ πρέ%
πουσαν τελειότητι, ὁ δὲ Βαρνάβας ἐτίμα τὸ φιλάνθρωπον˙
λαβὼν οὖν τὸν Μάρκον, ἔπλευσεν εἰς Κύρον, καὶ πᾶσαν
διελθὼν καὶ μαθητεύσας λαὸν ἱκανόν, ἦλθεν εἰς Σαλα%
μίνην˙ κἀκεῖ διέτριβεν θαυματουργῶν καὶ κηρύσσων τὴν\CCSG{465}
βασιλείαν τοῦ θεοῦ, καὶ προσετέθη ὄχλος πολὺς τῷ
κυρίῳ. Διελέγετο δὲ τοῖς Ἰουδαίοις ἐν τῇ συναγωγῇ
κατὰ πᾶν σάββατον, ἐπιδεικνὺς ἐκ τῶν γραφῶν εἶναι
τὸν Χριστὸν Ἰησοῦν. Πάντες δὲ εὐλαβοῦντο αὐτὸν διὰ
τὴν περικεχυμένην τῷ προσώπῳ αὐτοῦ θείαν χάριν˙ ἦν\CCSG{470}
γὰρ τὸ εἶδος αὐτοῦ ἀγγελικὸν καὶ τὸ σχῆμα ἀσκητικόν˙
σύνοφρυς δὲ ὑπῆρχεν, ὀφθαλμοὺς ἔχων χαροποιούς, οὐ
βλοσσυρὸν βλέποντας, ἀλλ’ εὐλαβῶς κάτω νεύοντας˙ στόμα
σεμνὸν καὶ χείλη εὐπρεπῆ, γλυκασμὸν μέλιτος ἀποστά%
ζοντα --- οὐ γὰρ ἐφθέγγετο πώποτε περιττὸν τοῦ δέον\CCSG{475}%
τος ---˙ βάδισμα κατεσταλμένον καὶ ἀκενόδοξον, καὶ ἁπαξ%
απλῶς ὅλος διόλου στήλη ἦν καθαρὰ τοῦ Χριστοῦ ὁ
ἀπόστολος Βαρνάβας, πᾶσαν ἀρετὴν ἀποστίλβουσα.

Ἐπιμένοντος δὲ αὐτοῦ ἐν τῇ Σαλαμινέων πόλει καὶ
διδάσκοντος τὸν λόγον τοῦ θεοῦ, ἐπῆλθον ἐκ τῆς Συρίας\CCSG{480}
Ἰουδαῖοι καὶ ἀντέλεγον τοῖς λεγομένοις ὑπ’ αὐτοῦ, βλασφη%
μοῦντες˙ ἐπήγειραν δὲ ὄχλον κατ’ αὐτοῦ, λέγοντες ὅτι
οὐδὲν ἀληθὲς λέγει, ἀλλὰ καὶ ὃν λέγει Ἰησοῦν, πλάνος
τίς ἦν καὶ ἀντίθεος, τὸν νόμον καὶ τοὺς προφήτας καὶ
τὸ σάββατον ἀθετῶν˙ καὶ ἧσαν ζητοῦντες εὐκαιρίαν ἀνε\CCSG{485}%
λεῖν αὐτόν. Ὁ δὲ ἅγιος ἀπόστολος τοῦ Χριστοῦ Βαρ%
νάβας, συναγαγὼν πάντας τοὺς ἀδελφούς, εἶπεν αὐτοῖς˙
\enquote{Ὑμεῖς ἐπίστασθε πῶς μεθ’ ὑμῶν τὸν ἅπαντα χρόνον
ἐγενόμην, νουθετῶν ἕνα ἕκαστον καὶ παρακαλῶν ἐμμένειν
τῇ χάριτι καὶ τῇ πίστει τοῦ κυρίου ἡμῶν Ἰησοῦ Χριστοῦ,\CCSG{490}
καὶ τὰς ἐντολὰς αὐτοῦ φυλάσσειν καὶ ποιεῖν αὐτὰς καὶ
ἀπέχεσθαι ἀπὸ παντὸς πονηροῦ πράγματος˙ δεῖ γὰρ
πάντας ἡμᾶς παραστῆναι τῷ βήματι τοῦ Χριστοῦ, ἵνα
κομίσηται ἕκαστος τὰ διὰ τοῦ σώματος, πρὸς ἃ ἔπραξεν,
εἴτε ἀγαθά, εἴτε κακά˙ παράγει γὰρ τὸ σχῆμα τοῦ\CCSG{495}
κόσμου τούτου καὶ μέλλει ὁ κύριος ἐλθεῖν ἐξ οὐρανοῦ
κρῖναι ζῶντας καὶ νεκρούς. Μὴ οὖν ἀμελήσητε, γινώ%
σκοντες ὅτι ἐν ᾗ ἂν ὥρᾳ οὐ δοκεῖτε, ὁ κύριος ἡμῶν
ἔρχεται. Κακοπαθήσατε οὖν καὶ τῇ ἐλπίδι στηρίξατε τὰς
καρδίας ὑμῶν, ὅτι ἡ παρουσία τοῦ κυρίου ἤγγικεν.\CCSG{500}
Μνημονεύετε ὅσα ὑμῖν λελάληκα, λέγων ὅτι τοῦ παρόντος
βίου ὀλιγοχρόνιά ἐστιν, εἴτε τὰ χρηστά, εἴτε τὰ λυπηρά,
καὶ πάντα ταχὺ παρελεύσεται, τοῦ δὲ μέλλοντος αἰῶνος
πάντα ὁμοίως αἰώνια καὶ ἀτελεύτητα˙ οὐδὲ γὰρ ἡ βα%
σιλεία τῶν οὐρανῶν ποτὲ παρελεύσεται, οὐδὲ ἡ κρίσις\CCSG{505}
τέλος ἔχει, ἀλλ’ ἀεὶ διαμένει, ἀθάνατα καὶ ἀκατάπαυστα
κολάζουσα τοὺς ἁμαρτωλούς. Σπουδάσατε οὖν ἀμέμπτους
καὶ ἀσπίλους εὑρεθῆναι ὑμᾶς ἐν τῇ ἡμέρᾳ ἐκείνῃ, ἵνα
μὴ ἐμπέσητε εἰς ἐκείνην τὴν γέενναν τὴν ἀθάνατον καὶ
ἀτελεύτητον. Μνημονεύσατε ὅσα ὁ θεὸς ἐποίησε σημεῖα\CCSG{510}
καὶ τέρατα ἐν ὑμῖν δι’ ἐμοῦ, τοῦ δούλου αὐτοῦ, καὶ
προσεύξασθε ὑπὲρ ἐμοῦ˙ ἐγὼ γὰρ ἤδη σπένδομαι καὶ
ὁ καιρὸς τῆς ἀναλύσεώς μου ἐφέστηκε, καθὼς ὁ κύριος
ἡμῶν Ἰησοῦς Χριστὸς ἐδήλωσέ μοι. Τὸν ἀγῶνα τὸν
καλὸν ἠγώνισμαι, τὸν δρόμον τετέλεκα, τὴν πίστιν\CCSG{515}
τετήρηκα, νῦν ἀπόκειταί μοι ὁ τῆς δικαιοσύνης στέφανος,
οὐ μόνον δὲ ἐμοί, ἀλλὰ καὶ πᾶσι τοῖς ἀγωνιζομένοις
διὰ τὸ ὄνομα αὐτοῦ.}

Καὶ ταῦτα εἰπών, σὺν πᾶσιν αὐτοῖς προσηύξατο˙ ἱκανὸς
δὲ κλαυθμὸς ἐγένετο πάντων, τοῦ ἀποστόλου εἰρηκότος\CCSG{520}
ὅτι ὁ καιρὸς τῆς ἀναλύσεώς μου ἐφέστηκεν. Λαβὼν δὲ
ὁ Βαρνάβας ἄρτον καὶ ποτήριον, καὶ ποιήσας πᾶσαν
τὴν ἀκολουθίαν, μετέλαβε μετὰ τῶν ἀδελφῶν τῆς εὐ%
χαριστίας τῶν μυστηρίων. Καὶ μετὰ ταῦτα παραλαβὼν
τὸν Μάρκον καὶ ἀναχωρήσας κατ’ ἰδίαν, εἶπεν αὐτῷ˙ \enquote{Ἐν\CCSG{525}
τῇδε τῇ ἡμέρᾳ δεῖ με τελειωθῆναι εἰς τὰς χεῖρας τῶν
ἀπειθούντων Ἰουδαίων˙ σὺ δὲ ἐξελθὼν ἔξω τῆς πόλεως
κατὰ δυσμάς, εὑρήσεις τὸ σῶμα μου˙ καὶ τοῦτο θάψας,
ἔξελθε ἀπὸ Κύπρου καὶ πορεύου πρὸς τὸν Παῦλον καὶ
ἴσθι μετ’ αὐτοῦ, ἕως οὗ ὁ κύριος οἰκονομήσει τὰ κατὰ\CCSG{530}
σέ˙ μέλλει γὰρ τὸ ὄνομά σου μεγαλυνθῆναι ἐν ὅλῃ
τῇ οἰκουμένῃ.}

Μετὰ ταῦτα εἰσῆλθε Βαρνάβας εἰς τὴν συναγωγὴν καὶ
ἐδίδασκε τοὺς Ἰουδαίους, πείθων αὐτοὺς περὶ τοῦ κυρίου
Ἰησοῦ, ὅτι οὗτός ἐστιν ὁ Χριστὸς ὁ Υἱὸς τοῦ θεοῦ\CCSG{535}
τοῦ ζῶντος. Πλησθέντες δὲ θυμοῦ οἱ ἀπὸ Συρίας Ἰου%
δαῖοι, ἀναστάντες ἐπέβαλον ἐπ’ αὐτὸν τὰς χεῖρας καὶ
ἔθεντο αὐτὸν ἐν οἴκῳ σκοτεινῷ ἐν τῇ συναγωγῇ ἕως
ἑσπέρας βαθείας˙ ἐξαγαγόντες δὲ αὐτὸν καὶ βασανίσαντες
ἱκανῶς, ἤγαγον διὰ νυκτὸς ἔξω τῆς πόλεως καὶ ἐκεῖ\CCSG{540}
αὐτὸν κατέλευσαν οἱ παράνομοι˙ καὶ ἅψαντες πυρὰν
μεγάλην, ἔρριψαν ἐκεῖ τὸν μακάριον πρὸς τὸ μηδὲ
λείψανον αὐτοῦ εὑρεθῆναι. Προνοίᾳ δὲ θεοῦ ἀκέραιον
ἔμεινε τὸ σῶμα τοῦ ἀποστόλου καὶ οὐδὲν αὐτὸ ἔβλαψεν
ἡ πυρά. Μάρκος δὲ κατὰ τὰ διατεταγμένα αὐτῷ ἐξελθὼν\CCSG{545}
ἔξω τῆς πόλεως κατὰ δυσμὰς μετά τινων ἀδελφῶν
κρυφῇ, συνεκόμισαν τὸ λείψανον τοῦ ἁγίου Βαρνάβα καὶ
θάψαντες ἐν σπηλαίῳ ὡς ἀπὸ σταδίων πέντε τῆς πόλεως,
ἀνεχώρησαν, ποιήσαντες κοπετὸν μέγαν ἐπ’ αὐτόν.

Ἐγένετο δὲ ἐν ἐκείνῳ τῷ καιρῷ διωγμὸς μέγας ἐπὶ\CCSG{550}
τὴν ἐκκλησίαν τὴν ἐν Σαλαμίνῃ, καὶ πάντες διεσπάρησαν
ἄλλος ἀλλαχοῦ˙ καὶ λοιπὸν ἄγνωστον γέγονε τὸ μνῆμα
τοῦ ἁγίου ἀποστόλου Βαρνάβα. Μάρκος δὲ ἐκπλεύσας
ἀπὸ τῆς Κυπρίων χώρας, ἦλθε πρὸς Παῦλον εἰς Ἔφεσον
καὶ ἀνήγγειλεν αὐτῷ περὶ τῆς τοῦ Βαρνάβα τελειώσεως˙\CCSG{555}
ὁ δὲ ἀκούσας, ἔκλαυσε κλαυθμῷ μεγάλῳ ἐπ’ αὐτόν, κα%
τέσχε δὲ παρ’ ἑαυτῷ τὸν Μάρκον. Μετὰ δὲ ταῦτα τοῦ
Πέτρου κατὰ ἀποκάλυψιν θεοῦ ἀπαίροντος ἐπὶ τὴν Ῥώμην,
παρέλαβε μεθ’ ἑαυτοῦ τὸν Μάρκον, τρόπον τινὰ τεκνο%
ποιήσας αὐτόν˙ ἐκεῖ συνέταξε τὴν εὐαγγελικὴν ἱστορίαν˙\CCSG{560}
ἥντινα ὁ Πέτρος ἀναγνοὺς καὶ ἀρεσθείς, ἔγνω θεό%
πνευστον αὐτὴν εἶναι˙ καὶ χειροτονήσας τὸν Μάρκον,
ἀπέστειλεν αὐτὸν ὡς ἱκανώτατον εἰς Ἀλεξάνδρειαν τὴν
πρὸς Αἴγυπτον καὶ Λιβύην καὶ Πεντάπολιν˙ ὃς παρα%
γενόμενος, εὐηγγελίσατο αὐτοῖς τὸν Χριστὸν Ἰησοῦν˙\CCSG{565}
πολὺς δὲ ἀριθμὸς ὁ πιστεύσας ἐπέστρεψεν ἐπὶ τὸν
κύριον. Ποιήσας δὲ ἔτη ἐννέα διδάσκων ἐν αὐτοῖς τὸν
λόγον τοῦ θεοῦ, μαρτυρίῳ ἐτελειώθη καὶ ἐτέθη ἐν Ἀλε%
ξανδρείᾳ.
 




\saut

Μετὰ δὲ πολὺν χρόνον τοῦ χριστιανισμοῦ πλατυνθέντος\CCSG{570}
καὶ χριστιανῶν βασιλέων τῆς Ῥωμαίων ἀρχῆς βασιλευ%
σάντων, δυνάμεις οὐ τὰς τυχούσας ἐποίει ὁ θεὸς ἐν
ᾧ τόπῳ ἀπέκειτο τὸ λείψανον τοῦ ἁγίου ἀποστόλου καὶ
γενναίου μάρτυρος Βαρνάβα˙ πολλοὶ γὰρ τῶν ἐχόντων
πνεύματα ἀκάθαρτα, παριόντα διὰ τοῦ τόπου, βοῶντα\CCSG{575}
φωνῇ μεγάλῃ, ἐξήρχοντο, πολλοὶ δὲ παραλελυμένοι καὶ
χωλοὶ καὶ ὑπὸ ποικίλων νόσων καὶ βασάνων συνεχό%
μενοι, ἐρχόμενοι παρεκοιμῶντο ἐν τῷ τόπῳ, οἵτινες
ἐθεραπεύοντο ἅπαντες˙ καὶ ἦν χαρὰ μεγάλη ἐν τῇ Σα%
λαμινέων πόλει˙ καὶ ὅτι μὲν θεία τίς ἦν δύναμις ἡ\CCSG{580}
ἐνεργοῦσα ἐν τῷ τόπῳ, ἐγίνωσκον, τί δέ ἐστι τὸ αἴτιον
τῆς τοσαύτης ἀφθόνου χάριτος, οὐκ ἐγίνωσκον˙ τὸν δὲ
χῶρον ἐκεῖνον τόπον τῆς ὑγείας οἱ ἐγχώριοι ἐπωνόμαζον.

Τοῦ δὲ μακαρίου Μαρκιανοῦ καταπαύσαντος τὴν ἀρχήν,
παρέλαβε τὴν βασιλείαν ὁ τῆς θείας λήξεως Λέων˙ τούτῳ\CCSG{585}
ὑπῆρχε γαμβρός, Ζήνων τίς ὀνόματι, Ἴσαυρος τῷ γένει,
ὅστις καὶ ἐβασίλευσε μετ’ αὐτόν. Κατ’ ἐκεῖνον δὲ τὸν και%
ρὸν εὑρέθη ἐν τῇ εὐαγεστάτῃ μονῇ τῶν Ἀκοιμήτων
διάβολος τίς μονάζων, ὥσπερ ἐν ἀποστόλοις Ἰούδας,
τοὔνομα Πέτρος, τὸ ἐπιτήδευμα κναφεύς˙ οὗτος δὲ τὴν\CCSG{590}
ἐν Καλχηδόνι ἁγίαν σύνοδον ἀποστρεφόμενος, τῶν Εὐ%
τυχιανιστῶν ὑπερεμάχει δογμάτων. Τοῦτον οἱ τῆς ἁγίας
μονῆς ἐκείνης, ὡς λυμεῶνα καὶ φθορέα καὶ τῶν ἀπο%
στολικῶν δογμάτων ἐχθρόν, ἐδίωξαν τοῦ μοναστηρίου˙
αὐτὸς δὲ καταλαβὼν τὴν Κωνσταντινούπολιν, τὸν τῶν\CCSG{595}
κολάκων βίον ἐζήλωσεν, ἐξ οἰκίας εἰς οἰκίαν περιερ%
χόμενος καὶ γαστριζόμενος˙ εὑρὼν δέ τινας τῶν ἐν τέλει
τῆς βδελυρᾶς αὐτοῦ αἱρέσεως ὄντας, ἐκολλήθη αὐτοῖς
καὶ δι’ αὐτῶν γίνεται γνώριμος τῷ γαμβρῷ τοῦ βασιλέως,
πατρικίῳ ὄντι τὸ τηνικαῦτα καὶ κόμητι ἐξκουβιτόρων˙\CCSG{600}
καὶ περιθέμενος εὐλαβείας πρόσωπον, ἦν σὺν αὐτῷ
ἀδιαλείπτως, μὴ τολμῶν δημοσιεῦσαι τὴν ἰδίαν ἀσέβειαν.
Ἀπαίροντι δὲ τῷ Ζήνωνι ἐπὶ τὰ μέρη τῆς ἀνατολῆς,
συνείπετο ὁ Κναφεὺς ἕως Ἀντιοχείας. Εὑρὼν δὲ ἐκεῖ
πολλοὺς τῶν Ἀπολιναριστῶν, κατὰ τοῦ πατριάρχου λοι\CCSG{605}%
πὸν ἐνεανιεύσατο, διεγείρων κατ’ αὐτοῦ τοὺς ἀτάκτους
τοῦ λαοῦ καὶ λοιδορῶν τὴν ἐν Καλχηδόνι σύνοδον καὶ
Νεστοριανὸν καλῶν τὸν πατριάρχην. Ἐν πολλῇ δὲ ἀκα%
ταστασίᾳ καὶ θορύβῳ τῆς Ἀντιοχέων ὑπαρχούσης,
προσῆλθεν ὁ Κναφεὺς τῷ κόμητι, μετὰ δόλου λέγων˙\CCSG{610}
\enquote{Ἐὰν μὴ διαδεχθῇ ὁ ἐπίσκοπος τῆσδε τῆς πόλεως,
ἀμήχανον ἡσυχίαν ἄγειν τὸν δῆμον.} ἅμα καὶ ταξάμενος
αὐτῷ χρυσίου ποσότητα πολλήν, εἴπερ τύχοι τοῦ αἰ%
τουμένου˙ ἐξέφανε γὰρ αὐτῷ καὶ τοῦ ἰδίου σκοποῦ τὰ
κεκρυμμένα. Τότε πείθει ὁ Κναφεὺς τοὺς τὰ ὅμοια\CCSG{615}
αὐτῷ νοσοῦντας καί τινας θυμελικοὺς καὶ ἑτέρους τῶν
δημοτῶν πονηροὺς ἄνδρας, καὶ ἀναφέρει τῷ βασιλεῖ,
πάνδεινα κατὰ τοῦ ἐπισκόπου ψευδόμενος˙ ἀλλ’ οὐδὲν
ὤνησεν ἡ αὐτοῦ δεινότης, τοῦ βασιλέως τῶν ἀποστολικῶν
δογμάτων ὑπερασπίζοντος.\CCSG{620}

Τοῦ δὲ τὰ τῇδε καταλείψαντος καὶ εἰς τὴν ἀγήρω
βασιλείαν μετατεθέντος, τῆς βασιλείας διάδοχος γίνεται
ὁ Ζήνων. Εὐθέως δὲ οἱ προειρημένοι Ἀντιοχείας δεητικὸν
ἀνήγαγον τῷ βασιλεῖ, αἰτούμενοι τὸν Κναφέα ἐπίσκοπον˙
ὃ δὴ καὶ γέγονε, τοῦ χρυσίου πείθοντος πάντας τοὺς\CCSG{625}
τῆς βασιλικῆς αὐλῆς εἰς τὴν ὑπὲρ αὐτοῦ συνηγορίαν.
Εὐθέως οὖν ἅμα τῇ χειροτονίᾳ ἀνεθεμάτισε δημοσίᾳ τὴν
ἁγίαν ἐν Καλχηδόνι σύνοδον˙ θέλων δὲ ἀρέσαι τοῖς
Ἀπολιναρισταῖς, θεοπασχίαν νοσοῦσι, καινοτομίαν κα%
κίστην ἐπενόησε τοῦ εἰπεῖν ἐν τῷ Τρισαγίῳ ἐπὶ τέλει\CCSG{630}
τοῦ ὕμνου˙ \enquote{Ὁ σταυρωθεὶς δι’ ἡμᾶς.} Ταῦτα μαθόντες οἱ
ἅγιοι ἐπίσκοποι καὶ πατέρες ἡμῶν, διηγέρθησαν γενναίως
κατὰ τῆς αὐτοῦ κακοδοξίας˙ καὶ πρῶτον μὲν ἐπειράθησαν
διὰ παραινετικῶν γραμμάτων ἀνακαλέσασθαι αὐτὸν ἐκ
τοῦ βαράθρου τῆς ἀσεβείας, ὡς δὲ εἶδον αὐτὸν ἀντι\CCSG{635}%
λέγοντα καὶ μᾶλλον θρασυνόμενον κατὰ τῆς ὀρθοδόξου
πίστεως, τότε ἀπεφήναντο κατ’ αὐτοῦ καὶ ἀνεθεμάτισαν
αὐτὸν πάντες οἱ κατὰ τὴν οἰκουμένην ἐπίσκοποι.

Ἐνταῦθα δὲ γενόμενος τοῦ λόγου, ἡδέως ἂν ἐροίμην
τοὺς ἐκ τῶν ἡμετέρων ἀβασανίστως παραδεξαμένους τὴν\CCSG{640}
καινοτομίαν ταύτην, ἁπλότητι λογισμοῦ καὶ οὐ κακίᾳ
γνώμης. Τίνος χάριν, ἀδελφοί μου, καταλιπόντες ἐν τούτῳ
τῷ μέρει τὴν τῶν πατέρων ὀρθόδοξον διδασκαλίαν, τὴν
ἐφευρεθεῖσαν ὑπὸ τῶν αἱρετικῶν καινοτομίαν κατεδέ%
ξασθε; Ἐχρῆν συνιέναι ὅτι οὐ σεσοφισμένοις μύθοις\CCSG{645}
ἐξακολουθήσαντες, οἱ ἅγιοι ἡμῶν πατέρες παρέδωκαν
ἡμῖν ἄδειν τοῦτον τὸν ὕμνον, ἀλλ’ ἐξ ἀποκαλύψεως θειο%
τέρας οὐχ ἑνὶ καὶ δευτέρῳ μόνῳ ἀποκαλυφθείσης, οὐδὲ
ἐν παραβύστῳ τινί, ἀλλ’ ἡμέρας μέσης, παντὶ τῷ φιλο%
χρίστῳ λαῷ τῆς βασιλίδος Κωνσταντινουπολιτῶν, ἐπὶ τοῦ\CCSG{650}
ὁσίου τρισμακαρίστου πατρὸς ἡμῶν καὶ ἐπισκόπου Πρό%
κλου, τοῦ ἐν διδασκάλοις δοκιμωτάτου. Πᾶν δὲ τὸ ἐκ
θείας χάριτος θεόθεν ἡμῖν ἀποκαλυπτόμενον ἀνελλιπές
ἐστι πάντως καὶ παντέλειον εἰς σωτηρίαν, μὴ ἐπιδεόμενον
τῆς ἐξ ἀνθρωπίνων συλλογισμῶν προσθήκης ἢ ἀφαι\CCSG{655}%
ρέσεως κατὰ τὸ εἰρημένον τῇ θεοπνεύστῳ Γραφῇ˙ \enquote{Τὸ
ῥῆμα, ὃ ἐγὼ ἐντέλλομαί σοι σήμερον, φύλαξαι σφόδρα
τοῦ ποιῆσαι αὐτό, οὐ προσθήσεις ἐπ’ αὐτῷ, οὐδὲ ἀφελεῖς
ἀπ’ αὐτοῦ.} Τῷ ὄντι γὰρ προσθεῖναί τι τοῖς θείοις χρησμοῖς
ἢ ἀφελεῖν ἐπισφαλὲς καὶ ἐπικίνδυνον˙ οὐ γὰρ ἐσμὲν\CCSG{660}
διορθωταὶ τοῦ θεοῦ, ἀλλ’ ὑποτακτῖται. Καὶ πολλά ἐστι
περὶ τούτου εἰπεῖν πρὸς τοὺς ἡμετέρους καὶ δεῖξαι ὅτι
ἰδιωτ\textins{ε}ία ἐσχάτη ἐστὶ τὸ παραδέξασθαι τὴν καινοτομίαν
τοῦ αἱρετικοῦ, κἂν μὴ φρονῶσι τὰ τοῦ αἱρετικοῦ.

Οὕτως τοιγαροῦν ὁ Κναφεὺς Πέτρος ἀναθεματισθείς,\CCSG{665}
ὡς εἴρηται, ὑπὸ πάντων τῶν ἐπισκόπων καὶ τοῦ Ζήνωνος
φυγόντος ἐκ τῆς βασιλείας διὰ τὴν ἐπανάστασιν Βασι%
λίσκου, φυγὰς ᾤχετο εἰς ἀγνώστους τόπους. Ὡς δὲ
ἐπανῆλθεν ὁ Ζήνων εἰς τὴν ἑαυτοῦ βασιλείαν, Πέτρον
μὲν τὸν Κναφέα ἀναζητήσας, ἀπεκατέστησε τῇ Ἀντιοχέων\CCSG{670}
ἐπίσκοπον, τῶν ἐπ’ αὐτῷ ἀναθεματισμῶν μηδαμῶς λυ%
θέντων, τὸν δὲ μακάριον Καλανδίωνα τὸν πατριάρχην
εἰς Ὤασιν ἐξωστράκισε. Τοῦ Κναφέως βίᾳ ἀπολαβόντος
τὸν θρόνον, ἅτε ἀπεγνωσμένος λοιπὸν καὶ μηδεμίαν
ἐλπίδα ἔχων, ἀδεῶς μᾶλλον δὲ ἀθέως ἐχρήσατο τῇ\CCSG{675}
τυραννίδι, φονεύων καὶ δημεύων καὶ φυγαδεύων πάντας
τοὺς μὴ βουλομένους κοινωνεῖν αὐτοῦ τῇ ἀσεβείᾳ. Ἀλλὰ
ταῦτα μὲν παρήσω, πολλὰ ὄντα καὶ ἰδίας συγγραφῆς
δεόμενα, βαδιοῦμαι δὲ πρὸς τὸ κατεπεῖγον, δεικνὺς πᾶσι
τοῦ ἁγίου ἀποστόλου Βαρνάβα τὴν ἀμέτρητον χάριν καὶ\CCSG{680}
ὅσην ἔχει περὶ τὴν ἰδίαν πατρίδα φροντίδα τε καὶ
κηδεμονίαν. Μὴ ἀρκεσθεὶς τοιγαροῦν τοῖς ἀναριθμήτοις
κακοῖς, οἷς ἐποίησε κατὰ τὴν ἀνατολὴν ὁ Κναφεύς, ἀλλὰ
καὶ ἐπὶ τὰς οὐδαμόθεν αὐτῷ ὑποκειμένας ἐπαρχίας ἐπέ%
βαλε τὰς χεῖρας˙ χρυσίῳ γὰρ ἅπαξ ἐξαγοράσας τοῦ\CCSG{685}
βασιλέως καὶ τῶν περὶ αὐτὸν τὴν εὐχέρειαν, καὶ αὐτοῦ
τοῦ θεοῦ τὸν νόμον περιεφρόνησε --- καὶ γοῦν τοὺς
φιλοθέους καὶ φιλοπίστους καὶ ὀρθοδόξους Κυπρίους παν%
τοίως κακῶσαι διενοήθη, ἐπειδὴ προγονικὴν εὐσέβειαν
φυλάττοντες, κοινωνῆσαι τῇ ἀσεβείᾳ αὐτοῦ οὐκ ἐβούλον\CCSG{690}%
το ---, καὶ τὸν ἐξαρχῆς καὶ ἄνωθεν ἁγιώτατον καὶ ἐλεύθερον
καὶ ἀποστολικὸν θρόνον τῆς Κύπρου ἁρπάσαι καὶ ὑφ’ ἑ%
αυτὸν ποιῆσαι ἐπεχείρησεν˙ καὶ ἀναφορὰν ποιεῖται πρὸς
τὸν κρατοῦντα, ψευδηγορίας μεστήν, ὅτι φησὶν ὁ λόγος
τοῦ θεοῦ ἀπὸ Ἀντιοχείας ἐν Κύπρῳ ἐξήχηται καὶ δεῖ\CCSG{695}
τὴν ἐκκλησίαν Κύπρου ὑπὸ τὸν Ἀντιοχείας θρόνον εἶναι,
ἐπειδὴ ἀποστολικὸς καὶ πατριαρχικὸς θρόνος τυγχάνει˙
ἀλλ’ οὐκ ἔλαθεν ὁ ἀποστάτης αἱρετικὸς ὤν, τὰ ἐν Ἐφέσῳ
ἐπὶ τῆς κατὰ Κύριλλον ἁγίας συνόδου ὁρισθέντα πα%
ραχαράσσων. Ἀλλ’ ὁ κήρυξ τῆς εὐσεβείας, ὁ ἁγιώτατος\CCSG{700}
ἐν ἀποστόλοις Βαρνάβας, ἤλεγξεν αὐτοῦ τὸ ἀνόητον, ἐν
καιρῷ τῆς πατρίδος γενόμενος ἀντιλήπτωρ. Τοῦ γὰρ
ἐπισκόπου τῆς Σαλαμινέων κελευσθέντος καταλαβεῖν τὴν
βασιλεύουσαν καὶ ἐπὶ τοῦ οἰκουμενικοῦ πατριάρχου κριθῆ%
ναι πρὸς τοὺς Ἀντιοχείας, ἀπεναρκώθη τῷ φόβῳ, ὑφορώ\CCSG{705}%
μενος τὰς τοῦ Κναφέως συσκευάς˙ Ἀνθέμιος δὲ ἦν ὁ
θαυμάσιος ἀνήρ, ὡς τὰ μάλιστα ὀρθοδοξότατος καὶ βίῳ
ἀκηλιδώτῳ λελαμπρυσμένος, ὀλιγοστὸς δὲ πρὸς διάλεξιν
τῶν ἀντιδιατιθεμένων. Ἀμηχανοῦντι τοίνυν καὶ ἀδημο%
νοῦντι περὶ τὴν ἀποδημίαν, νύκτωρ ἐφίσταταί τις\CCSG{710}
αὐτῷ, καθεύδοντι ἐν ἰδιάζοντι τόπῳ, θεείκελον ἔχων τὸ
πρόσωπον καὶ φωτὸς μαρμαρυγὰς ἀπαστράπτων, στολὴν
ἐκ φωτὸς ἱεροπρεπῶς ἀναβεβλημένος, καὶ φησὶ πρὸς
αὐτόν˙ \enquote{Τί οὕτω λελύπησαι, ἐπίσκοπε, καὶ τίς ἡ περὶ
σὲ αὕτη ῥαθυμία καὶ ἵνα τί συνέπεσέ σου τὸ πρόσωπον;\CCSG{715}
Οὐδὲν πείσῃ δεινὸν ὑπὸ τῶν ἀντιδίκων.} Καὶ ταῦτα
εἰπών, ἀπέστη ἀπ’ αὐτοῦ. Ἔξυπνος δὲ γενόμενος ὁ ἐπί%
σκοπος μετὰ φόβου, ἔπεσεν ἐπὶ πρόσωπον μετὰ πολλῶν
δακρύων καὶ ἐδέετο τοῦ θεοῦ, λέγων˙ \enquote{Κύριε Ἰησοῦ
Χριστέ, Υἱὲ τοῦ θεοῦ τοῦ ζῶντος, εἰ φείδῃ σου τῆσδε\CCSG{720}
τῆς ἐκκλησίας --- οἶδα δὲ ὅτι φείδῃ --- καὶ εἰ παρὰ σοῦ
γέγονεν ἡ ὀπτασία αὕτη, καταξίωσον δευτερῶσαί μοι
καὶ τρισσῶσαι τὴν ὅρασιν, ἵνα πληροφορηθῶ ὅτι σὺ
μετ’ ἐμοῦ εἶ.} Τότε σπουδαιοτέρως ἐκέχρητο τῇ προ%
σευχῇ, μηδενὶ συντυγχάνων. Τῇ δὲ ἐπιούσῃ νυκτὶ πά\CCSG{725}%
λιν ἐφίσταται αὐτῷ ὁ αὐτὸς ἀνὴρ ἐν τῇ αὐτῇ ἰδέᾳ
καὶ ἐν τῷ αὐτῷ σχήματι, λέγων πρὸς αὐτόν˙ \enquote{Εἶπον
σοι ἤδη ὅτι οὐδὲν τῶν ἀπειλουμένων σοι συμβήσεται˙
βαῖνε τοίνυν προθύμως ἐπὶ τὴν βασιλίδα, μηδὲν ὑφορώ%
μενος.} Καὶ ταῦτα εἰπών, ἀνεχώρησεν. Ἀναστὰς δὲ ὁ\CCSG{730}
ἐπίσκοπος, ηὐχαρίστησε πάλιν τῷ κυρίῳ, μηδὲν λαλήσας
μηδενί. Ἐξεδέχετο δὲ καὶ τὴν τρίτην ὅρασιν, καὶ τῇ
ἐπερχομένῃ νυκτὶ ἐπέστη αὐτῷ ὁ αὐτὸς ἀνήρ, ἐμβρι%
θέστερον λέγων πρὸς αὐτόν˙ \enquote{Ἕως τίνος οὐ πιστεύεις
τοῖς λόγοις μου, οἵτινες πληρωθήσονται ἐν ταύταις ταῖς\CCSG{735}
ἡμέραις; Βάδιζε προθύμως ἐπὶ τὴν βασιλίδα καὶ μετὰ
δόξης ὑποστρέφεις εἰς τὸν θρόνον σου˙ οὐδὲν πείσῃ
δεινὸν ὑπὸ τῶν ἐναντίων, θεοῦ ὑπερασπίζοντός σου
δι’ ἐμοῦ, τοῦ δούλου αὐτοῦ.} Ἀπεκρίθη ὁ ἐπίσκοπος, τοῦ
στόματος αὐτοῦ ἀνεῳχθέντος˙ \enquote{Σὺ γὰρ τίς εἶ, κύριέ μου,\CCSG{740}
ὁ ταῦτα λαλῶν παρ’ ἐμοί;} Ὁ δέ˙ \enquote{Ἐγὼ εἰμί} φησι
\enquote{Βαρνάβας ὁ μαθητὴς τοῦ κυρίου ἡμῶν Ἰησοῦ Χριστοῦ,
ὁ ἀφορισθεὶς ὑπὸ τοῦ ἁγίου Πνεύματος εἰς ἀποστολὴν
ἐθνῶν μετὰ Παύλου τοῦ ἀποστόλου, τοῦ σκεύους τῆς
ἐκλογῆς˙ καὶ τοῦτο σοι ἐστὶ τὸ σημεῖον˙ πορεύου} φησὶν\CCSG{745}
\enquote{ἔξω τῆς πόλεως κατὰ δυσμὰς ἀπὸ σταδίων πέντε εἰς
τὸν τόπον τὸν λεγόμενον τῆς ὑγείας --- δι’ ἐμοῦ γὰρ
ἐνεργεῖ ὁ θεὸς τὰ θαύματα ἐν τῷ τόπῳ ἐκείνῳ ---, καὶ
ὄρυξον ἐπὶ τὴν κερατέαν καὶ εὑρήσεις σπήλαιον καὶ
λάρνακα ἐν αὐτῷ˙ ἐκεῖ μου τὸ πᾶν σῶμα ἀπόκειται\CCSG{750}
καὶ εὐαγγέλιον ἰδιόχειρον, ὃ ἐξέλαβον ἀπὸ Ματθαίου
τοῦ ἁγίου ἀποστόλου καὶ εὐαγγελιστοῦ. Καὶ ἐπειδὴ οἱ
ἀντίδικοί σου δικαιολογοῦσιν, ἄνω καὶ κάτω λέγοντες
ὅτι ἀποστολικός ἐστιν ὁ θρόνος Ἀντιοχείας, ἀντιδικαιο%
λόγησαι καὶ σὺ πρὸς αὐτοὺς ὅτι καὶ ὁ ἐμὸς θρόνος\CCSG{755}
ἀποστολικός ἐστι καὶ ἀπόστολον ἔχω ἐν τῇ πατρίδι
μου.}

Ταῦτα εἰπών, ἀπῆλθε˙ καὶ ἀναστὰς ὁ ἐπίσκοπος καὶ
προσκυνήσας τῷ κυρίῳ, συνήγαγεν ἅπαντα τὸν ἱερὸν
κλῆρον καὶ τὸν φιλόχριστον λαόν, καὶ ἐξῆλθε μετὰ\CCSG{760}
σταυροφανείας εἰς τὸν τόπον τὸν δειχθέντα αὐτῷ μετὰ
πολλῆς παρασκευῆς˙ καὶ ποιήσας εὐχήν, ἐκέλευσεν ὀρυγῆ%
ναι τὸν τόπον˙ καὶ μικρὸν ὀρύξαντες, εὗρον σπήλαιον,
πεφραγμένον ἐν λίθοις, καὶ τούτους ἀποκυλίσαντες, εὗρον
τὴν σορόν, καὶ ταύτην ἀποσκεπάσαντες, εὗρον τὸ τίμιον\CCSG{765}
λείψανον τοῦ ἁγίου καὶ ἐνδόξου ἀποστόλου Βαρνάβα,
πνέον εὐωδίας χάριτος πνευματικῆς˙ εὗρον δὲ καὶ τὸ
εὐαγγέλιον, ἐπὶ τὸ στῆθος αὐτοῦ κείμενον˙ καὶ σφρα%
γίσαντες τὴν σορὸν μολίβδῳ καὶ ποιήσαντες εὐχὴν καὶ
προσκυνήσαντες, ἀνεχώρησαν, τοῦ ἐπισκόπου προσκα\CCSG{770}%
ταστήσαντος ἐν τῷ τόπῳ ἄνδρας εὐλαβεῖς ἑσπεριναῖς
καὶ ἑωθιναῖς ὑμνῳδίαις τὸν θεὸν δοξολογεῖν.

Αὐτὸς δὲ παραλαβὼν τῶν ἐπισκόπων τοὺς ἐπισήμους,
ὥρμησεν ἐπὶ τὴν βασιλίδα καὶ κατήχθη ἐν τῷ ἐπι%
σκοπείῳ˙ ἐμηνύθη δὲ τῷ βασιλεῖ ἡ παρουσία αὐτοῦ καὶ\CCSG{775}
ἐπέτρεψε τῷ πατριάρχῃ διακοῦσαι μετὰ τῆς ἐνδημούσης
συνόδου τὰ μεταξὺ τῶν μερῶν καὶ τὸ παριστάμενον
διαλαλῆσαι. Τοῦ δὲ συνεδρίου συγκροτηθέντος καὶ τῆς
ὑποθέσεως κινηθείσης, οἱ ἐξεναντίας χρώμενοι, ὡς ᾤ%
οντο, τοῖς ἑαυτῶν δικαίοις, ἔλεγον ὅτι πατριαρχικός ἐστιν\CCSG{780}
ὁ τῆς Ἀντιοχέων θρόνος καὶ ἀποστολικός, καὶ δεῖ τὰς
ἄλλας ἐπαρχίας εἶναι ὑπ’ αὐτόν. Ἐδόκουν δὲ τοῖς παροῦσιν
εὔλογα λέγειν. Ὁ δὲ μακάριος Ἀνθέμιος, μικρὸν ἐπισχών,
ἀνθυπήνεγκε, λέγων˙ \enquote{Ἀλλὰ καὶ ὁ ἐμός, ὦ βέλτιστοι,
θρόνος ἀποστολικός ἐστιν ἄνωθεν καὶ ἐξαρχῆς, ἐλευθερίᾳ\CCSG{785}
τετιμημένος, καὶ ἀπόστολον ὁλόσωμον ἔχω ἐν τῇ πατρίδι
μου, τὸν ἐν ἁγίαις εὐφημίαις Βαρνάβαν τὸν τρισμακά%
ριον.} Τούτου λαληθέντος, οὐκέτι ἦν τόπος ἀντιλογίας,
τῶν ἐπισκόπων κυρωσάντων σιωπῇ τὸν λόγον˙ καὶ οἱ
ἐξεναντίας δὲ εἱστήκεισαν μετ’ αἰσχύνης ἐνεοί, τὸ εὔλογον\CCSG{790}
τῆς ἀποκρίσεως ἐκπληττόμενοι.

Ταῦτα μαθών, ὁ βασιλεύς, μετὰ σπουδῆς μεταπεμψά%
μενος τὸν ἐπίσκοπον Κύπρου, ἐπυνθάνετο περὶ τῆς ἀπο%
καλύψεως τοῦ ἁγίου ἀποστόλου Βαρνάβα˙ ὁ δὲ μηδὲν
τἀληθοῦς ἀποκρύψας, πάντα καθεξῆς διηγήσατο τῷ βα\CCSG{795}%
σιλεῖ˙ ὁ δὲ ἀκούσας, ἐχάρη λίαν, θαυμάζων ἐπὶ τῇ τοῦ
θεοῦ πλουσίᾳ χάριτι, ὅτι ἐν τοῖς χρόνοις τῆς βασιλείας
αὐτοῦ τηλικοῦτον θαῦμα ἐποίησεν. Τότε ἀπεσόβησεν ἀπ’ αὐ%
τοῦ τὸν Ἀντιοχείας ἐπίσκοπον, κελεύσας τοῦ λοιποῦ
τὸ παράπαν μὴ παρενοχλεῖσθαι τὸν ἐπίσκοπον Κύπρου\CCSG{800}
ἕνεκεν τῆς ὑποθέσεως ταύτης. Τότε παρεκάλει τὸν μα%
κάριον Ἀνθέμιον τὸν ἐπίσκοπον, λέγων˙ \enquote{Ἐπειδήπερ ὁ
κύριος ἡμῶν Ἰησοῦς ὁ Χριστὸς ἐν τοῖς χρόνοις τῆς
ἁγίας σου ἱεραρχίας εὐδόκησεν ἀποκαλύψαι τὸν ἅγιον
αὐτοῦ ἀπόστολον Βαρνάβαν, κέλευσον ἐνταῦθά μοι τὸ\CCSG{805}
τάχος ἀνακομισθῆναι τὸ εὑρεθὲν εὐαγγέλιον, χαριζόμενός
μοι ἐν τούτῳ, τῷ τέκνῳ σου, χάριν ὅτι μάλιστα πλείστην˙
ἕξεις γὰρ κἀμὲ ἀπὸ τοῦ νῦν εἰς πάντα ὑπακούοντα
τῇ σῇ πατρωσύνῃ.}

Ἐπινεύσας δὲ ὁ ἐπίσκοπος, ἀπέστειλεν ἕνα τῶν σὺν\CCSG{810}
αὐτῷ ἐπισκόπων μετὰ πιστοτάτου ἀνθρώπου τοῦ βασι%
λέως˙ καὶ λαβόντες τὸ εὐαγγέλιον, ἀνήγαγον ἐν Κων%
σταντινουπόλει, ἔχον ἐκ θυΐνων ξύλων τὰ πτύχια. Λαβὼν
δὲ ὁ βασιλεὺς καὶ καταφιλήσας καὶ χρυσίῳ κατακοσμή%
σας, ἀπέθετο ἐν τῷ παλατίῳ καὶ μέχρι τῆς σήμερον\CCSG{815}
φυλάττεται˙ ἐν γὰρ τῇ μεγάλῃ πέμπτῃ τοῦ Πάσχα καθ’ ἕ%
καστον ἐνιαυτὸν ἐν αὐτῷ ἀναγινώσκουσι τὸ εὐαγγέλιον
ἐν τῷ εὐκτηρίῳ τοῦ παλατίου. Τὸν δὲ ἐπίσκοπον με%
γάλως ὁ βασιλεὺς τιμήσας, ἀπέλυσεν αὐτὸν ἐπὶ τὴν
Κύπρον μετὰ χρημάτων πολλῶν καὶ κελεύσεως, ἐντει\CCSG{820}%
λάμενος αὐτῷ ἐγεῖραι ναὸν τῷ ἁγίῳ ἀποστόλῳ Βαρνάβᾳ
ἐν τῷ τόπῳ ἔνθα εὑρέθη τὸ τίμιον αὐτοῦ λείψανον˙
πολλοὶ δὲ καὶ τῶν μεγιστάνων δεδώκασιν αὐτῷ χρήματα
εἰς τὴν οἰκοδομὴν τοῦ ναοῦ. Ὁ δὲ καταλαβὼν τὴν
Κυπρίων καὶ συναγαγὼν πλῆθος τεχνιτῶν καὶ ἐργατῶν,\CCSG{825}
οὐ παρέργως εἴχετο τῆς οἰκοδομῆς, ἀλλ’ ἤγειρε ναὸν
τῷ ἀποστόλῳ παμμεγεθέστατον, λαμπρὸν τοῖς μηχανήμασι,
λαμπρότερον τῇ ποικιλίᾳ τῆς διακοσμήσεως, ἐμβόλοις
ἔξωθεν κύκλῳ περιεσφιγμένον. Κατὰ δὲ τὸ κλίτος τοῦ
ναοῦ τὸ πρὸς λίβα, ἐποίησεν αὐλὴν μεγάλην, τέσσαρας\CCSG{830}
στοὰς ἔχουσαν˙ καὶ οἰκίσκους ἔνθεν καὶ ἔνθεν τῆς αὐλῆς
οἰκοδομήσας, καταμένειν ἐν αὐτοῖς προσέταξε τοὺς τὰς
θείας ἐν τῷ ναῷ λειτουργίας ἐπιτελοῦντας μοναχούς˙
τὸν δὲ ὑδραγωγὸν ἐνέγκας ἀπὸ μακρόθεν, ἀναβλύσαι
πλουσίως πεποίηκε, ἐν μέσῳ τῆς αὐλῆς ὑδροδοχεῖον\CCSG{835}
κάλλιστον οἰκοδομήσας, ὅπως οἱ παροικοῦντες ἐν τῷ
τόπῳ καὶ οἱ ξένοι ἔχωσιν ἀφθόνως τὴν τοῦ νάματος
ἀπόλαυσιν. Πολλὰς δὲ καὶ ἑτέρας ξενίας ᾠκοδόμησεν ἐν
τῷ τόπῳ εἰς ἀνάπαυσιν τῶν ἐπιδημούντων ξένων˙ καὶ
ἦν ἰδεῖν ἐκεῖνον τὸν χῶρον, τῇ ὡραιότητι μιμούμενον\CCSG{840}
μικρὰν τινὰ πόλιν καὶ τερπνὴν λίαν. Τὴν δὲ ἁγία τοῦ
ἀποστόλου θήκην ἀπέθετο ἐκ δεξιῶν τοῦ θυσιαστηρίου,
ἀργυρίῳ ἱκανῷ καὶ μαρμάροις κατακοσμήσας τὸν τόπον˙
τὴν δὲ ἡμέραν τῆς ἐνδόξου μνήμης τοῦ τρισμακαρίου
ἀποστόλου καὶ γενναίου μάρτυρος Βαρνάβα ἐδικαίωσαν\CCSG{845}
γίνεσθαι καθ’ ἕκαστον ἐνιαυτόν, κατὰ μὲν Ῥωμαίους τῇ
πρὸ τριῶν εἰδῶν Ἰουνίων, κατὰ δὲ Κυπρίους Κων%
σταντιεῖς μηνὶ Μεσωρὶ τοῦ καὶ δεκάτου ἑνδεκάτῃ, κατὰ
δὲ Ἀσιανοὺς ἤτοι κατὰ Παφίους μηνὶ Πληθυπάτῳ τοῦ
καὶ ἐννάτου ἐννεακαιδεκάτῃ, συναγόμενοι καὶ τὰς πνευ\CCSG{850}%
ματικὰς ἐπιτελοῦντες λειτουργίας, εἰς δόξαν τοῦ Πατρὸς
καὶ τοῦ Υἱοῦ καὶ τοῦ ἁγίου Πνεύματος, ὅτι αὐτῷ ἡ
δόξα εἰς τοὺς αἰῶνας τῶν αἰώνων ἀμήν.


\saut

Ἡ μὲν οὖν ἀποκάλυψις τῶν λειψάνων τοῦ ἁγίου
ἀποστόλου Βαρνάβα τοῦτον ἔχει τὸν τρόπον˙ τὰ δὲ ἐκ\CCSG{855}
τῆς ἁγίας αὐτοῦ θήκης ἀναβρύοντα καθ’ ἑκάστην θαύματα,
εἴ τις ἐθελήσειε γράψαι, οὐδὲ ὁ πᾶς αὐτῷ, ὡς οἶμαι,
ἐπαρκέσειε χάρτης. Ἡμεῖς δέ, ἐπὶ τὴν τῶν προλαβόντων
μνήμην ἀναδραμόντες, τοῦτον τὸν πένταθλον τῆς εὐσε%
βείας ἀγωνιστὴν καὶ στεφανίτην ἀνυμνήσωμεν, μακα\CCSG{860}%
ρίσωμεν, εὐφημήσωμεν˙ οὗτος γὰρ ὡσεὶ ἐλαία κατάκαρπος
ἐν τῷ οἴκῳ τοῦ θεοῦ πεφυτευμένος, καρποὺς εὐωδίας
καθ’ ἑκάστη τῷ θεῷ προσφέρει˙ οὗτος βασιλέων δόξα,
ἱερέων εὐφροσύνη, λαῶν ἀγαλλίαμα˙ οὗτος θλιβομένων
παράκλησις, καταπονουμένων ἀντίληψις, ἀπελπισμένων ἐλ\CCSG{865}%
πίς, ἀπεγνωσμένων βοηθός, ξένων παρηγορία, νοσούντων
ἰατρεία, ὑγιαινόντων φύλαξ, πηγὴ χαρισμάτων πνευμα%
τικῶν, ἐκκλησίας τεῖχος, ὀρθοδόξων στήριγμα, πιστῶν
ὀχύρωμα, καύχημα πάσης τῆς οἰκουμένης.

Ἀλλὰ σοῦ μὲν ὁ ἔπαινος, ὦ πανόλβιε καὶ τρισμακάριε\CCSG{870}
ἀπόστολε Χριστοῦ τοῦ θεοῦ ἡμῶν Βαρνάβα, πάντα νοῦν
καὶ λόγον ἀνθρώπων ὑπερβάλλει˙ τῆς δὲ ἡμετέρας
πτωχείας τὴν εὐγνωμοσύνην ὅμως ἀποδεξάμενος, πρέ%
σβευε ὑπὲρ ἡμῶν πρὸς ὃν ἠγάπησας, Χριστὸν τὸν μο%
νογενῆ Υἱὸν καὶ λόγον τοῦ θεοῦ, ὅπως ῥύσηται ἡμᾶς\CCSG{875}
ἐκ τοῦ ἐνεστῶτος αἰῶνος πονηροῦ καὶ δώσῃ ἡμῖν εὑρεῖν
ἄφεσιν ἁμαρτιῶν καὶ ἔλεος καὶ οἰκτιρμοὺς ἐν τῇ ἡμέρᾳ
ἐκείνῃ τῇ φοβερᾷ, ὅταν ἔλθῃ κρῖναι ζῶντας καὶ νεκρούς.
Τὸν δὲ σὸν διάδοχον τὸν διέποντα τὸν ἁγιώτατόν σου
θρόνον, τὸν σύμφρονα τῆς σῆς ἀκλινοῦς πίστεως, τὸν\CCSG{880}
γνήσιον ὑποφήτην τῆς σῆς ὀρθοδοξίας, τὸν μιμητὴν τῆς
σῆς ἀρετῆς, τὸν ποιμένα καὶ πατέρα ἡμῶν καὶ ἱεράρχην,
τὸν παρόντα νῦν καὶ φαιδρύνοντα τὴν ἁγίαν σου μνήμην,
ἱκέτευε ἐκτενῶς τὸν θεὸν φυλαχθῆναι τῷ ἁγιωτάτῳ σου
θρόνῳ ἐν πολλαῖς ἐτῶν ἀνακυκλώσεσιν, ὑγιαίνοντα καὶ\CCSG{885}
ποιμαίνοντα ἐν εἰρήνῃ τὸν λαὸν αὐτοῦ, ἐν πάσῃ ὁσιότητι
καὶ δικαιοσύνῃ ὀρθοτομοῦντα τὸν λόγον τῆς ἀληθείας.
Τῆς δὲ πατρίδος σου πάσης φρόντισον, ὡς πάντοτε, καὶ
νῦν ταῖς σαῖς ἁγίαις εὐχαῖς, φυλάττων αὐτὴν ἀπὸ παντὸς
κακοῦ καὶ ἀπὸ σκανδάλων τῶν ἐργαζομένων τὴν ἀνομίαν,\CCSG{890}
ἵνα εἰρηνικῶς, σωφρόνως τε καὶ εὐσεβῶς ζήσωμεν ἐν
τῷ νῦν αἰῶνι, προσδεχόμενοι τὸ ἔλεος τοῦ κυρίου ἡμῶν
Ἰησοῦ Χριστοῦ εἰς ζωὴν αἰώνιον, ἧς γένοιτο πάντας
ἡμᾶς ἐπιτυχεῖν, χάριτι καὶ οἰκτιρμοῖς καὶ φιλανθρωπίᾳ
τοῦ κυρίου ἡμῶν Ἰησοῦ Χριστοῦ, μεθ’ οὗ τῷ Πατρὶ ἡ\CCSG{895}
δόξα σὺν τῷ ἁγίῳ Πνεύματι, νῦν καὶ ἀεὶ καὶ εἰς τοὺς
αἰῶνας τῶν αἰώνων ἀμήν.


