
\titulus{Voyages et martyre des saints apôtres \np{Barthélémy} et \np{Barnabé}}

Et la traversée des vies des autres saints et de leurs luttes, est utile à la vie en même temps qu'elle est aussi juste ;
celle des voyages des apôtres aussi, assurément la transmission de leur combat par l'écriture aussi,  
elle est plus importante en vue d'une surabondance de joie et en outre elle est estimée.
C'est assurément pourquoi il est nécessaire de parcourir en un abrégé ce que \ctrad{j'ai observé}{\altertrad{on a observé}.} de ceux qui aujourd'hui sont mis en avant, je parle de \np{Barthélémy} et de \np{Barnabé}, 
afin que les passionnés des belles actions aient une connaissance claire d'en quel lieu chacun à proclamé la parole et par quelle sorte de fin il a quitté la vie : ainsi, il faut commencer le récit .

\np{Barthélémy} le tout à fait honoré, l'enseignant de la piété et  très scrupuleux gardien des commandements du Christ, 
le co-désigné des douze apôtres\bibleall{Mt|10:3+Mc|3:18+Lc|6:14+Ac|1:13} et guide inégaré  des égarés,
le héraut de l'\nl{Arabie Heureuse} \ctrad{est connu}{\altertrad{fait connaître}} et est un enseignant.
Car celle-ci lui a été affecté au sort par la grâce de l'Esprit.
Ayant atteint celle-ci et y ayant \ctrad{déposé}{nuance durative.} la semence de la foi,
il ne faut pas dire qu'il a été soumis à de telle méchancetés auprès d'eux :
tantôt persécuté et tantôt frappé, 
une autre fois de nouveau  enfermé,
accablé de fatigue par la faim et la soif, 
continuant de souffrir par le froid et par la chaleur,
et à la fin de tout,
acceptant pour ainsi dire la maltraitance,
tous les outrages, 
toute forme de pénible supplice ;
de même qu'il a illuminé au final beaucoup par le baptême divin et qu'il a attiré beaucoup vers la connaissance de la vérité, %%% on est encore sur des participes, mais actif. C'est difficile en Fr à rendre
il leur a magnifiquement transmis l'Évangile selon \np{Matthieu},
l'écrivant dans la langue du pays, 
et  par surcroit il a aussi établi des églises pour eux
et il a institué des prêtres pour celles-ci.

Donc comme il obtenait  pour lui ces choses honnêtement et ce qui s'y rapportait quant au but, 
partant de ces lieux, il s'éloigne vers la ville d'\nl[ville d'Arménie][]{Albanie} de la  
\nl[][Arménie]{Grande Arménie}, y annonçant également l'Évangile du Christ,
conduisant beaucoup de monde à la vérité,
continuant d'abord d'y souffrir par de nombreuses et incalculables tribulations.
De là il va à \nl{Arbanopolis} et enseigne le Christ Dieu et démiurge de la plénitude des choses :
souffrant en celle-ci les plus terribles tourments,
à la fin aussi il est livré à la croix et ainsi il magnifie sa vie.

Mais les semences du mal, les serviteurs des démons, par surcroît ne s'arrêtèrent pas là,
mais les athées l'ayant placé dans un cercueil plombé, le livrèrent à la mer.
Dieu ne négligeant pas son propre serviteur,
la préserva jusqu'à sa \nl{Sicile} : %%% mais à quoi fait allusion le αὐτῆς ?
et en effet après que  le cercueil eut échoué dans l'île de \nl{Lipari},
les chrétiens du lieu, discernant aussi qui il était,
une révélation divine leur ayant fait savoir qu'il s'agissait du cadavre de \np{Barthélémy},
le déposèrent avec zèle et gravité.
Et jusqu'à là, voici les informations à son sujet.

Assurément le noble \np{Barnabé}, lui qui comme un soleil brille radieux au chœur des soixante-dix,
%% beaucoup d'ajout dans le manuscrit Vat 1991, reprenant plus dans les détails Alexandre de Chypre
\alexandre{ornement}{le vénérable ornement des \npe{Chypriotes}}, 
\alexandre{Levi}{de la tribu de Lévi},
\alexandre{provenait}{celui-ci provenait}
clairement
de \alexandre{Samuel}{Samuel le prophète en ce qui concerne la famille}. 
Et en effet \alexandre{ancetre}{les ancêtres de celui-ci, retenus à \nl{Chypre}\bibleall{Ac|4:36} à cause de circonstances de guerres, y habitaient avec distinction}, %%% on remarque un changement dans la reprise : le 
ils étaient très illustres et rencontraient des facilités.
\alexandrefr{C'est aussi pourquoi}, comme le dit \np[d'Alexandrie][]{Clément} l'auteur des Stromates,
ils avaient laissé en héritage \alexandre{champ}{à \nl{Jérusalem}} une certain bien,
dont ils distribuèrent la juste part à \np{Barnabé}, % il y aurait ici des // textuels, mais j'avoue avoir du mal à les cerner
qui \alexandre{Chypre}{était né Chypriote}\indexnpe{Chypriotes} et était considéré comme un enfant du pays.
Donc \alexandrefr{ses parents}  l'envoyèrent lorsqu'il fut adolescent à \alexandrefr{\nl{Jérusalem}},
l'exhortant à s'instruire \alexandrefr{des prophètes} et de \alexandrefr{de la loi},
dont il s'instruisait \alexandrefr{aux pieds de \np{Gamaliel}} --- Le maître de la loi --- \alexandre{education}{ayant pour condisciple \np{Paul}, qui s'appelait encore \np[Paul]{Saül}.}
\alexandrefr{De cette façon} donc il apprenait \alexandrefr{la loi}, \alexandre{recitation}{comme s'il n'avait pas besoin des livres.}
Lorsque le Sauveur \alexandre{paralytique}{guérit le paralytique à la porte des brebis}, il avait \np{Barnabé} comme admirateur du prodige.  
C'est donc pourquoi, ayant scruté la profondeur de son cœur, il lui \alexandre{communiquer}{communiqua} aussi un discours et assurance dans la parole.


Et donc \np{Barnabé} ayant atteint  \alexandre{maisonMarie}{la maison de \np[mère de Jean-Marc][]{Marie} la mère de \np[Jean-Marc]{Jean}, qu'on appelait \np[Jean-Marc]{Marc} --- \ctrad{on disait qu'elle était  sa tante}{\lit{la parole la tenait pour sa tante}}} ---  lui dit : 
\enquote{\alexandre{doitvenir}{Celui qui doit venir}
que \alexandre{desirvoir}{nos pères ont désiré voir}, voici qu'il est arrivé
.}
\alexandre{visitetemple}{Et immédiatement laissant} tout tomber, elle se rendit au sanctuaire et s'agenouillant au pied du Christ, elle le supplia de venir dans sa maison.
Et il s'y rendit tout de suite,  et c'est là qu'il accompli \alexandre{paque}{la Pâque avec ses disciples} ;
de nouveau \alexandrefr{une parole} est parvenue \alexandrefr{jusqu'à nous}, ainsi :  \alexandrefr{celui qui porte la cruche d'eau},
\alexandrefr{que} le  Christ a confié \alexandrefr{aux disciples pour qu'il le suive},  \alexandrefr{c'était  \np[Jean-Marc]{Marc}} le sage, \alexandre{cruche}{le fils de cette \np[mère de Jean-Marc]{Marie}}.
En outre  dans cette maison,  \alexandrefr{et après l'ascension,   les disciples montèrent} à nouveau \alexandre{ascension}{avec le reste des frères, parmi lesquels il y avait aussi \np{Barnabé} et \np[Jean-Marc]{Marc}} : car \alexandrefr{\np{Barnabé}} aussi \alexandrefr{a suivi} le Christ qui allait \alexandre{voyage}{de \nl{Jérusalem} en \nl{Galilée}}, et ainsi il a été compté parmi les soixante-dix, %%% ajout par rapport à Alexandre
appelé par tous \emph{fils de consolation}.
Car \alexandre{vente}{entendant le Seigneur dire : \enquote{vendez vos biens}} et ce qui suit, \alexandrefr{n'hésitant en rien, abandonnant les biens} de ses ancêtres, \alexandrefr{qui étaient de grand prix}, \alexandrefr{il les vendit} au plus offrant %% trouver dans dictionnaire grec byzantine diapwlew
 partageant avec ceux qui étaient dans le besoin, \alexandrefr{gardant seulement} un champ à \nl{Jérusalem} pour \alexandrefr{sa propre subsistance} ; qu'\alexandrefr{après la descente   de l'Esprit} Saint il échangea également  contre sa valeur, il la porta et \alexandre{ventebis}{la  déposa aux pieds des apôtres}.
 
\alexandrefr{Il parlait et se disputait avec \np[Paul]{Saül}, désirant le conduire} à la connaissance \alexandre{moquerie}{du Seigneur.
\np[Paul]{Saül} se moquait de \np{Barnabé} cependant qu'il blasphémait} au sujet du Christ.
Voyant aussi \alexandrefr{les prodiges qui eurent lieu} auprès \alexandrefr{des apôtres et la foule qui prenait le parti de la parole de la foi, il se mordit l'âme}, et \alexandrefr{se jetant sur \np{Étienne}, et ne pouvant s'opposer à la sagesse et à l'esprit, par lequel il parlait,  l'ayant fait périr} par colère grâce \alexandre{Etienne}{aux déserteurs, il suscita une grande persécution contre l'Église qui était à \nl{Jérusalem}}.% vérifier concordance avec txt alexandre
Néanmoins après qu'\alexandrefr{il eut été privé de vue, \np{Barnabé}} le  rencontrant \alexandre{bienfaiteur}{dit  : \enquote{Pourquoi poursuis-tu le bienfaiteur ? Cesse de détruire l'Église du Christ.}} %% trouver le γουν
Il parla ainsi, et \np{Paul} se couvrant de regrets pour ses actes, demandait la pardon et confessait que Christ est Dieu. % confessait le Christ Dieu ? Confessait Dieu le Christ ?
\alexandrefr{Alors le grand \np{Barnabé} prenant sa main le conduisit auprès des apôtres,} leur faisant savoir qu'il était une brebis au lieu d'un \alexandre{loup}{loup}.
\alexandrefr{Alors \np{Paul} leur décrivit comment à \nl{Damas} il avait vu le Seigneur sur la route, et il était avec eux un avec eux un enseignant de la parole} et un héraut   \alexandre{Damas}{à \nl{Jérusalem}.}
Ainsi \alexandrefr{les \npe{juifs}} entreprenaient de \alexandrefr{le faire périr}, mais \alexandrefr{les apôtres} le firent sortir \alexandre{juifs}{vers sa propre patrie pour proclamer} et enseigner l'Évangile ;
aussitôt certes ils confièrent également à \alexandre{pasteur}{\np{Barnabé}, comme il était puissant,  d'être pasteur pour la très sainte Église qui se tenait là.}
Et lorsqu'il eut fait ceci, \alexandrefr{un peuple} important \alexandrefr{fut établi} pour \alexandre{peuple}{le Seigneur}.

\alexandrefr{De là} donc \alexandrefr{il partit}, parcourant \alexandrefr{toutes les villes et les régions}, et jusqu'à ce qu'il parvienne à celle de \alexandrefr{\nl{Rome} : car lui-même proclama à \nl{Rome} l'évangile pour tous} les autres. Puis  \alexandre{traversee}{fuyant l'adoration} il \alexandre{atteindre}{atteignit} \nl{Alexandrie}.
\alexandrefr{De là,} il alla \alexandrefr{à nouveau à \nl{Antioche}}, et à partir d'elle il rejoignit \alexandrefr{\nl{Tarse}.}
Et y \alexandrefr{ayant trouvé \np{Paul}} et \alexandrefr{s'étant rendu à \nl{Antioche}} il termina \alexandre{peuple}{une année complète}, enseignant avec lui le peuple.
Puis \alexandrefr{quatorze ans après la Passion de Seigneur}, ainsi qu'il le dit, ils se rendirent de nouveau à \alexandre{Jerusalem}{\nl{Jérusalem}} et \alexandrefr{ayant serré la main droite des apôtres},
de telle sorte qu'\alexandrefr{eux proclament aux nations et que ceux qui étaient autours de \np{Pierre} proclament à la circoncision,
ils se rendirent à \nl{Antioche},} emmenant aussi \alexandrefr{\np[Jean-Marc]{Marc}. Depuis} cette ville
\alexandrefr{ils allèrent à \nl{Chypre} et ils} l'\alexandrefr{avait entièrement parcouru depuis} \nl{Salamine} et jusqu'à \alexandrefr{\nl{Paphos}, lorsqu'ils aveuglèrent   \np{Élymas} et éclairèrent à la fois le proconsul et ils avaient enseigné un certain nombre de personnes lorsqu'ils prirent} de là \alexandre{partage}{la mer pour aller en \nl{Pamphylie}.}
Donc \np[Jean-Marc]{Marc}, en tant qu'il était nouveau et qu'il n'avait pas la force de se maintenir pour les combats, se rendit chez sa mère dans Jérusalem. 
Lorsque \np{Barnabé} et \np{Paul} s'y rendirent à nouveau, \alexandrefr{\np[Jean-Marc]{Marc} voyant} qu'ils étaient aimés de \alexandrefr{tous et qu'après de tels périls} et supplices \alexandrefr{ils étaient robustes} et bien portants, s'approche de \np{Barnabé} --- et parce qu'il craignait \np{Paul} --- et \alexandrefr{en larmes} il demandait \alexandre{remords}{le pardon}.
Celui-ci lui dit \enquote{\alexandrefr{Que soit faite la volonté du Seigneur. Seulement, toi, sois prêt} à réaliser \alexandre{pardon}{tes promesses}.}
Après cela, \alexandre{compagnon}{tandis que \np{Barnabé} s'approchait de \np{Paul}, il l'exhortait à ce que \np[Jean-Marc]{Marc} devienne un compagnon de voyage pour eux}, 
mais \np{Paul} ne voulu pas cela,  parce que Dieu l'avait prévenu très fortement de tout ce qui concernait \np[Jean-Marc]{Marc}. %%% vérifier sens
En conséquence, \alexandre{embrasement}{il advint un embrasement pour les apôtres}, non d'une colère envahissante, \alexandrefr{loin s'en faut --- car comment} les artisans de paix et les disciples \alexandrefr{du} pacifique \alexandrefr{Christ} seraient-ils pris par cette \alexandre{souffrance}{souffrance} ? ---, mais \alexandrefr{un embrasement de zèle} divin :
\alexandrefr{car tandis que lui recherchait une rigueur,
se faisant remarquer par la perfection apostolique,
\np{Barnabé} révérait} avec beauté \alexandre{zele}{l'amour de l'humain}.
Et c'est pourquoi, \alexandrefr{prenant \np[Jean-Marc]{Marc} il navigua jusqu'à \nl{Chypre}} %%% rendre le ap ?
et il atteignit \nl{Salamine},
\alexandre{prenant}{proclamant le royaume de Dieu.}
Considère comment la séparation des apôtres faisait beaucoup plus de personnes s'approchant du Christ, 
 ce qui est justement le choix de la prévoyance de Dieu : car  \np{Paul} les conduisait d'un côté, \np{Barnabé} de l'autre ; as-tu vu \alexandre{zeleminus}{un embrasement de zèle} divin et une méthode habile ?
 
Mais alors que \np{Barnabé} \alexandrefr{demeurait à} \nl{Salamine}, \alexandre{Syrie}{des \npe{Juifs} de \nl{Syrie}} qui étaient présents le dénigrèrent, convainquant le peuple qu'il n'enseignait pas la vérité.
Alors l'apôtre connaissant  leur complot et parce qu'il était sur le point de finir, \alexandrefr{prenant du pain et une coupe}, et accomplissant la tradition mystique,   \alexandrefr{participa aux mystères} non souillés, en donnant également une part \alexandre{eucharistie}{aux frères}.
Puis  \alexandre{Marc}{prenant \np[Jean-Marc]{Marc}} auprès de lui, il lui fait connaître sa fin et  lui confie de \alexandrefr{sortir de la ville par l'Ouest  % ou au couchant ?
trouver son corps et l'enterrer}, mais aussi de \alexandrefr{partir} alors \alexandrefr{auprès de \np{Paul}} avec ces informations, d'être avec lui et de proclamer la Parole. \enquote{\alexandrefr{Car} il dit que \alexandre{Marcnom}{tu es sur le point d'être magnifié sur toute la terre.}}
\alexandrefr{\np{Barnabé}} parla puis \alexandrefr{s'installa à la synagogue}, persuadant par des paroles divinement inspirés la foule au sujet du \alexandrefr{Christ}, qu'\alexandrefr{il est le fils du Dieu vivant}.
Alors, \alexandrefr{remplis de colère, les} \npe[juifs]{hébreux} qui \alexandrefr{venaient de \nl{Syrie}, mirent les mains sur lui} et l'enfermèrent \alexandrefr{dans une pièce sombre}. Puis \alexandre{mort}{au soir profond,  l'ayant emmené dehors et torturé fortement, ils le conduisirent de nuit hors de la ville ; et là} ils lui jetèrent des pierres comme au proto-combattant.
Et regarde leur degré de malfaisance : r\alexandrefr{allumant} aussi \alexandrefr{un feu}, ils l'y lancèrent au milieu, même si il avait été conservé \alexandrefr{pur par la providence de Dieu}.  Donc \alexandrefr{\np[Jean-Marc]{Marc}} faisant \alexandre{depouille}{selon sa prescription,  transporte avec quelques frères la dépouille de  \np{Barnabé} et l'enterre} avec révérence.
Et ainsi \alexandrefr{\np[Jean-Marc]{Marc} part se rendre auprès de \np{Paul} à \nl{Éphèse}} et décrit à celui-ci \alexandrefr{qu'elle avait la fin de \np{Barnabé}}.
Alors \np{Paul} pleurant vraiment,  \alexandrefr{retint \np[Jean-Marc]{Marc} auprès de lui. Après cela, alors que \np{Pierre} partait} pour \alexandrefr{\nl{Rome} il prit} aussi  \alexandre{PierreRome}{\np[Jean-Marc]{Marc} avec lui, adoptant pour ainsi dire} celui-ci.
De là donc, \np[Jean-Marc]{Marc} ayant écrit son évangile sous une proposition du premier des sièges et ayant fortement brillé,  brillé, il fut consacré par celui-ci  proèdre d'\nl{Alexandrie} et  y fut envoyé : y \alexandrefr{ayant enseigné la parole} de la foi \alexandre{martyrMarc}{durant neuf \textins{années}, il finit en martyr}. 
Et c'est assez. 

Au temps de l'Empereur \np{Zénon}, ce corps apostolique du beau \np{Barnabé} a été rendu visible de la manière qui suit. L'apôtre est apparu de nuit à l'évêque de \nl{Salamine} --- c'était le glorieux \np{Anthémios} --- \enquote{Viens-donc, lui dit-il, \np{Anthémios}, \alexandrefr{va hors de la ville, vers un lieu} éloigné d'elle de  \alexandrefr{cinq stades à l'Ouest} et \alexandrefr{dit de la guérison}, là où \alexandrefr{des prodiges} fréquents s'accomplissent \alexandrefr{à  travers moi : fouille à l'emplacement du caroubier et 
 tu trouveras une grotte et un coffre} qui y a été déposé, et dans celui-ci \alexandrefr{mon corps mis de côté et l'Évangile} du Christ, c'est celui \alexandre{vision}{que j'ai reçu de la main-même de \np{Matthieu} l'Évangeliste.}} %%% changement étrange sur le caractère autographe du manuscrit.
 Et \alexandrefr{l'évêque} : \enquote{Et \alexandrefr{Qui est-tu, monseigneur} ? dit-il, afin que j'aie une claire compréhension, de quel dépouille s'agit-il donc ?}. Et lui : \enquote{\alexandrefr{Moi je suis \np{Barnabé}}, dit-il,  l'\alexandre{reponse}{apôtre du Seigneur}.}
Se rendant aussitôt  donc au lieu et creusant avec foi, il trouva le coffre, celui dans lequel était le corps apostolique --- on aurait dit qu'il avait été inhumé à l'instant ---, et \alexandre{evangile}{l'évangile, déposé sur la poitrine} de celui-ci.
Il était également protégé par des sceaux de plomb, 
il vint rapidement à la ville impériale et décrit tout en détail à l'empereur. % litt décrit
Celui-ci, glorifiant grandement Dieu à son propos, réclame  à prendre seulement l'évangile.
L'ayant reçu, il le \alexandrefr{garnit d'or} et \alexandrefr{l'installe dans le palais}, et il reste \alexandrefr{conservé jusqu'à} maintenant.
\alexandrefr{Car} lors du \alexandrefr{grand Jeudi}, cοmme on dit, \alexandre{jeudi}{on y lit l'Évangile} du jour.

Ces joutes pour le Christ sont tiennes, tout-merveilleux,
et les combats et les luttes.
Et en outre toi, digne de Dieu,  tu es établi un rang  au dessus du monde,
demande aussi pour notre empereur bon et noble en tous
que lui soit donné:

\stanza
	une  vie très longue et triomphante des souffrances&
	ayant le désir de s'éloigner pleinement de toute impureté&
	protégée par la main de la parole de Dieu&
	supprimant des nombreuses ignorances chez les peuples&
	conquérant leurs régions et leurs villes&
	pour de brillants trophées et des victoires glorieuses&
	le don de tous les biens célèstes\&
\noindent
et le domaine du royaume de Dieu ; parce que la gloire convient à  Christ notre Dieu, maintenant et pour les siècles sans fin. Amen.


