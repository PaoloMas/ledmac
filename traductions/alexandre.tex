% analyse du vocabulaire :
% 		- disciples/apotres
%%% verifier coherence dans numerotation des psaumes


%\ledsection{\cite{BHG226}}


s
%%% choix du titre : martyre, louange, récit de voyage ? voir si cela correspond à des différences de fond 

% l.6
s



s

% l. 52
s

\np{Barnabé}, \reprise{ornement}le vénérable ornement des \npe{Chypriotes}\reprisef{ornement} et l'invincible défenseur de la terre habitée,
celui qui aime abondamment le Christ, qui  chaque jour  pour lui consacre son âme et qui règne avec lui aux siècles des siècles. %%% !!!! pas très orthodoxe. Voir  ce qu'il en est pour d'autre saint

%l. 132
Le discours a travaillé à exalter ce divin et trois fois regretté apôtre \np{Barnabé} par les éloges de regrets, et il n'a pas encore atteint le prélude.
Car le  merveilleux  se trouve inaccessible aux louanges. %?
C'est pourquoi négligeant %
ce qui semble un inabordable discours d'éloges,
 je laisserais  de côté  quelques un des petits détails pour nous tirés des Stromates et d'autres anciens ouvrages  concernant sa manière de vivre et ses vertus  à votre piété personnelle, et ainsi nous poserons un terme à l'énonciation,
abandonnant  à l'Écriture divinement inspirée le soin de couronner magnifiquement la tête de celui qui est digne d'être chanté. %%%% Absolument à relire
Car elle dit : \enquote{\np{Barnabé} \bible{Ac|11:24}{était un homme bon, rempli de l'Esprit saint et de foi.}}
Que pourrait-on dire de lui qui ne serait égal ou semblable en quelque façon ?

\saut

\reprise{provenait}Ce trois fois bienheureux provenait\reprisef{provenait} donc \reprise{Levi}de la tribu bénie de \np{Lévi}\reprisef{Levi}\bibleall{Ac|4:36}, 
il était originaire de celle de \np{Moïse} et d'\np{Aaron}\bibleall{Ex|2:1-2} % absent des actes, voir si l'iconographie détail plus
les grands prophètes de Dieu et les premiers du peuple de \npe{Qehath}\bibleall{Ex|6:18-20},
\reprise{Samuel}selon la race, de la parenté de \np{Samuel} le prophète\reprisef{Samuel}\bibleall{ICh|6:1-13}. 
\reprise{ancetre}Les ancêtres de celui-ci, retenus dans la région de \nl{Chypre}\bibleall{Ac|4:36} à cause de circonstances de guerres, %%% quelle guerre auraient pu faire que des juifs soients immigrés à Chypre ? l'auteur pense-t-il à qq chose de son tps ? voir également les variantes textuelles imporantes
y habitaient avec bienveillance\reprisef{ancetre}. %%% on remarque le contraste : ils sont contraint, mais restent bienveillants
Ils étaient \bible{Ac|22:12}{pieux selon la Loi} et tout à fait riches
 \reprise{champ}c'est aussi pourquoi ils avaient à \nl{Jérusalem} une fortune suffisante et un  champ très magnifique proche de la ville\reprisef{champ}, qui n'était pas seulement orné d'arbres fruitiers de toute sorte% fertiles plantes
mais qui par sa grandeur était aussi  le plus visible des domaines.
Car depuis que les enfants des \npe{Hébreux}, qui recevaient la prophétie de manière charnelle, avaient entendu le prophète \np{Isaïe}  dire : \enquote{\bible{Is|31:9|LXX}{Heureux celui qui a une descendance à \nl{Sion} et des possessions à \np{Jérusalem}}}, chaque personne qui en avait les moyens possédait un bien à \nl{Jérusalem}.

\reprise{Chypre}Lors de la naissance de de ce juste à \np{Chypre}\reprisef{Chypre}, comme ses parents \bible{Ex|2:2}{virent qu'il était beau}  à cause de Dieu, 
ils l'appelèrent aussitôt \np[Barnabé]{Joseph}\bibleall{Ac|4:36}, jugeant l'enfant digne du nom du patriarche\indexnp[le patriarche][]{Joseph} ; 
cependant que la noblesse des mœurs concurrença celle qu'il tenait de son nom.
Car Joseph signifie \enquote{ajout de Dieu}\bibleall{Gn|30:24}.
En effet le juste tint l'ajout de la grâce d'auprès de Dieu afin qu'il atteigne la perfection apostolique.
Joseph signifie également \enquote{gloire de Dieu}. %%% fantaisiste ? l'éditeur n'a pas trouvé en tt cas
Car il devint gloire de Dieu par son noble mode de vie.
Et que personne ne tienne ce discours pour une  exagération, 
mais qu'il croit qu'il relève de la divine écriture.
Du moins \np{Paul} dit : \enquote{\bible{ICo|11:7}{L'homme ne doit pas se voiler la face, étant image et gloire de Dieu.}}
Si donc \bible{Ac|9:15}{l'instrument de le l'élection} appelle l'homme commun \bible{ICo|11:7}{image et gloire de Dieu}, comment quelqu'un parlerait-il de l'homme très parfait selon Dieu ?

\reprise{education}Lorsque \np{Barnabé} devint plus grand, ses parents \bible{Lc|2:22}{le montèrent à \nl{Jérusalem}}
et le confièrent pour qu'il étudie avec précision la loi et les prophètes  \bible{Ac|22:3}{aux pieds de \np{Gamaliel}} :% pas dans les actes
il avait pour condisciple \np{Paul}, qui s'appelait encore \np[Paul]{Saül}\reprisef{education}.
\np{Barnabé} progressait ainsi chaque jour dans la connaissance\bibleall{Lc|2:52} et dans toute  vertu :
en vérité on ne  l'avait pas encore placé pour le service quotidien des Lévites à cause de l'insuffisance de l'âge, car il était toujours un jeune homme.
\bible{Lc|2:37}{Il ne s'écartait pas du temple, participant au culte par des jeûnes et de prières nuit et jour.}
\reprise{recitation}De cette façon il récitait la loi et le reste des écritures, comme s'il n'avait pas besoin de se  souvenir des lettres\reprisef{recitation} ; il en avait ainsi un amour silencieux, comme une mère de prudence. 
Il fuyait les bavardages nuisibles, les ayant en horreur, demeurant une offrande pure, parfaite, et non souillée : en tout il était glorieux par la vertu. %191 ressemble à certaisn discours du pro de jc. lieux commun ...


En ce temps là il arriva que le Seigneur se rendît à \nl{Jérusalem}, \reprise{paralytique}qu'il guérît le paralytique à la porte des brebis\reprisef{paralytique}\bibleall{Jn|5:1-9}  et qu'il accomplît dans le temple de nombreux autres signes et prodiges\bibleall{Mt|21:14}.   %%% comment rendre le kairos ? et παραγιγνομαι -> pas de subjonctif car c'est un fait certain
Voyant ces faits, le bienheureux était frappé de stupeur, et s'étant aussitôt approché il tomba à ses pieds et demandait à être béni par lui\bibleall{Lc|8:41}. 
Le Christ  pénétrant les cœurs, faisant connaître sa foi, l'accueillit avec bienveillance et  \textins{lui} \reprise{communiquer}fit part\reprisef{communiquer} des événements divins le concernant. 
Celui-ci brûlait totalement d'amour pour le \np{Seigneur}. 
Se rendant rapidement à 
\reprise{maisonMarie}\bible{Ac|12:12}{la maison de \np[mère de Jean-Marc][]{Marie}, la mère de \np[Jean-Marc]{Jean} qui était appelé \np[Jean-Marc]{Marc}},
celle-ci disait être sa tante --- c'est pourquoi on appelait \bible{Col|4:10}{\np{Jean-Marc} Marc le cousin de \np{Barnabé}} ---,
il  lui dit\reprisef{maisonMarie} :
\enquote{Ô femme, dit-il, vois ici ce que \reprise{desirvoir}nos pères ont désiré voir\reprisef{desirvoir}\bibleall{Mt|13:17},
car voici que \biblefr{\np{Jésus}}, un certain \bible{Mt|21:11}{prophète originaire de \nl{Nazareth} de \np{Galilée}} est au temple accomplissant des prodiges extraordinaires, et ainsi comme il apparaît à beaucoup, \biblefr{il est} le Messie, \bible{Mt|11:14}{\reprise{doitvenir}celui qui doit venir\reprisef{doitvenir}}.}
\reprise{visitetemple}Après avoir entendu ces faits  et laissé tomber ce qu'elle avait en main, la femme étonnée se rendit au temple de Dieu, et après avoir vu le seigneur et maître du temple, elle se jeta à ses pieds\bibleall{Jn|11:32}, demandant et disant :
\enquote{Seigneur, \biblefr{si j'ai trouvé grâce à tes yeux}, pars d'ici pour aller dans la maison de ta servante,
et \biblefr{bénis}  tes domestiques \bible{Gn|30:27}{par ta venue}.\reprisef{visitetemple}
}
Le Seigneur se pencha sur sa prière : 
s'étant réjouit, elle lui montre la chemin vers l'étage supérieur de sa maison.
%l. 215
Donc à partir de ce jour là, lorsque le Seigneur venait à \np{Jérusalem}, 
c'est là qu'il dormait avec ses disciples, 
\reprise{paque}c'est là qu'il fit la Pâque avec ses disciples\reprisef{paque}\bibleall{Mt|26:17-18+Mc|14:12-15+Lc|27:7-12},
c'est là qu'il enseigna ses disciples par l'échange des mystères ineffables.
\reprise{cruche}Car un parole vint à nous depuis les anciens :
\biblefr{celui qui porte la cruche d'eau}
que le Seigneur a désigné à ses disciples pour qu'ils le suivent\bibleall{Mc|14:23+Lc|22:10},
c'était \np[Jean-Marc]{Marc}, le fils de cette bienheureuse \np[mère de Jean-Marc][]{Marie}\reprisef{cruche} :
le Seigneur  dit comme un bon intendant \enquote{\bible{Mt|26:18}{\textins{allez} vers untel}},
comme disent les pères, faisant  ainsi connaître l'emplacement,
nous enseignant par énigmes que à quiconque se réconcilie avec lui même, le Seigneur demeure auprès de lui.% 4 juin 2013 16h20 iciq
Dans cet étage en effet, le Seigneur fit la Pâque\bibleall{Mt|26:17-18+Mc|14:12-15+Lc|22:7-12}. %%% plein de références à creuser. En fait pour Hesychaste / Eutyches c'est juste des références à des discours sur le lieux de la pâque =  SION. A verifier tt de même Hesychii Hierosolymitani Interpretatio Isaiae prophetae (Strasbourg fac th. Lyon SC, Paris IFEB)
Il  y apparut à ceux qui étaient autour de Thomas, ressuscité des morts\bibleall{Jn|20:24-29}.
\reprise{ascension}Là les disciples montèrent après l'ascension, venant du \nl{mont des oliviers},
étant au nombre de cent-vingt, parmi lesquels se trouvaient \np{Barnabé} et \np[Jean-Marc]{Marc}\bibleall{Ac|1:4-15}.\reprisef{ascension}
Là l'\np{Esprit} saint descendit en langues de feu sur les disciples le jour de la Pentecôte\bibleall{Ac|2:1-13}. 
Là a été installé la grande et très sainte \np{Sion}, la mère de toutes les Églises.%%% sans doute Sion = Eglise de Jerusalem. cf liturgie de st jacques, voir si Hesychii Hierosolymitani Interpretatio Isaiae prophetae parle de l'ecclesia de sion. Cf également Hesch. Const. Vita. Euty. 356-358 p. 14. Le paraclet est introuvable en europe. Chercher Der heilige Theodosios : Schriften des Theodoros und Kyrillos / hrsg. [mit Komm.] von Hermann Usener -> Geneve, Paris IFEB, Lyon SC, ENS ULM,Institut de Fr, Strasbourg BNF

\reprise{voyage}Alors \np{Barnabé} suivit le \np{Seigneur} qui retournait de \nl{Jérusalem} en \nl{Galilée}\reprisef{voyage}. 
Alors que de toute part de nombreuses personnes s'approchaient du \np{Seigneur}
et croyaient en lui, il dit aux disciples		: %%% tote + participial 0> redondant. comment traduire ? % l. 241
\enquote{\bible{Mt|9:37}{La moisson est abondante, mais les ouvriers peu nombreux}.}
Alors il consacra les soixante-dix disciples\bibleall{Lc|10:1}, 
parmi lesquels  le grand \np{Barnabé} se trouvait en premier, chef de chœur et choryphée. %%% déciédment !
Que personne cependant,
apprenant que les les apôtres lui ont attribué ce qualificatif\bibleall{Ac|4:36},
ne présume sans savoir
qu'il a reçu ce qualificatif sans inspiration divine. 
Car \np{Pierre} lui a attribué ce nom, 
à travers une révélation du saint \np{Esprit}
et aussi par une révélation du \np{Père} qui montre la divinité du \np{Fils}\bibleall{Mt|16:16-17} ;
et comme \bible{Mc|3:17}{\np[fils de Zébédée][]{Jacques} et \np{Jean} fils du tonnerre ont été nommés à partir de la qualité},
de même  aussi \np{Barnabé} est appelé   
\bible{Ac|4:36}{fils de la consolation}  à parti de la qualité,
devenu consolation de tous à cause d'une très sainte supériorité. 

\reprise{vente}Entendant le Seigneur enseigner et dire :
\enquote{\bible{Lc|12:33}{%
Vendez vos biens\reprisef{vente} et donnez une aumône,
faites vous des bourses qui ne vieillissent pas, 
un trésor perpétuel dans les cieux}},
\reprise{ventebis}n'hésitant en rien,
il abandonna les biens hérités de ses ancêtres --- car ils avaient quitté cette vie --- et qui étaient  de grand prix. 
Les vendant tous, 
il partagea avec ceux qui étaient dans le besoin, 
gardant seulement pour lui-même ce champ-là en vue de son propre entretien. 
Après la Mort, la Résurrection et l'Ascension du Seigneur et la venu Saint Esprit,
le divin \np{Barnabé} brûlant davantage d'amour envers le \np{Seigneur}, 
il rendit même ce champ-là et en tira une certaine somme, 
emportant tout,
\bible{Ac|4:37}{il  la déposa aux pieds des apôtres}\reprisef{ventebis},
n'en laissant rien du tout pour lui-même,
par l'exemple qu'il donnait lui-même, il excitait tout les  disciples à la même vertu. %%%  

\reprise{moquerie}\bible{Ac|9:21}{Il parlait et se disputait} avec \np[Paul]{Saül}, désirant le conduire à la même foi du \np{Seigneur}.  %%% variante intéressant sur aυτον et αυτου
En vérité \np[Paul]{Saül} s'appuyant sur la citoyenneté consciencieuse  selon la loi,
se moquait de \np{Barnabé}, comme une personne dupée, 
cependant qu'il blasphémait le \np{Seigneur}\reprisef{moquerie}, le surnommant \bible{Mt|13:55}{fils de charpentier}, ignorant, paysan et bandit. %%% Terme courant dans la polémique païenne pour désigné le Christ. Cf Martyrium Pionni 13.3
\reprise{Etienne}Quand il vit les  grands faits de prodiges qui eurent lieu à travers les apôtres et la foule du peuple qui  chaque jour prenait le parti de la parole de la foi\bibleall{Ac|2:43+Ac|5:12+Ac|5:14+Ac|11:24}, il se mordit l'âme. 
Jetant, avec \biblefr{les affranchis, les \npe{Cyrénéens} et les \npe{Alexandrins}}, une accusation contre \biblefr{\np{Étienne}} le grand rhéteur de l'Église, et ne pouvant \bible{Ac|6:9-10}{s'opposer à la sagesse et à l'esprit, par lequel il parlait}, il se dirigea vers la folie ;
et se remplissant de désir, il leva contre lui les déserteurs au sein du peuple, et l'ayant fait périr, il suscita \bible{Ac|8:1}{une grande persécution contre l'Église qui était à \nl{Jérusalem}}.\reprisef{Etienne} %%% susciter ou ranimer ? 
Mais alors que non préparé au combat il se rendait ainsi à \nl{Damas} pour faire du mal aux croyants, le Seigneur le renversa, le jetant face à terre\bibleall{Ac|9:3-5+Ac|22:6-8+Ac|26:12-15} ; et celui qui était tombé à terre, reconnu celui qu'il persécutait, et privé de la vue, il regarda tout à fait vers le sommet des cieux.
Retournant alors \biblefr{vers \nl{Jérusalem}}, il chercha à \biblefr{rejoindre les disciples}, et  \biblefr{tous le} fuyaient, craignant son importante cruauté\bibleall{Ac|9:26}. %%% midi arret on remarque que l'auteur zappe totalement le récit d'Ananias
\reprise{bienfaiteur}Le grand \np{Barnabé},  allant au devant de lui, dit :
\enquote{Jusqu'à quand, \np[Paul]{Saül}, rencontreras-tu \np[Paul]{Saül} ?
Pourquoi poursuis-tu ainsi le bienfaiteur  avec ardeur ?
Cesse de détruire\reprisef{bienfaiteur} le terrible mystère proclamé depuis longtemps par les prophètes\bibleall{Ac|9:21+Ga|1:13+Ga|1:23} et révélé en nos temps pour notre salut\bibleall{Rm|16:25-26+Col|1:26}.
}
Entendant cela, \np[Paul]{Saül} se jeta au pied de \np{Barnabé} avec beaucoup de larmes, il cria et dit :
\enquote{Viens avec moi, guide-moi vers la lumière et enseigne-moi la vérité ; car j'ai appris par l'expérience la vérité des tes paroles ; car moi j'ai blasphémé en parlant de lui comme \bible{Mt|13:55}{fils du charpentier}, maintenant je confesse qu'il est \biblefr{le fils} unique engendré \bible{Mt|16:16}{du Dieu vivant}, de même essence et recevant la même gloire et ayant le même trône,
co-éternel et partageant l'absence de commencement, \bible{He|1:3}{lui qui est un rayonnement de la gloire et une marque de l'être} du Dieu invisible,
\bible{He|1:3}{à la fin de ces jours-ci, pour nous et notre salut}, \bible{Ph|2:7}{il s'est anéanti, prenant la condition d'esclave}, c'est à dire d'un homme parfait issu de la sainte vierge et mère de Dieu \np[mère du Christ][]{Marie},
sans confusion, sans changement, sans division, sans séparation ;
et \bible{Ph|2:7-8}{reconnu comme homme à son apparence, il s'est abaissé lui-même, devenant obéissant jusqu'à la mort, et la mort sur la croix} ; lui qui est aussi ressuscité des morts le troisième jour et qui est apparu à vous, ses apôtres\bibleall{ICo|15:4-5}, 
\bible{Mc|16:19}{qui a été enlevé au ciel, qui s'est s'assis à la droite} du Père et qui \ctrad{reviendra}{\lit{revient}. Le symbole de Nicée-Constantinople utilise des participes présents.} %%% littéralement au présent
avec gloire juger les vivants et les morts\bibleall{IITm|4:1}  et son règne n'aura pas de fin\bibleall{Lc|1:33}.
}

Le divin \np{Barnabé} entendant cela  d'auprès du \bible{ITm|1:13}{blasphémateur et du persécuteur} fut stupéfié et sa face se couvrit de joie, comme s’il était une fleur matinale ;
l'embrassant et le baisant tendrement, il lui dit : \enquote{
Qui t'a appris, \np[Paul]{Saül}, à prononcer ces paroles-là divinement inspirées ?
Ou qui t'a fait confesser \np{Jésus} le Nazaréen fils de Dieu ? 
Ou à quel endroit a tu été instruit de cette exactitude des dogmes divins ?
} 
Celui qui était encore baissé et en larmes, avec de nombreuses componctions dit : %%%àd le encore pour rendre le parfait 
\enquote{Le Seigneur \np{Jésus} lui-même, celui qui de nombreuses fois a été dénigré et poursuivi par moi, le pécheur\bibleall{ITm|1:13}, m'a enseigné toutes ces choses ;
car pour ainsi dire \bible{ICo|15:8}{il est aussi apparu à moi, l'avorton}, et j'ai encore à l'oreille sa voix divine et douce ;
car ayant disposé tous les bienfaits, il me parla alors que j'étais étendu à faire pitié sur la face, me défendant bien plus que m'accusant : 
\enquote{\bible{Ac|22:7}{\np[Paul]{Saül}, \np[Paul]{Saül}, pourquoi me persécute tu ?}}
Et \biblefr{moi je répondis} avec frisson et crainte : 
\enquote{\bible{Ac|22:8}{Qui es-tu Seigneur ?}}
Le Seigneur me dit avec beaucoup de justice et de compassion :
\enquote{\bible{Ac|22:8}{Moi je suis \np{Jésus} le Nazaréen que tu persécute.}}
Et moi, stupéfié par sa patience indicible, je lui demandais :
\enquote{\bible{Ac|22:10}{Que ferai-je, Seigneur ?}}
Sur le champ il m'instruisit de tout cela et de plus nombreuses encore. 	%%%% rendre le toutwn. p. 339 
}

\reprise{loup}Alors le grand \biblefr{\np{Barnabé} prenant} sa main \bible{Ac|9:27}{le conduisit auprès des apôtres}, en disant :
\enquote{Pourquoi fuyez-vous le berger en pensant qu'il est un loup ?\reprisef{loup}
Pourquoi poursuivez-vous le timonier comme un pirate ? %%%% vocabulaire de la mer
Pourquoi méprisez-vous le plus brave comme un déserteur ?
Pourquoi congédiez-vous le témoin de mariage comme un usurpateur dans la chambre nuptiale ?  %%% autre terme ? +-moins littérale ?
Car l'Église se trouve être un lit nuptial spirituel, dont le Seigneur a désigné de sa propre main un  berger, timonier et protecteur.}  %%% l. 347. Qu'est ce que cela veut-il dire ? trouver autre chose que rencontrer8
\reprise{Damas}Alors \np{Paul}  \biblefr{leur décrivit} les  très  grands choses qui lui étaient arrivés \biblefr{sur la route}, et qu'il \biblefr{avait vu le Seigneur, lui avait parlé et comment à \nl{à Damas} il avait discouru librement au sujet du nom du Seigneur ;} et il était avec eux un enseignant de la parole du Seigneur à \nl{Jérusalem}\bibleall{Ac|9:27-28}.\reprisef{Damas}
\reprise{juifs}Il était tout à fait pesant pour les \npe{Juifs} que celui qui la veille persécutait \np{Jésus} le proclame ce jour là Fils de Dieu, et ils  \bible{Ac|9:23}{voulurent le faire périr}. 
Les apôtres écoutant ses enseignements l'envoyèrent prêcher dans sa propre patrie\bibleall{Ac|9:30}.\reprisef{juifs} %%% trouver terme




