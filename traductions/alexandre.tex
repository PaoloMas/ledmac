% analyse du vocabulaire :
% 		- disciples/apotres
%%% verifier coherence dans numerotation des psaumes


%\ledsection{\cite{BHG226}}


\titulus{Louange du saint apôtre \np{Barnabé} par \np{Alexandre} le moine, à la demande du prêtre et gardien des clefs de son auguste sanctuaire, dans lequel on trouvera aussi le récit de la découverte de son saint tombeau.}
%%% choix du titre : martyre, louange, récit de voyage ? voir si cela correspond à des différences de fond 

% l.6
Votre  paternel amour de la connaissance proposa aux mendiants que nous  sommes un très grand projet de discours, 
ô meilleurs des pères et très considéré père des ascètes : 
car il m'a envoyé pour louer le divin,
et  ainsi en vérité trois fois bienheureux, % que faire du ws ?
très remarquable \bible{Ac|4:36}{fils de la consolation} parmi les apôtres et lumière de la terre habitée, le très renommé \np{Barnabé}.
Mais moi j'invoquais ma propre  inculture  contre  ce projet hors norme, durant un temps complet je refusais d'obéir,
craignant l'entreprise :
car aussi, quelle parole pourrait atteindre la perfection apostolique ?
Car une telle entreprise ne s'est pas seulement révélée impossible pour moi, mais aussi difficile pour des nombreuses personnes. 
Je présume en effet que toute nature humaine échouerait à la description des bonnes actions du très grand apôtre \np{Barnabé}.
Aussi si quelqu'un entreprend de dire quelque chose à son sujet, il omettra les actes glorieux en de nombreux points, et si \bible{Jc|3:16}{le sage et le savant} réussi beaucoup et est très puissant en ce qui concerne les discours, il se retrouvera cependant culbuté par l'\ctrad{étendu}{\lit{abîme}.} des merveilles.
Donc tel juste et tel autre juste ont reçu chacun une grâce de l'Esprit saint\bibleall{Rm|12:6+ICo|7:7+IP|4:10} :
ainsi tous les apôtres accueillirent le trésor des grâces et ils  présentèrent toute la  vertu. %% comment rendre le μεν ουν ?  
Comment donc le malheureux que je suis, noyé par d'infinies épreuves %%% vocabulaire marin
pourrait-il traverser à la nage la mer apostolique ? 
Car j'ai une faible voix pour faire des descriptions, je suis lent à parler, et moi \biblefr{je} ne \bible{Ex|4:10}{suis} pas  bon parleur pour décrire à un homme les vertus du divin apôtre \np{Barnabé}. 
En effet celui-ci a précisément  accompli ensemble toutes les vertus  , comme personne ne l'a fait pour aucune d'entre elles.
À cause de cela j'ai renvoyé l'ordre à de nombreuses personnes, sachant que ma propre indignité n'était pas assez puissante pour la description des  très célèbres succès de l'homme.% lieu commun



Se rappelant maintenant de l'écriture divinement inspirée, qui dit déjà qu'un \biblefr{fils insolent} est \bible{Pr|13:1|LXX}{dans la perdition}
alors que \bible{Pr|13:1|LXX}{celui qui écoute} est \bible{Pr|24:22|LXX}{hors de celle-ci},  
avec peine et de nombreuses craintes, je me suis redressé moi-même pour accomplir  ton commandement, 
pensant plus fortement être appelé à cause d'un manque  plutôt qu'être condamné à cause de l'absence complète de connaissance. 
Seulement \bible{Rm|15:30}{combattez avec moi dans les prières} % passage biblique
pour que l'Esprit totalement saint devienne pour moi une aide, 
car il  sera agréable, ainsi que je le pense, 
en prononçant aussi la bénédiction auprès des gens sans valeurs que nous sommes,
comme lorsque par le maître il rendit agréable  l'offrande  de deux pièces de la pauvre veuve\bibleall{Mc|12:41-44+Lc|21:1-4}.
Car la quantité de don n'est pas un critère d'acceptation pour Dieu, %%% peut etre changer la formulation,
mais  l'offrande faite avec une bonne pensée, soit petite,  soit grande, est sainte\bibleall{IICo|9:7}.

% l. 52
\emph{Louant} \np{Barnabé}, je louerai en même temps la prêtrise sacrée %%% redoublement du terme 
de tous les apôtres.
Car leur appel est unique et la gloire est propre, %%%% est ce lié au cas chypriote d'insister sur la tension commun/propre ? il me semble que dans la laudatio de tite par André de Crète c'est aussi commun
car la fonction est unique et la gloire est propre,
la disposition est unique et la dignité est propre,
les combats sont identiques et les couronnes ne s'échangent pas,
la citoyenneté est commune, 
leurs offices aussi sont égaux quant à l'honneur.
Allons en conséquence, si tu le souhaites, 
emmenons le divin \np{Barnabé} au milieu du discours
lorsque  nous tissons pour lui seulement l'éloge :  car je suis brûlant de passion pour l'homme.
Je sais et je suis persuadé que  l'apôtre est pour vous très doux et simplement appelé par son nom.
Donc honorons-le  selon sa puissance, car avec puissance il a prononcé des démonstrations de sagesse.
Consentons à ce que le discours en son honneur ait en tous mots des trophées de victoire.

\saut 
% l. 66 ss
Et bien ! Que \np{Barnabé}, le grand parmi les apôtres, soit chanté pour les plus petits de nous.
Que soit béni \bible{Ac|4:36}{le fils de la consolation} auprès de tout ceux qui croient en Christ.
Qu'il soit glorifié par toute la création celui qui est honoré par le Père\bibleall{Jn|12:26}, appelé par le Fils et conduit à la perfection par le Saint Esprit. 
\np{Barnabé}, le grand rhéteur de l'Église, la trompette de la prédication évangélique, la voix de l'unique engendré, la cithare de l'Esprit, le plectre de la grâce. 
\np{Barnabé}, le puissant soldat des guerres pour le Christ, 
celui qui dans le corps a mis en fuite le tyran sans corps avec toute l'armée qui partageait son apostasie, 
le pieux gymnaste des bons disciples,
l'infaillible guide du troupeau du Christ,
l'éloquent verger de Dieu,
celui qui a acquis en lui-même les fertiles arbres fleuris de tout mérite,
le fleurissant feuillage de la foi,
la très suave rose de l'amour, % l. 82
la pousse d'amarante de l'espérance, 
le très odoriférant \ctrad{bouquet}{\lit{épi}.} de la grâce,
le fertile serment de la vivifiante vigne,
la \ctrad{suave}{\lit{mielleuse}.} grappe de la résurrection, 
la ferme digue de la persévérance % Vocabulaire marin ? προβολος désigne tt ce qui est mis en avance. Mais avec l'adjectif ...
le fantassin porteur de la croix de la modération, % ascèse ?
le preux général contre les meurtriers du Christ,
la pousse du Christ crucifié s'étendant jusqu'au ciel,
\np{Barnabé}, \bible{Ac|4:36}{le fils de la consolation},
le pieux enseignant,
la colonne et la toiture de la foi,
la tour inébranlable, % inébranlable précédé d'un article, ce qui met en valeur l'adjectif
le fondement indissoluble, % idem
le rocher immuable, %idem
le port emmenant le calme aux agités,  % idem, apaisant : littéralement porteur de calme + vocabulaire marin (mais il s'agit d'une reprise d'un auteur, voir ce qu'il en est dans l'original)
la consolation de tous les opprimés,
\bible{Lc|12:42}{l'intendant  fidèle et avisé},
le meilleur des architectes, celui qui  depuis la terre a construit la voie céleste et la vie semblable aux anges,
le chef des Églises, % 
le protecteur des pauvre, le chorège  des indigents et le  nourricier,
l'encouragement des dépouillés et le père des orphelins plein de sollicitudes, 
\np{Barnabé}, le trésor des mystères du Christ,
le dispensateur des dogmes orthodoxes de la  sainte Église de Dieu,
l'honnête médecin des malades,
l'infaillible joie des bien portants,
le gardien vigilant  du troupeau du Christ, 
le pèlerin sur la terre et \bible{Ph|3:20}{citoyen dans les cieux}, 
celui qui proclame les pieuses dogmes dans le monde et possède des couronnes éternelles dans les cieux, 
le guide porteur de Dieu pour les peuples et le très bienheureux divin héraut des Églises,
la prairie de la \bible{IICo|2:15}{bonne odeur du Christ},
le buisson de roses des vertus du vent céleste,
le très fertile champ  des charismes du Christ,
celui qui suit sur la terre la citoyenneté incorruptible, aussi maîtrisé dans sa vie sur terre  que les anges incorporels dans les cieux, %%% le ως indique une comparaison d'égalité alors qu'on a comp de sup
celui qui brille dans l'ascèse, qui progresse dans les contemplations et qui est magnifié dans les merveilles, % sur le même plan
\bible{Mt|13:46}{la perle de grand prix} de la virginité,
\ctrad{l'émeraude}{\lit{le béryl}.} choisie de la pureté, % littéralement : le beryl
le pur bijou de la tempérance,
celui qui a le \bible{IICo|13:3}{Christ dans ses propres paroles},
l'\bible{Ac|9:15}{instrument de la louange} de Dieu,
celui qui abandonne le monde et qui en lui \bible{Ph|3:8}{considère tout comme déchet},
\biblefr{pour qu'au Christ} seul \bible{ITm|1:17+Tb|13:7,11}{soit donné le royaume des siècles},
celui qui  prend sa propre croix\bibleall{Mt|16:24+Mc|8:34+Lc|9:23} sans aucun doute et qui suit le Christ avec ardeur comme un véritable disciple, 
le chasseur des démons et l'assaillant du diable,
celui qui déambule aux quatre coins de la terre habitée par la marche sans fatigue comme un bon ouvrier et ami du maître, 
et qui à tous les peuples propage la foi du Christ à travers l'Évangile,
celui qui se prépare lui-même pour le séjour de la trinité consubstantielle et le temple de Dieu engendré auprès de tous.
\np{Barnabé}, \reprise{ornement}le vénérable ornement des \npe{Chypriotes}\reprisef{ornement} et l'invincible défenseur de la terre habitée,
celui qui aime abondamment le Christ, qui  chaque jour  pour lui consacre son âme et qui règne avec lui aux siècles des siècles. %%% !!!! pas très orthodoxe. Voir  ce qu'il en est pour d'autre saint

%l. 132
Le discours a travaillé à exalter ce divin et trois fois regretté apôtre \np{Barnabé} par les éloges de regrets, et il n'a pas encore atteint le prélude.
Car le  merveilleux  se trouve inaccessible aux louanges. %?
C'est pourquoi négligeant %
ce qui semble un inabordable discours d'éloges,
 je laisserais  de côté  quelques un des petits détails pour nous tirés des Stromates et d'autres anciens ouvrages  concernant sa manière de vivre et ses vertus  à votre piété personnelle, et ainsi nous poserons un terme à l'énonciation,
abandonnant  à l'Écriture divinement inspirée le soin de couronner magnifiquement la tête de celui qui est digne d'être chanté. %%%% Absolument à relire
Car elle dit : \enquote{\np{Barnabé} \bible{Ac|11:24}{était un homme bon, rempli de l'Esprit saint et de foi.}}
Que pourrait-on dire de lui qui ne serait égal ou semblable en quelque façon ?

\saut

\reprise{provenait}Ce trois fois bienheureux provenait\reprisef{provenait} donc \reprise{Levi}de la tribu bénie de \np{Lévi}\reprisef{Levi}\bibleall{Ac|4:36}, 
il était originaire de celle de \np{Moïse} et d'\np{Aaron}\bibleall{Ex|2:1-2} % absent des actes, voir si l'iconographie détail plus
les grands prophètes de Dieu et les premiers du peuple de \npe{Qehath}\bibleall{Ex|6:18-20},
\reprise{Samuel}selon la race, de la parenté de \np{Samuel} le prophète\reprisef{Samuel}\bibleall{ICh|6:1-13}. 
\reprise{ancetre}Les ancêtres de celui-ci, retenus dans la région de \nl{Chypre}\bibleall{Ac|4:36} à cause de circonstances de guerres, %%% quelle guerre auraient pu faire que des juifs soients immigrés à Chypre ? l'auteur pense-t-il à qq chose de son tps ? voir également les variantes textuelles imporantes
y habitaient avec bienveillance\reprisef{ancetre}. %%% on remarque le contraste : ils sont contraint, mais restent bienveillants
Ils étaient \bible{Ac|22:12}{pieux selon la Loi} et tout à fait riches
 \reprise{champ}c'est aussi pourquoi ils avaient à \nl{Jérusalem} une fortune suffisante et un  champ très magnifique proche de la ville\reprisef{champ}, qui n'était pas seulement orné d'arbres fruitiers de toute sorte% fertiles plantes
mais qui par sa grandeur était aussi  le plus visible des domaines.
Car depuis que les enfants des \npe{Hébreux}, qui recevaient la prophétie de manière charnelle, avaient entendu le prophète \np{Isaïe}  dire : \enquote{\bible{Is|31:9|LXX}{Heureux celui qui a une descendance à \nl{Sion} et des possessions à \np{Jérusalem}}}, chaque personne qui en avait les moyens possédait un bien à \nl{Jérusalem}.

\reprise{Chypre}Lors de la naissance de de ce juste à \np{Chypre}\reprisef{Chypre}, comme ses parents \bible{Ex|2:2}{virent qu'il était beau}  à cause de Dieu, 
ils l'appelèrent aussitôt \np[Barnabé]{Joseph}\bibleall{Ac|4:36}, jugeant l'enfant digne du nom du patriarche\indexnp[le patriarche][]{Joseph} ; 
cependant que la noblesse des mœurs concurrença celle qu'il tenait de son nom.
Car Joseph signifie \enquote{ajout de Dieu}\bibleall{Gn|30:24}.
En effet le juste tint l'ajout de la grâce d'auprès de Dieu afin qu'il atteigne la perfection apostolique.
Joseph signifie également \enquote{gloire de Dieu}. %%% fantaisiste ? l'éditeur n'a pas trouvé en tt cas
Car il devint gloire de Dieu par son noble mode de vie.
Et que personne ne tienne ce discours pour une  exagération, 
mais qu'il croit qu'il relève de la divine écriture.
Du moins \np{Paul} dit : \enquote{\bible{ICo|11:7}{L'homme ne doit pas se voiler la face, étant image et gloire de Dieu.}}
Si donc \bible{Ac|9:15}{l'instrument de le l'élection} appelle l'homme commun \bible{ICo|11:7}{image et gloire de Dieu}, comment quelqu'un parlerait-il de l'homme très parfait selon Dieu ?

\reprise{education}Lorsque \np{Barnabé} devint plus grand, ses parents \bible{Lc|2:22}{le montèrent à \nl{Jérusalem}}
et le confièrent pour qu'il étudie avec précision la loi et les prophètes  \bible{Ac|22:3}{aux pieds de \np{Gamaliel}} :% pas dans les actes
il avait pour condisciple \np{Paul}, qui s'appelait encore \np[Paul]{Saül}\reprisef{education}.
\np{Barnabé} progressait ainsi chaque jour dans la connaissance\bibleall{Lc|2:52} et dans toute  vertu :
en vérité on ne  l'avait pas encore placé pour le service quotidien des Lévites à cause de l'insuffisance de l'âge, car il était toujours un jeune homme.
\bible{Lc|2:37}{Il ne s'écartait pas du temple, participant au culte par des jeûnes et de prières nuit et jour.}
\reprise{recitation}De cette façon il récitait la loi et le reste des écritures, comme s'il n'avait pas besoin de se  souvenir des lettres\reprisef{recitation} ; il en avait ainsi un amour silencieux, comme une mère de prudence. 
Il fuyait les bavardages nuisibles, les ayant en horreur, demeurant une offrande pure, parfaite, et non souillée : en tout il était glorieux par la vertu. %191 ressemble à certaisn discours du pro de jc. lieux commun ...


En ce temps là il arriva que le Seigneur se rendît à \nl{Jérusalem}, \reprise{paralytique}qu'il guérît le paralytique à la porte des brebis\reprisef{paralytique}\bibleall{Jn|5:1-9}  et qu'il accomplît dans le temple de nombreux autres signes et prodiges\bibleall{Mt|21:14}.   %%% comment rendre le kairos ? et παραγιγνομαι -> pas de subjonctif car c'est un fait certain
Voyant ces faits, le bienheureux était frappé de stupeur, et s'étant aussitôt approché il tomba à ses pieds et demandait à être béni par lui\bibleall{Lc|8:41}. 
Le Christ  pénétrant les cœurs, faisant connaître sa foi, l'accueillit avec bienveillance et  \textins{lui} \reprise{communiquer}fit part\reprisef{communiquer} des événements divins le concernant. 
Celui-ci brûlait totalement d'amour pour le \np{Seigneur}. 
Se rendant rapidement à 
\reprise{maisonMarie}\bible{Ac|12:12}{la maison de \np[mère de Jean-Marc][]{Marie}, la mère de \np[Jean-Marc]{Jean} qui était appelé \np[Jean-Marc]{Marc}},
celle-ci disait être sa tante --- c'est pourquoi on appelait \bible{Col|4:10}{\np{Jean-Marc} Marc le cousin de \np{Barnabé}} ---,
il  lui dit\reprisef{maisonMarie} :
\enquote{Ô femme, dit-il, vois ici ce que \reprise{desirvoir}nos pères ont désiré voir\reprisef{desirvoir}\bibleall{Mt|13:17},
car voici que \biblefr{\np{Jésus}}, un certain \bible{Mt|21:11}{prophète originaire de \nl{Nazareth} de \np{Galilée}} est au temple accomplissant des prodiges extraordinaires, et ainsi comme il apparaît à beaucoup, \biblefr{il est} le Messie, \bible{Mt|11:14}{\reprise{doitvenir}celui qui doit venir\reprisef{doitvenir}}.}
\reprise{visitetemple}Après avoir entendu ces faits  et laissé tomber ce qu'elle avait en main, la femme étonnée se rendit au temple de Dieu, et après avoir vu le seigneur et maître du temple, elle se jeta à ses pieds\bibleall{Jn|11:32}, demandant et disant :
\enquote{Seigneur, \biblefr{si j'ai trouvé grâce à tes yeux}, pars d'ici pour aller dans la maison de ta servante,
et \biblefr{bénis}  tes domestiques \bible{Gn|30:27}{par ta venue}.\reprisef{visitetemple}
}
Le Seigneur se pencha sur sa prière : 
s'étant réjouit, elle lui montre la chemin vers l'étage supérieur de sa maison.
%l. 215
Donc à partir de ce jour là, lorsque le Seigneur venait à \np{Jérusalem}, 
c'est là qu'il dormait avec ses disciples, 
\reprise{paque}c'est là qu'il fit la Pâque avec ses disciples\reprisef{paque}\bibleall{Mt|26:17-18+Mc|14:12-15+Lc|27:7-12},
c'est là qu'il enseigna ses disciples par l'échange des mystères ineffables.
\reprise{cruche}Car un parole vint à nous depuis les anciens :
\biblefr{celui qui porte la cruche d'eau}
que le Seigneur a désigné à ses disciples pour qu'ils le suivent\bibleall{Mc|14:23+Lc|22:10},
c'était \np[Jean-Marc]{Marc}, le fils de cette bienheureuse \np[mère de Jean-Marc][]{Marie}\reprisef{cruche} :
le Seigneur  dit comme un bon intendant \enquote{\bible{Mt|26:18}{\textins{allez} vers untel}},
comme disent les pères, faisant  ainsi connaître l'emplacement,
nous enseignant par énigmes que à quiconque se réconcilie avec lui même, le Seigneur demeure auprès de lui.% 4 juin 2013 16h20 iciq
Dans cet étage en effet, le Seigneur fit la Pâque\bibleall{Mt|26:17-18+Mc|14:12-15+Lc|22:7-12}. %%% plein de références à creuser. En fait pour Hesychaste / Eutyches c'est juste des références à des discours sur le lieux de la pâque =  SION. A verifier tt de même Hesychii Hierosolymitani Interpretatio Isaiae prophetae (Strasbourg fac th. Lyon SC, Paris IFEB)
Il  y apparut à ceux qui étaient autour de Thomas, ressuscité des morts\bibleall{Jn|20:24-29}.
\reprise{ascension}Là les disciples montèrent après l'ascension, venant du \nl{mont des oliviers},
étant au nombre de cent-vingt, parmi lesquels se trouvaient \np{Barnabé} et \np[Jean-Marc]{Marc}\bibleall{Ac|1:4-15}.\reprisef{ascension}
Là l'\np{Esprit} saint descendit en langues de feu sur les disciples le jour de la Pentecôte\bibleall{Ac|2:1-13}. 
Là a été installé la grande et très sainte \np{Sion}, la mère de toutes les Églises.%%% sans doute Sion = Eglise de Jerusalem. cf liturgie de st jacques, voir si Hesychii Hierosolymitani Interpretatio Isaiae prophetae parle de l'ecclesia de sion. Cf également Hesch. Const. Vita. Euty. 356-358 p. 14. Le paraclet est introuvable en europe. Chercher Der heilige Theodosios : Schriften des Theodoros und Kyrillos / hrsg. [mit Komm.] von Hermann Usener -> Geneve, Paris IFEB, Lyon SC, ENS ULM,Institut de Fr, Strasbourg BNF

\reprise{voyage}Alors \np{Barnabé} suivit le \np{Seigneur} qui retournait de \nl{Jérusalem} en \nl{Galilée}\reprisef{voyage}. 
Alors que de toute part de nombreuses personnes s'approchaient du \np{Seigneur}
et croyaient en lui, il dit aux disciples		: %%% tote + participial 0> redondant. comment traduire ? % l. 241
\enquote{\bible{Mt|9:37}{La moisson est abondante, mais les ouvriers peu nombreux}.}
Alors il consacra les soixante-dix disciples\bibleall{Lc|10:1}, 
parmi lesquels  le grand \np{Barnabé} se trouvait en premier, chef de chœur et choryphée. %%% déciédment !
Que personne cependant,
apprenant que les les apôtres lui ont attribué ce qualificatif\bibleall{Ac|4:36},
ne présume sans savoir
qu'il a reçu ce qualificatif sans inspiration divine. 
Car \np{Pierre} lui a attribué ce nom, 
à travers une révélation du saint \np{Esprit}
et aussi par une révélation du \np{Père} qui montre la divinité du \np{Fils}\bibleall{Mt|16:16-17} ;
et comme \bible{Mc|3:17}{\np[fils de Zébédée][]{Jacques} et \np{Jean} fils du tonnerre ont été nommés à partir de la qualité},
de même  aussi \np{Barnabé} est appelé   
\bible{Ac|4:36}{fils de la consolation}  à parti de la qualité,
devenu consolation de tous à cause d'une très sainte supériorité. 

\reprise{vente}Entendant le Seigneur enseigner et dire :
\enquote{\bible{Lc|12:33}{%
Vendez vos biens\reprisef{vente} et donnez une aumône,
faites vous des bourses qui ne vieillissent pas, 
un trésor perpétuel dans les cieux}},
\reprise{ventebis}n'hésitant en rien,
il abandonna les biens hérités de ses ancêtres --- car ils avaient quitté cette vie --- et qui étaient  de grand prix. 
Les vendant tous, 
il partagea avec ceux qui étaient dans le besoin, 
gardant seulement pour lui-même ce champ-là en vue de son propre entretien. 
Après la Mort, la Résurrection et l'Ascension du Seigneur et la venu Saint Esprit,
le divin \np{Barnabé} brûlant davantage d'amour envers le \np{Seigneur}, 
il rendit même ce champ-là et en tira une certaine somme, 
emportant tout,
\bible{Ac|4:37}{il  la déposa aux pieds des apôtres}\reprisef{ventebis},
n'en laissant rien du tout pour lui-même,
par l'exemple qu'il donnait lui-même, il excitait tout les  disciples à la même vertu. %%%  

\reprise{moquerie}\bible{Ac|9:21}{Il parlait et se disputait} avec \np[Paul]{Saül}, désirant le conduire à la même foi du \np{Seigneur}.  %%% variante intéressant sur aυτον et αυτου
En vérité \np[Paul]{Saül} s'appuyant sur la citoyenneté consciencieuse  selon la loi,
se moquait de \np{Barnabé}, comme une personne dupée, 
cependant qu'il blasphémait le \np{Seigneur}\reprisef{moquerie}, le surnommant \bible{Mt|13:55}{fils de charpentier}, ignorant, paysan et bandit. %%% Terme courant dans la polémique païenne pour désigné le Christ. Cf Martyrium Pionni 13.3
\reprise{Etienne}Quand il vit les  grands faits de prodiges qui eurent lieu à travers les apôtres et la foule du peuple qui  chaque jour prenait le parti de la parole de la foi\bibleall{Ac|2:43+Ac|5:12+Ac|5:14+Ac|11:24}, il se mordit l'âme. 
Jetant, avec \biblefr{les affranchis, les \npe{Cyrénéens} et les \npe{Alexandrins}}, une accusation contre \biblefr{\np{Étienne}} le grand rhéteur de l'Église, et ne pouvant \bible{Ac|6:9-10}{s'opposer à la sagesse et à l'esprit, par lequel il parlait}, il se dirigea vers la folie ;
et se remplissant de désir, il leva contre lui les déserteurs au sein du peuple, et l'ayant fait périr, il suscita \bible{Ac|8:1}{une grande persécution contre l'Église qui était à \nl{Jérusalem}}.\reprisef{Etienne} %%% susciter ou ranimer ? 
Mais alors que non préparé au combat il se rendait ainsi à \nl{Damas} pour faire du mal aux croyants, le Seigneur le renversa, le jetant face à terre\bibleall{Ac|9:3-5+Ac|22:6-8+Ac|26:12-15} ; et celui qui était tombé à terre, reconnu celui qu'il persécutait, et privé de la vue, il regarda tout à fait vers le sommet des cieux.
Retournant alors \biblefr{vers \nl{Jérusalem}}, il chercha à \biblefr{rejoindre les disciples}, et  \biblefr{tous le} fuyaient, craignant son importante cruauté\bibleall{Ac|9:26}. %%% midi arret on remarque que l'auteur zappe totalement le récit d'Ananias
\reprise{bienfaiteur}Le grand \np{Barnabé},  allant au devant de lui, dit :
\enquote{Jusqu'à quand, \np[Paul]{Saül}, rencontreras-tu \np[Paul]{Saül} ?
Pourquoi poursuis-tu ainsi le bienfaiteur  avec ardeur ?
Cesse de détruire\reprisef{bienfaiteur} le terrible mystère proclamé depuis longtemps par les prophètes\bibleall{Ac|9:21+Ga|1:13+Ga|1:23} et révélé en nos temps pour notre salut\bibleall{Rm|16:25-26+Col|1:26}.
}
Entendant cela, \np[Paul]{Saül} se jeta au pied de \np{Barnabé} avec beaucoup de larmes, il cria et dit :
\enquote{Viens avec moi, guide-moi vers la lumière et enseigne-moi la vérité ; car j'ai appris par l'expérience la vérité des tes paroles ; car moi j'ai blasphémé en parlant de lui comme \bible{Mt|13:55}{fils du charpentier}, maintenant je confesse qu'il est \biblefr{le fils} unique engendré \bible{Mt|16:16}{du Dieu vivant}, de même essence et recevant la même gloire et ayant le même trône,
co-éternel et partageant l'absence de commencement, \bible{He|1:3}{lui qui est un rayonnement de la gloire et une marque de l'être} du Dieu invisible,
\bible{He|1:3}{à la fin de ces jours-ci, pour nous et notre salut}, \bible{Ph|2:7}{il s'est anéanti, prenant la condition d'esclave}, c'est à dire d'un homme parfait issu de la sainte vierge et mère de Dieu \np[mère du Christ][]{Marie},
sans confusion, sans changement, sans division, sans séparation ;
et \bible{Ph|2:7-8}{reconnu comme homme à son apparence, il s'est abaissé lui-même, devenant obéissant jusqu'à la mort, et la mort sur la croix} ; lui qui est aussi ressuscité des morts le troisième jour et qui est apparu à vous, ses apôtres\bibleall{ICo|15:4-5}, 
\bible{Mc|16:19}{qui a été enlevé au ciel, qui s'est s'assis à la droite} du Père et qui \ctrad{reviendra}{\lit{revient}. Le symbole de Nicée-Constantinople utilise des participes présents.} %%% littéralement au présent
avec gloire juger les vivants et les morts\bibleall{IITm|4:1}  et son règne n'aura pas de fin\bibleall{Lc|1:33}.
}

Le divin \np{Barnabé} entendant cela  d'auprès du \bible{ITm|1:13}{blasphémateur et du persécuteur} fut stupéfié et sa face se couvrit de joie, comme s’il était une fleur matinale ;
l'embrassant et le baisant tendrement, il lui dit : \enquote{
Qui t'a appris, \np[Paul]{Saül}, à prononcer ces paroles-là divinement inspirées ?
Ou qui t'a fait confesser \np{Jésus} le Nazaréen fils de Dieu ? 
Ou à quel endroit a tu été instruit de cette exactitude des dogmes divins ?
} 
Celui qui était encore baissé et en larmes, avec de nombreuses componctions dit : %%%àd le encore pour rendre le parfait 
\enquote{Le Seigneur \np{Jésus} lui-même, celui qui de nombreuses fois a été dénigré et poursuivi par moi, le pécheur\bibleall{ITm|1:13}, m'a enseigné toutes ces choses ;
car pour ainsi dire \bible{ICo|15:8}{il est aussi apparu à moi, l'avorton}, et j'ai encore à l'oreille sa voix divine et douce ;
car ayant disposé tous les bienfaits, il me parla alors que j'étais étendu à faire pitié sur la face, me défendant bien plus que m'accusant : 
\enquote{\bible{Ac|22:7}{\np[Paul]{Saül}, \np[Paul]{Saül}, pourquoi me persécute tu ?}}
Et \biblefr{moi je répondis} avec frisson et crainte : 
\enquote{\bible{Ac|22:8}{Qui es-tu Seigneur ?}}
Le Seigneur me dit avec beaucoup de justice et de compassion :
\enquote{\bible{Ac|22:8}{Moi je suis \np{Jésus} le Nazaréen que tu persécute.}}
Et moi, stupéfié par sa patience indicible, je lui demandais :
\enquote{\bible{Ac|22:10}{Que ferai-je, Seigneur ?}}
Sur le champ il m'instruisit de tout cela et de plus nombreuses encore. 	%%%% rendre le toutwn. p. 339 
}

\reprise{loup}Alors le grand \biblefr{\np{Barnabé} prenant} sa main \bible{Ac|9:27}{le conduisit auprès des apôtres}, en disant :
\enquote{Pourquoi fuyez-vous le berger en pensant qu'il est un loup ?\reprisef{loup}
Pourquoi poursuivez-vous le timonier comme un pirate ? %%%% vocabulaire de la mer
Pourquoi méprisez-vous le plus brave comme un déserteur ?
Pourquoi congédiez-vous le témoin de mariage comme un usurpateur dans la chambre nuptiale ?  %%% autre terme ? +-moins littérale ?
Car l'Église se trouve être un lit nuptial spirituel, dont le Seigneur a désigné de sa propre main un  berger, timonier et protecteur.}  %%% l. 347. Qu'est ce que cela veut-il dire ? trouver autre chose que rencontrer8
\reprise{Damas}Alors \np{Paul}  \biblefr{leur décrivit} les  très  grands choses qui lui étaient arrivés \biblefr{sur la route}, et qu'il \biblefr{avait vu le Seigneur, lui avait parlé et comment à \nl{à Damas} il avait discouru librement au sujet du nom du Seigneur ;} et il était avec eux un enseignant de la parole du Seigneur à \nl{Jérusalem}\bibleall{Ac|9:27-28}.\reprisef{Damas}
\reprise{juifs}Il était tout à fait pesant pour les \npe{Juifs} que celui qui la veille persécutait \np{Jésus} le proclame ce jour là Fils de Dieu, et ils  \bible{Ac|9:23}{voulurent le faire périr}. 
Les apôtres écoutant ses enseignements l'envoyèrent prêcher dans sa propre patrie\bibleall{Ac|9:30}.\reprisef{juifs} %%% trouver terme

Cependant qu'ils avaient été dispersés sous la persécution de \np{Étienne}, qu'ils  s'étaient rendus à \nl{Antioche} et avaient annoncé l'Évangile du Seigneur \np{Jésus}, on avait entendu parler %mettre note sur le choix de trad pour euangelliw
d'eux à \nl{Jérusalem}\bibleall{Ac|11:19-20,22}. % verifier le sens du passif de akouw
\reprise{pasteur}Alors les apôtres envoyèrent  le bienheureux \np{Barnabé}, comme il était grand et puissant, être pasteur du troupeau du Christ pour la très saint Église qui se tenait là\bibleall{Ac|11:22+IP|5:2}\reprisef{pasteur} : \bible{Ac|18:27}{lorsque il y fut présent,  il se rendit très utile à ceux qui avaient eu la foi}, et à travers son divin enseignement, \reprise{peuple}\bible{Ac|11:24}{un peuple important fut établi auprès du Seigneur.}\reprisef{seigneur}
\reprise{traversee}De là, guidé par le Saint Esprit, il partit, \biblefr{traversant et évangélisant toutes les villes} et les régions \bible{Ac|8:40}{jusqu'à  son arrivé dans} la très grande \nl{Rome} : car lui-même proclama à \nl{Rome} l'évangile du Christ pour tous les autres disciples du Seigneur. 
Parce que de nombreuses personnes crurent et l'honorèrent excessivement, il quitta \nl{Rome}, rejetant la gloire des hommes, fuyant l'adoration\reprisef{traversee} : car de nouveau ce bienheureux avait en ce temps là l'avantage sur tous les hommes en ce qui concerne l'humilité ; et il l'a porta vers le plus haut degré ; et pour tous il rendit manifeste cela  à cause des histoires à son sujet. 
Car partout l'écriture divinement inspirée désigne comme  premier celui qui  s'est attaché à la seconde place, cédant les honneurs à ceux qui sont autour de lui,
imitant exactement le Seigneur qui a dit : \enquote{\bible{Mt|11:29}{Suivez-moi, car je suis doux et humble de cœur.}} %%%l. 380

Après que \np{Barnabé} eut  \reprise{atteindre}atteint\reprisef{atteindre} \nl{Alexandrie} d'\nl{Égypte} et qu'il y eut parlé du verbe de Dieu, il la quitta, \biblefr{parcourant} ensuite \bible{Ac|8:40}{toutes les villes, jusqu'à son arrivé à} \nl{Jérusalem}. 
\reprisef{peuple}De là, il partit à nouveau, atteignant \nl{Antioche} et \biblefr{voyant la grâce de Dieu} et l'Église qui s'accroissait, \bible{Ac|11:23}{il se réjouit} fortement.
Alors, \biblefr{il partit pour \nl{Tarse}, à la recherche de} \np{Paul}, \biblefr{et l'ayant trouvé, il se rendit à \nl{Antioche}} ; et ils y travaillèrent \biblefr{une année complète}\reprisef{peuple}, ils enseignèrent un peuple important %%% voir si important est le meilleur termne (cf plus haut) on a l'impression qu'il se répète par rapport au paragraphe précédent. Est-ce une manière de faire coincider txt des actes avec volonté ? 
et là ils appelèrent \bible{Ac|11:25-26}{les disciples chrétiens pour la première fois} %%% on a pas de rupture par rapport à avant _> impression que c'est B et P qui donne le terme de chrétiens
s'occupant auprès l'Église de ce qui concerne la répartition pour les pauvres, 
ils \reprise{Jerusalem}retournèrent à nouveau à \nl{Jérusalem} quatorze ans après la passion du sauveur %%% forme à l'indicative, contrairement à avant ->  voir comment rendre ? 
ainsi que l'écrit \np{Paul} lorsqu'il dit : 
\enquote{\bible{Ga|2:1}{Après quatorze ans, je me rendis à \nl{Jérusalem} avec \np{Barnabé}.}}\reprisef{Jerusalem} %%% pas de mention de Tite
\bible{Ac|9:25}{Ayant accompli le service} et \reprise{partage}serré \biblefr{la main droite} des apôtres, \biblefr{afin qu'}eux proclament \biblefr{aux nations} et que ceux qui étaient autours de \np{Pierre} proclament \bible{Ga|2:9}{à la circoncision}, 
ils se rendirent à \nl{Antioche}, prenant avec eux \np[Jean-Marc]{Marc}\reprisef{partage} comme serviteur. % voc à l'origine marin
Depuis \nl{Antioche}, \bible{Ac|13:4}{envoyés par le saint Esprit, ils allèrent à \nl{Chypre}} ;
et ils avaient parcouru entièrement l'île depuis \nl{Salamine} jusqu'à \bible{Ac|13:6}{\nl{Paphos}}, évangélisant et faisant des prodiges, lorsqu'ils aveuglèrent   \np{Élymas} et éclairèrent à la fois le proconsul\bibleall{Ac|13:6-12}, et ils \bible{Ac|14:21}{avaient enseigné un certain nombre de personnes} %%% voir comment rendre le ikanos -> avoir une cohérence
lorsque \biblefr{ceux qui étaient autour de} \np{Barnabé} \biblefr{prirent la mer depuis} \nl{Chypre} \bible{Ac|13:13}{pour se rendre en \nl{Pamphylie}}.\reprisef{partage} %%% on remarque ici que ce n'est plus ceux qui sont autour de Paul, mais ceux qui sont autour de Barnabé !
Lorsque \np[Jean-Marc]{Marc} vit les apôtres qui s'avançaient d'un même mouvement %%% verifier sur quoi porte le ὁμόσε
auprès de ceux qui étaient exposés au danger pour l'Évangile,
et que ceux-ci s'apprêtaient alors à être honorés en 
renonçant \textins{à la vie}  et qu'ils s'avançaient pour la guerre des incroyants, 
il fléchit devant ce qui devait arriver, 
comme il était jeune, faible et inachevé en ce qui concerne le mépris de la mort, 
 \bible{Ac|13:13}{il retourna} à \nl{Jérusalem} chez sa propre mère, abandonnant les apôtres.

Comme donc les apôtres \biblefr{\np{Barnabé}} et \np{Paul} avaient achevé \biblefr{le travail}, auquel \bible{Ac|13:2}{ils avaient été appelés}, laissant pour le Christ une assemblée nombreuse, ils se rendirent à \nl{Antioche}\bibleall{Ac|13:14} ;
de là ils furent envoyés par la grâce de Dieu vers les nations\bibleall{Ac|14:26}.
\reprise{remords}Il advint qu'ils eurent besoin de monter à \nl{Jérusalem} auprès des apôtres qui étaient autour de \np{Pierre}, à cause des pseudo-apôtres qui enseignaient aux disciples de se faire circoncire et de craindre la loi\bibleall{Ac|15:1-2}.
\np[Jean-Marc]{Marc} voyant qu'ils étaient craints de tous et qu'après de tels mauvais traitements et périls ils étaient robustes et en pleine forme, 
\bible{Mt|26:25+Lc|22:62}{se lamenta amèrement} en remarquant sa propre lâcheté ;
et alors qu'il avait honte de se présenter à \np{Paul}, il se présenta à \np{Barnabé} %%% chercher ce qu'on dit ce de Mc dans les Ac Canon et apocryphes.
avec des larmes et il tomba à ses pieds, 
demandant de recevoir le pardon pour les égarements,
offrant une prière pour avoir une assurance dans ce qui adviendra %%% pas sûr de comprendre tt à fait la construction / le sens -> A REVOIR
et disant \enquote{\biblefr{évidemment je suis capable} d'endurer tout types de mort 
\bible{Ac|21:13}{pour le nom de notre Seigneur \np{Jésus}}-Christ.}\reprisef{remords}
\np{Barnabé}, ce grand homme dans les vertus, courbé par les nombreux  larmes, l'exhortait à cesser ses pleurs, \reprise{pardon}en disant \enquote{\bible{Ac|21:14}{Que soit faite la volonté du Seigneur}.
Seulement, toi, sois prêt à accomplir tes promesses.}\reprisef{pardon}
Prenant par écrit les doctrines enseignés par les apôtres qui étaient à \nl{Jérusalem}, 
ils descendirent à \nl{Antioche} et ils étaient joyeux avec les frères. 
\np[Jean-Marc]{Marc} était accompagné de \np{Barnabé}, sans pouvoir parler librement\bibleall{Ac|15:30-35}.

Après cela, il paru bon à ceux qui étaient autour de \np{Barnabé} et  \np{Paul}
de traverser toutes les villes et d'aller visiter les frères.
\reprise{compagnon}Tandis que \np{Barnabé} s'approchait, il exhortait \np{Paul}
à ce que \np[Jean-Marc]{Marc}, 
qui se tenait prêt à combattre jusqu'à la mort pour la foi en Christ, 
devienne un compagnon de voyage pour eux ;\reprisef{compagnon} %%%l. l. 443
\biblefr{\np{Paul}} conseilla à son tour de \biblefr{ne pas le recevoir} auprès d'eux.
En conséquence, cela \biblefr{devint} une cause pour qu'\bible{Ac|15:38-39}{ils se séparent l'un de l'autre}, le Seigneur organisant ceci en vue de l'utilité : %προφασις traduit par cause plutôt que prétexte à cause de la précision du dernier morceau de phrase. On notera l'ironie d'une division qui conduit à la συμφερον
car Dieu se préparait à proclamer  \np[Jean-Marc]{Marc} berger et enseignant de peuples et de nations. %% voir comment j'ai traduit διδασκαλον avant
Mais que personne ne prenne souffrance  pour lui-même en  ce qui est dit dans l'histoire sacrée des Actes au sujet de \np[Jean-Marc]{Marc} : \bible{Ac|15:39}{un \ctrad{embrasement}{l'auteur joue dans ce paragraphe sur la double connotation de \textgreek{παροξυσμός}, méliorative et péjorative. La meilleure traduction serait sans doute \enquote{excitation}. Malheureusement l'auteur utilise l'expression \textgreek{παροξυσμὸς ἀγάπης}, dont une traduction littérale risquerait d'avoir des connotations totalement ignorés du texte grec.} advint}.
En effet les apôtres du Christ n'ont pas été porté vers la \reprise{souffrance}souffrance de colère et de désir,  loin s'en faut.
Comment donc en serait-il pour ceux-ci qui \bible{Ga|5:24}{ont crucifié la chair avec les passions et les désirs}, poussant des cris d'une voix forte vers tous ceux qui croient en Christ et disant :\reprisef{souffrance}
\enquote{\bible{Ep|4:31}{Que toute amertume, colère, désir, cri, blasphème disparaissent de chez vous avec tout mal.}} %%% pas sur de comprendre bien le πως γαρ, surtout que pas de pt de suspension.
Car nous avons appris de l'Écriture divine la diversité des embrasements, car elle dit :
\enquote{\bible{He|10:24}{Aussi, veillons les uns sur les autres pour l'embrasement d'amour et des bonnes œuvres.}}
\reprise{zele}\reprise{zeleminus}\reprise{embrasement}\biblefr{Il advint} donc \bible{Ac|15:39}{un embrasement} %%% notons le jeu sur παροξυσμος, trouver comment le rendre en francais
pour les apôtres,\reprisef{embrasement} c'est à dire un zèle de Dieu.\reprisef{zeleminus}
Car tandis que \np{Paul} recherchait une rigueur,
se faisant remarquer par la perfection apostolique
\np{Barnabé} révérait l'amour de l'humain :\reprisef{zele} %%% l'éditeur met en italique mais sans preciser ref biblique
\reprise{prenant} donc \bible{Ac|15:39}{\np[Jean-Marc]{Marc}, il navigua jusqu'à \nl{Chypre}},
traversant tout\bibleall{Ac|14:21} et instruisant un peuple conséquent, il arriva à \nl{Salamine}. %%%% on suppose que le debarquement se fait à Paphos ? Depuis Antioche ? ou alors l'auteur n'a pas le sens geographique ? mais cela m'étonnerait car il utilie les 2 calendriers 
Et là, il s'occupait à faire des prodiges et à \bible{Lc|9:2+Ac|28:31}{proclamer le royaume de Dieu}\reprisef{prenant}, et une \biblefr{foule} nombreuse \bible{Ac|11:24}{prit parti pour le Seigneur}.
\bible{Ac|17:17+Ac|18:4}{Chaque Sabbat, il discutait avec les juifs dans la synagogue} \bible{Ac|18:28}{démontrant à partir des Écritures que Jésus était le Christ.}
Tous le recevait avec bienveillance, à cause de la grâce divine qui se dégageait  de sa face :
car son apparence était angélique et sa figure ascétique.
Il avait les sourcils qui se joignaient %%% signe de beauté d'après le gaffiot
les yeux d'un regard brillant,
il n'avait pas un regard effrayant, mais avec circonspection il inclinait la tête vers le bas %%% ?
une auguste bouche et des lèvres de belle apparence qui répandaient une douceur de miel --- 
car jamais il ne laissait  échapper de son inutile à son besoin --- 
une démarche ordonnée et sans vaine gloire, 
et d'une manière générale l'apôtre  \np{Barnabé} était entièrement et totalement une stèle pure du Christ, brillant de toute vertu. %% verifier ordre des mots

\reprise{Syrie}Tandis qu'il restait dans la ville de \nl{Salamine} et qu'il \bible{Ac|18:11}{enseignait la parole de Dieu},
\biblefr{des juifs} survinrent de \nl{Syrie}\reprisef{Syrie} et \biblefr{ils s'opposèrent à ce qu'}il \bible{Ac|13:45}{avait dit en blasphémant} ; %%% on remarque que les juifs viennent de Syrie, ce ne sont pas les autochtones. Or on avait des juifs avant qui étaient bien.
ils rassemblèrent une foule contre lui, \bible{Ac|14:19|varia lectio}{disant qu'il ne disait rien de vrai}, mais aussi que le \np{Jésus} dont il parlait  était un vagabond quelconque et  un ennemi de dieu, qui repoussait la loi, les prophètes et le sabbat\bibleall{Mt|5:17+Jn|7:12,47} ; et ils étaient à la recherche 
d'une opportunité de le réfuter\bibleall{Mt|26:16+Lc|22:2,6}.
Le saint apôtre du Christ \np{Barnabé}, réuni avec tous frères, \biblefr{leur dit} : \enquote{\bible{Ac|20:18}{Vous, vous vous savez comment j'ai été avec vous tout le temps} \bible{Ac|20:32}{avertissant chacun} et \biblefr{exhortant à demeurer} dans la grâce et \bible{Ac|14:22}{dans la foi} de notre Seigneur \np{Jésus}-Christ, \bible{Lv|22:31}{à conserver ses prescriptions}, \bible{Lv|26:3}{à les accomplir} et à \bible{Jb|1:1,8}{s'écarter de toutes mauvaises actions} :
\bible{IICo|5:10}{car il faut que tous nous comparaissions au tribunal du Christ\bibleall{Rm|14:10},
afin que chacun reçoive en fonction de ce qu'il en fait dans le corps, soit de bien, soit de mal
}
\bible{ICo|7:31}{car la figure de ce monde passe} et le Seigneur \biblefr{est sur le point de venir} du ciel \bible{IITm|4:1}{juger les vivants et les morts}. % 497
Donc ne soyez pas insouciants, sachant que \biblefr{que vous ne connaissez pas l'heure} 
où notre Seigneur \bible{Mt|24:44+Lc|12:40}{viendra}. 
\biblefr{Supportez} donc \bible{IITm|4:5}{la souffrance} et par l'espérance \bible{Jc|5:8}{affermissez votre cœur, car la venue du Seigneur s'est approchée.} 
\biblefr{Souvenez vous} comment  \biblefr{je vous ai} grandement \bible{Jn|16:4}{parlé}, %% trouvez une syntaxe pour combiner citation plus allusion 
disant qu'il reste peu de temps à la vie présente, soit pour les joies, soit pour les peines,  
et que tout passera rapidement, 
alors qu'arrive le siècle où  de manière identique tout sera éternel et sans fin :
car le royaume des cieux ne passera jamais, de même le jugement n'a pas de fin,
mais toujours il persiste, immortel et incessant, châtiant les pécheurs.
\biblefr{Empressez-vous} donc d'\biblefr{être trouvés} irréprochables et \bible{IIP|3:14}{sans tâche} en ce jour-là,
afin que vous ne tombiez pas dans cet géhenne immortelle et sans fin.
Souvenez-vous \bible{Ac|15:12}{quels grands signes et merveilles le Seigneur à fait} en vous à travers moi, son serviteur, et priez en ma faveur :
\bible{IITm|4:6}{car moi je suis déjà versé en sacrifice et le moment de ma libération est arrivé}
\bible{IIP|1:14}{ainsi que notre Seigneur \np{Jésus}-Christ me l'a fait connaître}
\biblefr{J'ai mené le beau combat,
j'ai achevé la course, j'ai gardé la foi},
maintenant \bible{IITm|4:7-8}{la couronne de justice m'est réservée,
pas seulement à moi, mais aussi à tous ceux qui}
combattent à cause de son nom.
} %%% jolie reprise du discours de Paul. 

\bible{Ac|20:36-37}{Et lorsqu'il eut dit cela, il s'agenouilla avec eux tous ; chacun d'entre eux eut des lamentations},
l'apôtre leur ayant dit \enquote{\bible{IITm|4:6}{le moment de ma libération est arrivé.}}
\reprise{eucharistie}Prenant un pain et une coupe, et exécutant tous le rite,  \np{Barnabé} participa avec les frères à l'eucharistie des mystères.\reprisef{eucharistie} %%% traduire autrement ?
Et après cela \reprise{Marc}prenant \np[Jean-Marc]{Marc}\reprisef{Marc} et \bible{Ac|23:19}{se retirant à l'écart}, il lui dit :
\reprise{Marcnom}\enquote{Aujourd'hui, il me faut être rendu parfait dans les mains des juifs infidèles : toi, en sortant au dehors de la ville par l'Ouest tu trouvera mon corps ; % Est-ce par l'Ouest ou bien "au coucher du soleil" ? En plus kata + genitif ne correspond à rien dans les 2 cas
et après l'avoir enterré, quitte \nl{Chypre}, rend toi auprès de \np{Paul} et reste avec lui, jusqu'à ce que le Seigneur organise ce qui te concerne, car  ton nom est sur le point  d'être magnifié dans toute la terre habitée.}\reprisef{Marcnom}

%%% voir // avec les AcBar.
\reprise{mort}Après cela, \np{Barnabé} se rendit dans la synagogue et il enseigna aux juifs, 
\biblefr{tentant de les persuader à propos du Seigneur} \bible{Ac|28:23}{\np{Jésus}} que \bible{Ac|9:22+Ac|17:3}{celui-ci est le Christ} \bible{Mt|16:16}{le fils du Dieu vivant}.
% pq l'auteur remplace-t-il l'asie par la syrie ? est-ce lié au conflit avec antioche ? rechercher les délimitations exactes à son époque
Remplis de colère, \biblefr{les juifs qui venaient} de \nl{Syrie}, \bible{Ac|21:27}{mirent les mains sur lui} en se levant et le placèrent dans une pièce sombre de la synagogue jusqu'au soir profond ; %%% trouver autrechose
l'ayant emmené dehors et torturé fortement, ils le conduisirent de nuit hors de la ville\bibleall{Ac|7:58} et là les criminels le lapidèrent ;\reprisef{mort}
\reprise{depouille}et allumant un grand feu, ils y jetèrent le bienheureux pour qu'on ne trouve pas même sa dépouille.
Par la providence de dieu, le corps de l'apôtre resta pur et le feu ne l'endommagea même pas. %%% dans Ac de Barnabé le sort de la dépouille est ≠. On est du reste un peu étonné de penser que ces actes eut pu servir à justifier auto cephalie
Selon sa prescription, \np[Jean-Marc]{Marc} le sortit en cachette de la ville par l'ouest %%% cohérence ?  ils sont deja sorti de la ville. même question : est-ce par l'ouest où bien au coucher du soleil (sachant que c'est déjà le soir profond)
avec quelques frères,
ils transportèrent la dépouille de saint \np{Barnabé} et l'ayant enterré\reprisef{depouille} dans une grotte située à cinq stades, %%% voir la distance, voir les fouilles sur le tombeau de 
ils se retirèrent en \bible{Ac|8:2}{poussant  d'importantes lamentations à son sujet}.

\biblefr{Il y eut à ce} moment-\biblefr{là} \biblefr{une grande persécution sur l'Église qui était à} \nl{Salamine} et \bible{Ac|8:2}{tous se dispersèrent} de part et d'autre ; %%% allos αλλακου ?
et le tombeau du saint apôtre \np{Barnabé} devint désormais ignorée.
\reprise{PierreRome}Quittant la région de \nl{Chypre}, \np[Jean-Marc]{Marc} alla auprès de \np{Paul} à \nl{Ephèse}
et il lui annonça quelle avait été la fin de \np{Barnabé} ;
lorsqu'il entendit cela, il fondit en de grandes larmes à son sujet, il retint \np[Jean-Marc]{Marc} auprès de lui.
Après cela, alors que \np{Pierre} partait à \nl{Rome} suivant une révélation de Dieu, il prit \np[Jean-Marc]{Marc} avec lui, l'adoptant pour ainsi dire\bibleall{IP|5:13}.\reprisef{PierreRome}
Là il composa l'histoire évangélique :
en lisant et améliorant celle-ci, \np{Pierre} sut qu'elle était divinement inspiré. 
Et après avoir imposé les mains à \np{Marc}, à très juste titre il l'envoya 
 à \nl{Alexandrie} d'\nl{Égypte}, de \nl{Libye} et de la \nl{Pentapole}. %%% qu'elle est cette pentapole 
Lorsqu'il fut arrivé, il leur \biblefr{annonça l'Évangile} du Christ \bible{Ac|11:20-21}{\np{Jésus} :
un grand nombre se convertit, croyant au Seigneur}.
\reprise{martyrMarc}Devenant  pour eux enseignant de la parole de Dieu pendant neuf ans, il finit en martyr et reposa à \nl{Alexandrie}\reprisef{martyrMarc}.

\saut

Longtemps après, alors que le christianisme se répandaient et que des empereurs chrétiens régnaient sur l'Empire romain, \bible{Ac|19:11}{Dieu fit des miracles qui n'étaient jamais arrivés} dans le lieux où la dépouille du saint apôtre et noble martyr \np{Barnabé} avait été mis de côté :
\biblefr{car beaucoup  d'esprits impurs}, lorsqu'ils passaient dans ce lieu, \biblefr{Ac|8:7}{sortaient de ceux qui étaient possédés en poussant de grands cris, beaucoup de paralysés et de boiteux}, \bible{Mt|4:24}{des oppressés par des maladies diverses et des tourments} venaient se coucher dans ce lieu, ils \biblefr{guérissaient} tous ; et il y eut une grande \bible{Ac|8:7-8}{joie dans la ville} de \np{Salamine}.
Et qu'en ce lieu il y avait une certaine puissance divine en action, on le savait, 
mais quelle était la cause d'une si abondante grâce, on ne le savait pas.
Les habitants de cette région appelaient l'endroit \enquote{le lieu de la guérison}.

Lorsque le bienheureux \np{Marcien} cessa de détenir le pouvoir, \np{Léon} reçut le titre d'Empereur par une désignation divine : 
son gendre, un certain \np{Zénon},  Ιsaurien par la race, était son lieutenant,
il régna aussi après lui. %%% la basileia> comment traduire ?
En ce temps là, un certain diable fut trouvé parmi les moines dans le très pur monastère des Veilleurs , ainsi que \np{Judas} parmi les apôtres, son nom était \np[le foulon][]{Pierre},  %%% Pierre le Foulon, patriarche miaphysite d'Antioche https://fr.wikipedia.org/wiki/Pierre_le_Foulon, 
il travaillait comme foulon ;  rejetant l'enseignement du saint concile de \nl{Chalcédoine}, il enseignait les dogmes d'\np{Eutychès}.
Les membres de ce saint monastère l'en chassèrent, parce qu'il était un destructeur, un corrupteur et un ennemi des dogmes apostoliques. %%% on a pas de particule, mais genetif -> orig geog. est-ce que monh et monastherios peut etre traduit par des termes ≠ ? couvent est-il adapaté ? il me semble que c'est uniquement pour les ordres mendiants.
Lorsqu'il arriva à \nl{Constantinople}, il rechercha avec ardeur la vie des flatteurs, en se rendant de maison en maison et en se remplissant le ventre.
Trouvant certain d'entre eux qui étaient au faîte de son ignoble hérésie, il les suivit et  grâce à eux il devint un familier par alliance  de l'empereur ; %%% bizarre l'histoire de l'alliance ...
et prenant une apparence de piété, il était sans arrêt avec lui, ne craignant pas de rendre publique sa propre impiété.
Lorsque \np{Zénon} partit pour les contrées de l'Est, le Foulon\indexnp[le foulon][]{Pierre} l'accompagna jusqu'à \nl{Antioche}.
Trouvant là beaucoup des Apollinaristes,  il parla vigoureusement au sujet de la postérité du patriarche, soulevant contre lui les membres indisciplinés du peuple, injuriant le concile de \nl{Chalcédoine} et invoquant le patriarche \np{Nestorius}.
Étant à l'origine de beaucoup de troubles et  de tumultes à \nl{Antioche}, 
le Foulon\indexnp[le foulon][]{Pierre} s'approcha du gouverneur, il lui dit avec malice : 
\enquote{Si l'évêque de cette ville n'est pas remplacé, je n'aurais pas moyen de diriger calmement le peuple.}, 
au même moment, il lui donna aussi une grande quantité de pièce d'or, quoiqu'il eut réussi dans sa requête : car il mit pour lui aussi en lumière les secrets de ses propres ténèbres.
À ce moment, le Foulon\indexnp[le foulon][]{Pierre} persuada tous ceux qui souffraient des mêmes maladies que lui, ainsi que quelque acteurs et d'autres mauvais hommes \ctrad{du peuple}{\lit{des peuples}.}, 
et il en référa à l'empereur, mentant très habillement contre l'évêque. 
Mais aucune de ses ruses ne fut utile, l'empereur défendant les dogmes apostoliques.

Lorsque celui-ci abandonna ainsi les choses terrestres  et qu'il laissa un enfant en bas-âge pour l'empire,  \np{Zénon} devint héritier de l'empire. %%% allusion à Léon 2, âgé de 4 ans lors du décés de Léon 1
Aussitôt, les antiochiens sus-cités apportèrent une requête à l'empereur, demandant que Le Foulon\indexnp[le foulon][]{Pierre} soit évêque. 
Celui-ci le devint donc aussi, l'or convainquant tous ceux qui étaient à la cour impériale de plaider sa cause.
Aussitôt donc il anathématisa le saint concile de \nl{Chalcédoine} avec une imposition publique : achevant de donner satisfactions aux Apollinaristes, atteint de théopaschie, songeant à une très mauvaise innovation pour dire dans le Trisagion à la fin de l'hymne \enquote{Qui a été crucifié à cause de nous}.
Les saints évêques et nos pères ont enseigné cela, se dressant courageusement contre sa mauvaise doctrine ; 
et ils essayèrent d'abord de l'interpeller par des écrits d'avertissement, mais lorsqu'ils virent qu'il contredisait la foi orthodoxe et bien plus s'enhardissait contre, tous les évêque à travers la terre habitée se manifestèrent contre lui et l'anathématisèrent.  


À ce moment-là une  parole surgit,  %%% comprendre sens
j'interrogerais volontiers ceux qui parmi nous ont reçu cette innovation, avec droiture de raisonnement et sans méchanceté de pensée.
À cause de quoi, mes frères, abandonnant dans ce domaine l'enseignement orthodoxe des pères, avez-vous reçu   l'innovation inventé par les hérétiques ? 
Il fallait se rendre compte que \bible{IIP|1:16}{ce n'est pas en suivant des fables trompeuses} 
que nos saints pères nous ont transmis de chanter cet hymne, 
mais à partir d'une révélation venue de Dieu,  
qui n'a pas été révélée une fois et une seconde unique fois,
ni dans une quelconque cachette, %%%%
mais au milieu de nous, 
pour tous le peuple de \nl{Constantinople} la capitale qui aime le Christ.,
à l'époque du notre pieux et trois fois bienheureux père et évêque \np{Proclus},
le très estimé parmi les enseignants.
Tous ce qui continuellement nous a   été divinement révélée  par la grâce divine est totalement et pleinement en vue du salut, 
il n'y a pas besoin en plus d'ajout ou de soustraction issues de syllogismes venant d'hommes,
selon ce qui est dit dans l'Écriture divinement inspirée :
\enquote{la \biblefr{
parole que moi je te prescrit aujourd'hui, prend} précieusement \biblefr{soin} de la \bible{Dt|13:1}{mettre en pratique, tu n'y ajoutera ni n'y enlèvera rien}.}
Car en réalité, il est dangereux et périlleux d'ajouter quelque chose aux oracles divins ou d'en enlever : car nous ne sommes pas des correcteurs de Dieu, mais des subordonnées. 
Et à ce sujet il y a beaucoup à  dire pour nous-même et à montrer que 
l'ignorance finale est ce qui démontre l'innovation de l'hérétique, et même si ce qui vient de l'hérétique n'est pas sensé.


C'est donc pourquoi \np[le foulon][]{Pierre} a été anathématisé, comme il est dit,
par tous les évêques et lorsque \np{Zénon} a fui le pouvoir impérial à cause du soulèvement de \np{Basilicus}, le fuyard se rendait dans un lieu inconnu. %%% trouver comment rendre la Basileia
Au moment où \np{Zénon} revint vers son propre pouvoir impérial, après avoir recherché \np[le foulon][]{Pierre} il le rétablit évêque d'\nl{Antioche}, tandis que les anathèmes posés sur lui n'ayant nullement été déliés, il ostracisa vers \nl{Oasis} le bienheureux patriarche \np{Calandion}. %%% Il semblerait que cette Oasis soit en Afrique. Chercher une histoire récente des patriarches d'Antioche. Par ex Hierarchia ecclesiastica orientalis : series episcoporum ecclesiarum christianarum orientalium / a cura di Giorgio Fedalto
Le Foulon\indexnp[le foulon][]{Pierre} ayant retrouvé le trône  par la force,
et puisqu'il avait renoncé au futur  et conservé aucun espoir, 
usa de la tyrannie pour assassiner, confisquer et bannir tous ceux qui voulaient pas communier à son impiété.
Mais tandis que je laisserais de côté ces choses,
qui sont nombreuses et ont besoin de leur propre récit, %%% conserver écrit ?
je marcherai vers ce qui est pressant, 
en montrant à tous l'immense grâce  du saint apôtre \np{Barnabé} et 
quel grand souci et sollicitude il a pour sa propre patrie.
Ainsi donc, Le Foulon\indexnp[le foulon][]{Pierre} ne s'est pas contenté des innombrables maux 
qu'il a commis au levant, 
mais il a aussi jeté ses mains sur les éparchies qui ne lui étaient en rien subordonnées :
car  il acheta une fois avec de l'or l'accord de l'empereur et de son entourage, et il a méprisé la loi de son Dieu 
--- et il a certes été disposé à faire toute sorte de mal aux \npe{Chypriotes} aimant Dieu, aimant la foi et orthodoxes, car conservant la pitié ancestrale, ils ne voulaient pas communier à son impiété ---,
et  entreprit de ravir le très saint, véritable et apostolique, depuis le  début et dès le commencement, siège de \nl{Chypre} et de se l'assujettir : %%% cf décision du concile d'Ephèse relative à Chypre (can.  VIII)
et il fait un recours auprès de celui qui gouvernait, un mensonge complet, puisqu'il dit que \biblefr{la parole de Dieu retentit depuis} \nl{Antioche}  \bible{ITh|1:8}{jusqu'à} \nl{Chypre} %%% comment on indique également que cf Ac 13, 1-5
et qu'il fallait que l'Église de \np{Chypre} dépende du siège de \nl{Antioche},
bien que le siège se trouve être apostolique et patriarcal. %%% pas sûr de comprendre le sens exact.
Mais en réalité,  l'hérétique apostat ne demeura pas caché, falsifiant les définitions du saint concile d'\nl{Éphèse} du temps de \np{Cyrille}.
Mais le héraut de la piété,  \np{Barnabé} le très saint parmi les apôtres, maudit sa sottise,  devenant à cette époque protecteur de la patrie.
Car lorsque on exhorta  l'évêque de \nl{Salamine} à rencontrer celui qui régnait, et à ce qu'il soit jugé face aux \npe{Antiochiens} devant le patriarche œcuménique, il fut engourdi par  à la peur, redoutant les machinations du Foulon\indexnp[le foulon][]{Pierre} : 
l'homme admirable était \np{Anthémios}, comme il était tous à fait très orthodoxe et illuminé par  une vie sans tâche\bibleall{Sg|4:9}, il y avait très peu de personne lui résistant dans la dialectique.
Ainsi donc, alors qu'il est dans l'embarras et se tourmente au sujet du voyage à l'étranger,
alors qu'il dort dans un endroit isolé, 
quelqu'un se place auprès de lui pendant la nuit,
avec une figure semblable à la divinité et brillant d'un fort éclat de lumière, 
il relève son habit de lumière comme il convient à un personne sainte, 
et il lui dit 
 : %%% tt cela est au présent
\enquote{Qu'est-ce qui t'a chagriné, évêque, quelle est cet moquerie à ton propos et \bible{Gn|4:6}{pourquoi ton visage est-il abattu ?}
Tu ne craindra aucun danger des adversaires.}
Et ayant dit cela, il s'éloigna de lui.
L'évêque qui avait peur s'étant réveillé, tomba sur la face avec beaucoup de larmes et demanda à Dieu : 
\enquote{Seigneur \np{Jésus}-Christ, \bible{Mt|16:16}{Fils du Dieu vivant},
si toi tu ménages cette Église --- et je sais que tu la ménages ---
et si cette vision vient d'auprès de toi, ordonne qu'une seconde et une troisième fois \ctrad{la vision}{terme différent mais synonyme du précédent.} me fortifie, afin que je sois assuré que toi tu es avec moi.}
Alors il acheva rapidement la prière, ne rencontrant personne.
\biblefr{La nuit suivante}, cet homme, dans cette tenue et dans cette apparence \bible{Ac|23:11}{lui apparut} à nouveau, lui disant :
\enquote{Je t'ai déjà dit qu'aucun de ceux qui te menacent ne se trouvera sur ta route, 
va maintenant avec empressement à la capitale, ne redoutant rien.}
Et ayant dit cela, il se retira.
En se levant l'évêque rendit à nouveau grâce au seigneur, ne disant rien à personne.
Il attendait encore la troisième vision, et la nuit qui suivit, cet homme se tint à nouveau auprès de lui, lui disant très sévèrement :
\enquote{Jusqu'à quand, \bible{Lc|1:20}{ne croiras-tu pas à mes paroles qui s'accompliront} en ces jours-ci ? Rends-toi avec empressement à la capitale, et tu retourneras avec gloire sur ton siège. 
Tu ne craindras aucun danger de tes ennemis, car à travers moi, son serviteur, Dieu sera ton bouclier\bibleall{Gn|15:1}.}
L'évêque répondit, ouvrant la  bouche\bibleall{Mt|5:2+Lc|1:64}: 
\reprise{reponse}\enquote{Car qui es-tu, monseigneur, qui tient ces paroles auprès de moi ?}
Celui-ci : \enquote{Moi je suis, dit-il, \np{Barnabé}, le disciple de notre Seigneur\reprisef{reponse} \np{Jésus}-Christ,
qui a été désigné par le saint Esprit pour l'apostolat des nations avec l'apôtre \np{Paul}\bibleall{Ac|13:2}, l'\bible{Ac|9:15}{instrument élu} ;
et \bible{Lc|2:12+IR|10:1+Is|37:30+Is|38:7}{ceci est pour toi un signe} : \reprise{vision}va, dit-il, hors de la ville à cinq stades à l'ouest 
au lieu dit de la guérison --- car c'est à travers moi que Dieu accomplit les prodiges en ce lieu ---, 
fouille à l'emplacement du caroubier et 
 tu trouveras une grotte et un cercueil dans celle-ci.
C'est là que mon corps tout entier est mis de côté, et un évangile autographe,  
celui que j'ai reçu de \np{Matthieu} le saint apôtre et évangéliste.\reprisef{vision}
Et lorsque tes adversaires plaideront en disant en \ctrad{long, en large et en travers}{\lit{de haut en bas}.} 
que le siège d'\nl{Antioche} est apostolique, toi aussi tu leur plaidera en retour : \enquote{Mon siège est aussi apostolique et j'ai un apôtre dans ma patrie.}}

Ayant dit cela, il partit : et après s'être levé et s'être prosterné devant le seigneur, l'évêque 
rassembla tous le clergé consacré et le peuple aimant le Christ, et il sortit en brandissant la croix vers le lieu qui lui avait été indiqué avec beaucoup de matériel.
Et après avoir fait une prière, il invita à creuser l'endroit ;  après avoir un peu creusé, ils trouvèrent une grotte, protégée par des pierres,  après les avoir roulées, ils trouvèrent la tombe, et après avoir enlevé son couvercle, ils trouvèrent la précieuse dépouille du saint et glorieux apôtre, \np{Barnabé}, exhalant d'une agréable odeur de grâce spirituelle.  
Il trouvèrent également \reprise{evangile}l'Évangile, posé sur sa poitrine\reprisef{evangile}.
Et après avoir avoir scellé la tombe avec du plomb, avoir fait une prière et s'être prosterné, ils se retirèrent, tandis que l'évêque plaçait en plus dans le lieu des hommes attentifs à louer Dieu par des hymnes le soir et le matin.

Prenant avec lui les insignes épiscopaux,  il se dirigea vers la capitale et fut reçu au palais épiscopal : sa venue fut révélée à l'empereur  et   il ordonna d'écouter de la bouche du patriarche  l'objet du litige lors du concile local et de s'entretenir des dispositions à prendre. %%% trouver ordre des mots
L'assemblée ayant été excitée et l'organisation bouleversée, ceux qui servaient lors du procès, comme on le présageait, dirent, selon leurs propres règles, que le patriarche était celui du siège d'\nl{Antioche} et qu'il était apostolique, et qu'il fallait que toutes les éparchies en dépendent.
Ils semblaient dire des paroles bienveillantes à ceux qui étaient présents. 
Le bienheureux \np{Anthémios}, un évêque mineur, rétorqua : \enquote{
Mais cependant le mien, ô excellents hommes, est un siège apostolique depuis le  début et dès le commencement,  %%% allusion au canon VIII du concile d'Ephèse
honoré par une liberté, et j'ai dans ma patrie un apôtre au corps intact, \np{Barnabé}, le trois fois bienheureux dans les saintes louanges.}
Ayant ainsi parlé, il n'y eut plus lieu à contradiction, les évêques confirmant le discours par un silence : et eux, muets de stupeurs, s'étaient tenu avec  honte lors du procès,  rejetant le beau discours de réponse.

L'empereur, ayant appris cela, envoya  avec empressement chercher  l'évêque de \nl{Chypre}, s'informant à propos de la découverte du saint apôtre \np{Barnabé}. 
Il ne cacha aucune vérité, racontant de suite  tout à l'empereur. 
Entendant cela, il se réjouit beaucoup, s'étonnant de l'abondante grâce de Dieu, qu'à l'époque de son règne il ait fait d'aussi importants prodiges.
Alors   il chassa d'auprès de lui l'évêque d'\nl{Antioche} par la peur, ordonnant à ceux qui restaient de ne surtout pas inquiéter l'évêque  de \np{Chypre} à cause de cette action. 
Ensuite il convoqua le bienheureux évêque \np{Anthémios} pour lui dire : 
\enquote{Puisque notre Seigneur \np{Jésus}-Christ a jugé bon de révéler son saint apôtre \np{Barnabé} au temps de ton épiscopat, ordonne de  ramener  rapidement pour moi à cet endroit l'évangile qui a été trouvé, en cela tu auras pleinement reçu de moi, ton fils, une grâce  qui sera absolument très grande :  %%% comment traduire χαριζω χαριν ? -> essayer de comprendre vraiment
car à partir de maintenant tu m'auras aussi  jusqu'à ce que tous écoutent ta paternité.}

Inclinant la tête, l'évêque envoya un des évêques qui était avec lui accompagné d'un très fidèle homme de l'empereur ; et après avoir pris l'évangile, ils retournèrent à \nl{Constantinople}, ils tenaient les tablettes en  bois de thuya(?). \reprise{jeudi}L'empereur les prenant, les baisant tendrement et les garnissant d'or, les déposa dans le palais et elles ont été conservés jusqu'à aujourd'hui.
Car le grand jeudi de Pâques on y lit l'Évangile de l'anniversaire dans la chapelle du palais.\reprisef{jeudi}
Après avoir  grandement  révéré l'évêque, l'empereur le renvoya à \nl{Chypre} avec beaucoup de biens et un ordre, lui commandant d'élever un sanctuaire pour le saint apôtre \np{Barnabé} au lieu même où sa précieuse dépouille a été trouvée.
Aussi beaucoup de Grands lui donnèrent de l'or pour la construction du sanctuaire.
Celui-ci, s'emparant de ce qui est à \nl{Chypre} et rassemblant une foule d'artisans et d'ouvrier, ne soutenait pas superficiellement la construction, mais élevait pour l'apôtre un sanctuaire tout à fait gigantesque,  brillant par les moyens utilisés, plus brillant par la diversité des arrangements, entouré à l'extérieur par un cercle de colonnes. 
Sur la \biblefr{colline} du sanctuaire vers le \bible{Ex|27:9}{sud-ouest}, il construisit une  grande cours qui avait quatre portiques :
après avoir construit  des maisonnettes d'un côté et de l'autre de la cours,  il confia aux moines qui accomplissent la liturgie divine dans le sanctuaire de demeurer dans celles-ci ;
amenant de loin  l'aqueduc,  il installa abondance de fontaines, 
construisant au milieu de la cours les plus belles citernes, de sorte que les habitants du lieu et les étrangers tirent abondamment profit de la source.
Il construisit aussi à cet endroit  beaucoup d'autres gites pour le repos des étrangers de passage : et il advint qu'on remarquât cette zone, imitant par ses richesses une quelconque  petite ville et tout à fait agréable.
Il transporta le saint cercueil de l'apôtre à droite de l'autel, arrangeant le lieu avec  une certaine quantité d'argent et des marbres : on décreta que soit établi selon chaque anniversaire le jour du glorieux souvenir du trois fois bienheureux apôtre devenu martyr \np{Barnabé} :  %%% H omet selon cet anniversaire
 pour les \npe{Romains} le troisième jour des ides de juin, %%% 11 juin 
 pour les \npe{Chypriotes} de \nl{Constantia} au mois de Mésore, qui est aussi le \ctrad{dixième}{le \textgreek{τοῦ καὶ δεκάτου} est de sens obscure. Nous avons traduit selon le plus probable, mais le mois de Mésore est le onzième dans le calendrier Égyptien.}, le \ctrad{onzième jour}{4 août.} %% Egyptiens 4 aout, Constantia = salamine -> manque bout de traduction
 pour les \npe{Asiates} assurément comme pour les habitants de \nl{Paphos} au mois de Plethypatos \ctrad{le dix-neuvième}{11 juin}	%% Et aussi chypre, Grumel p. 171 11 juin. ce qui explique Pq ce mélange de calendrier
 se rassemblant et accomplissant la liturgie spirituelle,
 pour la gloire du Père et du Fils et du Saint-Esprit,  à qui est la gloire pour les siècles des siècles. Amen.
 
\saut

Donc la découverte des reliques du saint-apôtre s'est ainsi déroulée :  
cependant que les prodiges qui chaque jour jaillissaient de sa sainte tombe,
si quelqu'un avait le désir de les écrire, pas même un volume entier, je pense, ne suffirait à cela\bibleall{Jn|19:25}. 
Nous donc qui accourons au souvenir de nos devanciers,  exaltons ce pentathlète de la piété combattant et couronné, béatifions-le, louons-le : 
car celui-ci a été planté \bible{Ps|52:10}{comme un un olivier en fleurs dans la maison de Dieu}, chaque jour il porte pour Dieu des fruits à l'odeur agréable ;
celui-ci est une gloire pour les empereurs, une joie pour les prêtres,  un transport de joie pour les peuples ;
celui-ci est une consolation pour les éprouvés, un secours pour les accablés,   une espérance pour les désespérés, une aide pour les découragés, une exhortation pour les étrangers, %%% ?
une guérison pour les malades, 
un gardien pour les convalescents, 
une eau vive de grâces spirituelles, %%% rupture dans l'ordre des mots
un rempart pour l'Église,
une fondation pour les orthodoxes,
une forteresse pour les croyants,
pour toute la terre habitée, une fierté. %%% ordre des mots
 %%% pas d'article dans les génitifs. Voir si tt cela a des // dans le texte du début
 
Mais ton éloge,  ô \np{Barnabé}, tout à fait heureux et trois fois bienheureux apôtre de Christ notre Dieu, dépasse toute pensée et parole humaine ;
désignant la sagesse de notre mendicité,  soit notre ambassadeur auprès de celui que tu as aimé, Christ l'unique Fils engendré et le verbe de Dieu, \biblefr{pour qu'il nous}  arrache du \bible{Ga|1:4}{présent monde mauvais} et que tu nous \biblefr{nous donnes} de trouver pardon des péchés,\biblefr{miséricorde} et compassions \bible{IITm|1:18}{en ce jour} effrayant, lorsqu'il viendra juger les vivants et les morts. %%% en note de bas de page, un autre passage à traduire eventuellement
Ton successeur qui a en charge ton très saint siège, celui qui partage les robustesses de ta foi,
le noble interprète de  ton orthodoxie, l'imitateur de ta vertu, le berger, notre père et grand-prêtre, celui qui est présent maintenant et entretient ta sainte mémoire, 
va supplier Dieu qu'il soit préservé par ton très saint siège  durant les nombreuses \ctrad{péripéties des temps}{\lit{changement des années}.}, %%% caractère politique ? ou pas ?
sanctifiant et faisant paître en paix son peuple, \biblefr{en} toute \bible{Eph|4:24}{piété et justice} ouvrant en ligne droite la parole de la vérité.
Préoccupe toi de toute ta patrie, comme toujours, maintenant aussi par tes saintes prières, la \bible{Ps|121:7}{conduisant loin de tous mal} et de \bible{Ps|141:9}{toute embûches  des ouvriers de l'iniquité}, 
afin \biblefr{qu'à l'époque présente nous vivions} paisiblement, \bible{IITm|2:15}{prudemment et pieusement}, \bible{Jude|1:21}{attendant la  miséricorde de notre Seigneur \np{Jésus-Christ} pour la vie éternelle},
qu'il advienne qu'unis nous la  rencontrions tous, par
grâce, compassions et amour pour l'homme de notre Seigneur \np{Jésus}-Christ,
avec qui le Père à la gloire avec le Saint Esprit, maintenant, à jamais et pour les siècle des siècles, amen. 


